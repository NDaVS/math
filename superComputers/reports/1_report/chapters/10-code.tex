\chapter{Выполнение работы}
\section{Код программы}
Для выполнения лабораторной работы был выбран ЯП Java \cite{сьерра2012изучаем}.

\begin{lstlisting}
	package ru.happines;
	
	import java.util.function.Function;
	
	@FunctionalInterface
	interface Delegate {
		void invoke(Function<Double, Double> f, double a, double b);
	}
	
	public class aboba {
		public static void sum(Function<Double, Double> f, double a, double b) {
			System.out.println("Sum = " + (f.apply(a) + b));
		}
		
		public static void multiply(Function<Double, Double> f, double a, double b) {
			System.out.println("Multiply = " + (f.apply(a) * b));
		}
		
		public static void main(String[] args) {
			Delegate del;
			
			del = aboba::sum;
			del.invoke(x -> x * x, 3, 4);
			
			del = aboba::multiply;
			del.invoke(x -> x + 2, 3, 4);
		}
	}
\end{lstlisting}

\section{Описание работы программы}

\begin{itemize}
	\item Создан функциональный интерфейс \texttt{Delegate} с методом \texttt{invoke}, принимающим функцию и два параметра типа \texttt{double}.
	\item Методы \texttt{sum} и \texttt{multiply} принимают функцию и два числа, выполняют соответствующую операцию и выводят результат.
	\item В методе \texttt{main} создается делегат \texttt{del}, которому присваиваются ссылки на методы \texttt{sum} и \texttt{multiply}.
	\item Вызов \texttt{del.invoke} выполняет соответствующую операцию с функцией, переданной в качестве аргумента.
\end{itemize}

\section{Результаты работы программы}

При запуске программы на консоли выводится:

\begin{verbatim}
	Sum = 13.0
	Multiply = 20.0
\end{verbatim}

\section*{Выводы}

В ходе лабораторной работы были изучены следующие аспекты работы с делегатами в Java:

\begin{itemize}
	\item Делегаты позволяют ссылаться на методы и вызывать их динамически.
	\item Функциональные интерфейсы в Java предоставляют удобный способ реализации делегатов.
	\item Применение делегатов повышает гибкость и расширяемость кода.
\end{itemize}