\chapter{Введение}
Современные реалии требуют обработки большого количества данных. 
При этом мощности компьютеров постепенно выхрдят на плато и нарушается закон Мура\cite{zubkova2014}. 
Что ставит перед человеком вопрос: 
"Как можно обработать гиганские массивы данных в относительно небольшой интервал времени?".

Решением этой проблемы является распараллеливание вычислений.
 Такой метод позволяет распределить вычисления на множество ЭВМ, 
 что, при правильной настройке, позволяет сократить время выполнения программ в разы.

Но это только верхушка айзберга. Для достижения этой цели требуется пройти путь с самого низа.

Рассмотрим ситуацию: имеется фреймворк для матмоделирования.
Каждая матмодель задаётся своим набором параметров и системой уравнений.
Для обработки такого поведения требуется механизм, который сможет принять метод, его параметры и произвести необходимыевычисления. 

И он есть - делегаты.

Цель лабораторной работы: изучить возможности применения делегатов в языке C\# / Java.
Задачи лабораторной работы:
\begin{itemize}
	\item освоить принципы работы с делегатами;
	\item освоить основные направления применения делегатов;
	\item изучить способы использования делегатов совместно с потоками.
\end{itemize}

