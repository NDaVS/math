%% Преамбула TeX-файла

% 1. Стиль и язык
\documentclass[utf8x]{G7-32} % Стиль (по умолчанию будет 14pt)
\usepackage[T2A]{fontenc}
\usepackage[russian]{babel}
\usepackage{rotating}
% Остальные стандартные настройки убраны в preamble.inc.tex.
\sloppy

% Настройки стиля ГОСТ 7-32
% Для начала определяем, хотим мы или нет, чтобы рисунки и таблицы нумеровались в пределах раздела, или нам нужна сквозная нумерация.


% Добавляем гипертекстовое оглавление в PDF
\usepackage[
bookmarks=true, colorlinks=true, unicode=true,
urlcolor=black,linkcolor=black, anchorcolor=black,
citecolor=black, menucolor=black, filecolor=black,
]{hyperref}

% Изменение начертания шрифта --- после чего выглядит таймсоподобно.
% apt-get install scalable-cyrfonts-tex

\IfFileExists{cyrtimes.sty}
    {
        \usepackage{cyrtimespatched}
    }
    {
        % А если Times нету, то будет CM...
    }

\usepackage{graphicx}   % Пакет для включения рисунков

% С такими оно полями оно работает по-умолчанию:
% \RequirePackage[left=20mm,right=10mm,top=20mm,bottom=20mm,headsep=0pt]{geometry}
% Если вас тошнит от поля в 10мм --- увеличивайте до 20-ти, ну и про переплёт не забывайте:
\geometry{right=20mm}
\geometry{left=30mm}

 \usepackage{booktabs}
 
% Пакет Tikz
\usepackage{tikz}
\usetikzlibrary{arrows,positioning,shadows}

% Произвольная нумерация списков.
\usepackage{enumerate}

% ячейки в несколько строчек
\usepackage{multirow}

% itemize внутри tabular
\usepackage{paralist,array}
\addto\captionsrussian{\renewcommand{\contentsname}{Оглавление}}

\makeatletter
\renewcommand{\l@section}{\@dottedtocline{1}{0em}{1.5em}} % Убирает отступы перед секциями
\renewcommand{\l@subsection}{\@dottedtocline{2}{1.5em}{2em}} % Убирает отступы перед подсекциями

\usepackage{hyperref}
\hypersetup{
    colorlinks=true,
    linkcolor=black,
    citecolor=black,
    filecolor=blue,
    urlcolor=blue
}


% Настройки листингов.
\include{listings.inc}

% Полезные макросы листингов.
\include{macros.inc}

\begin{document}
	
	\frontmatter % выключает нумерацию ВСЕГО; здесь начинаются ненумерованные главы: реферат, введение, глоссарий, сокращения и прочее.
	
	% Команды \breakingbeforechapters и \nonbreakingbeforechapters
	% управляют разрывом страницы перед главами.
	% По-умолчанию страница разрывается.
	
	% \nobreakingbeforechapters
	% \breakingbeforechapters
	
	aboba
	\newpage
	\pagenumbering{gobble} % Отключаем нумерацию
	\tableofcontents
	\pagebreak
	\pagenumbering{arabic}   % Включаем арабскую нумерацию страниц
	\setcounter{page}{3}  
	
	
	
	\section{Введение}
Одним из ключевых процессов в природе являются течения в жидких средах. 
Эти течения играют важную роль в переносе веществ и энергии из одной точки в другую. 
Примером могут служить течения в реках, морях и океанах, 
которые способствуют не только транспортировке питательных веществ, но и формированию экосистем.

Кроме того, аналогичные процессы происходят и в газообразной среде. 
В атмосфере воздух также движется, создавая ветры, которые переносят влагу, пыль и другие частицы. 
Эти воздушные течения влияют на климатические условия, распределение тепла и осадков, 
а также на миграцию животных и распространение семян растений.

Рассмотрим подобный перенос с помощью двумерного уравнения переноса.
	
	%\section{Цель и задачи проекта}

\textbf{Целью} данного проекта является разработка цифрового сервиса для анализа территорий с использованием данных дистанционного зондирования Земли (ДЗЗ) и сопутствующих геопространственных источников. Сервис предназначен для наглядного представления и оценки ключевых характеристик выбранной местности с учётом климатических, экологических и геофизических параметров.

Для достижения поставленной цели сформулированы следующие \textbf{задачи}:

\begin{itemize}
	\item Определение переченя параметров анализа (климат, рельеф, загрязнения, водные ресурсы и др.) и установка приоритетность их реализации на раннем этапе разработки;
	\item Выделуние сферы деятельности и группы людей для применения будущего сервиса.
	\item Поиск и верификация актуальных источников данных, включая открытые спутниковые платформы (например, Copernicus, NASA EarthData, USGS Earth Explorer);
	\item Предварительный анализ датасетов;
	\item Проектирование архитектуры программного решения, включая бэкенд, интерфейс пользователя и модуль интеграции с внешними API;
	\item Разработка минимально жизнеспособный продукт (MVP), демонстрирующий основные функции анализа и визуализации данных по выбранной территории.
\end{itemize}

	
	\mainmatter % это включает нумерацию глав и секций в документе ниже
	
	
	
\chapter{Выполнение работы}
\section{Код программы}
Для выполнения лабораторной работы был выбран ЯП Java \cite{сьерра2012изучаем}.

\begin{lstlisting}
	package ru.happines;
	
	import java.util.function.Function;
	
	@FunctionalInterface
	interface Delegate {
		void invoke(Function<Double, Double> f, double a, double b);
	}
	
	public class aboba {
		public static void sum(Function<Double, Double> f, double a, double b) {
			System.out.println("Sum = " + (f.apply(a) + b));
		}
		
		public static void multiply(Function<Double, Double> f, double a, double b) {
			System.out.println("Multiply = " + (f.apply(a) * b));
		}
		
		public static void main(String[] args) {
			Delegate del;
			
			del = aboba::sum;
			del.invoke(x -> x * x, 3, 4);
			
			del = aboba::multiply;
			del.invoke(x -> x + 2, 3, 4);
		}
	}
\end{lstlisting}

\section{Описание работы программы}

\begin{itemize}
	\item Создан функциональный интерфейс \texttt{Delegate} с методом \texttt{invoke}, принимающим функцию и два параметра типа \texttt{double}.
	\item Методы \texttt{sum} и \texttt{multiply} принимают функцию и два числа, выполняют соответствующую операцию и выводят результат.
	\item В методе \texttt{main} создается делегат \texttt{del}, которому присваиваются ссылки на методы \texttt{sum} и \texttt{multiply}.
	\item Вызов \texttt{del.invoke} выполняет соответствующую операцию с функцией, переданной в качестве аргумента.
\end{itemize}

\section{Результаты работы программы}

При запуске программы на консоли выводится:

\begin{verbatim}
	Sum = 13.0
	Multiply = 20.0
\end{verbatim}

\section*{Выводы}

В ходе лабораторной работы были изучены следующие аспекты работы с делегатами в Java:

\begin{itemize}
	\item Делегаты позволяют ссылаться на методы и вызывать их динамически.
	\item Функциональные интерфейсы в Java предоставляют удобный способ реализации делегатов.
	\item Применение делегатов повышает гибкость и расширяемость кода.
\end{itemize}
\chapter{Анализ математической модели}
\section{Аналитическое решение}
Очевидным точным решением является смещенный профиль волны.
\[
	u(x,t) = \phi(x_{0} - vt)
\]

\section{Численное решение}
Система (\ref{eq:sys}) является системой обыкновенного дифференциального уравнения в частных производных.
Для численного решения применяется методы конечных разностей\cite{Turchak2003}. Введём равномерную сетку:
\[
x_i = i \Delta x, \quad t^j = j \Delta t, \quad i = 0, \dots, N_x, \quad j = 0, \dots, N_t,
\]
где $\Delta x = \frac{L}{N_x}$, $\Delta t = \frac{T}{N_t}$. Обозначим $u_i^j = u(x_i, t^j)$.

Будем рассматривать три схемы решения: явную\cite{Turchak2003}, неявную\cite{Turchak2003} и "вверх по потоку" \cite{Patankar1984}.
В качетве профилья волны выберем следующую функцию:
\[
	\phi(x) = 4\exp(-100 x^4).
\]
Для простоты используем нулевые граничные условия. 
Скорость $c$ примем 1. Расстояние $L$ равняется 5. Соответственно, максимальное значение времи тоже равно 5.
Шаг по пространству возьмём 0.05, а по времени - 0.001. Такой выбор удовлетворяет условиям устойчивости всех трёх методов (\ref{eq:ust1} - \ref{eq:ust2}) и не требует больших вычислительных ресурсов.






\chapter{Поиск, верификация и предварительный анализ спутниковых данных}
Основным источником спутниковых данных служит Google Earth Engine (GEE) \cite{Cardille2024}. Этот сервис представляет собой распределённую базу данных, в которой хранятся сведения, полученные в результате работы множества космических программ. Все данные доступны для общего пользования.
\section{Поиск и отбор актуальных источников данных}

Для проведения пространственно-временного анализа климатических и ландшафтных показателей (NDVI, облачность, температура, высота местности) были использованы актуальные и проверенные источники спутниковых данных:

\begin{itemize}
	\item \textbf{Copernicus (ESA)} \cite{Copernicus2023} — источник снимков Sentinel-2 с высоким пространственным разрешением (10 м) и частотой съёмки (до 5 дней). Используемые коллекции:

		 \texttt{COPERNICUS/S2\_SR} — отражательная способность по спектральным каналам (в частности, B4 и B8 для расчёта NDVI);
		 \texttt{COPERNICUS/S5P/NRTI/L3\_CLOUD} — данные по облачности в режиме близком к реальному времени (переменная \texttt{cloud\_fraction}).

	\item \textbf{NASA EarthData} \cite{NASA_Earthdata2023} — глобальный источник цифровых моделей рельефа:  \texttt{NASA/NASADEM\_HGT/001} — цифровая модель высот с разрешением $\sim$30 м, на основе миссии SRTM.

	
	\item \textbf{ECMWF ERA5} \cite{ECMWF_Reanalysis_V5} — глобальные климатические данные, использованы для анализа температуры:
	 \texttt{ECMWF/ERA5\_LAND/HOURLY} — почасовые значения температуры на высоте 2 м (\texttt{temperature\_2m}), разрешение около 9 км.

\end{itemize}

\section{Верификация достоверности и актуальности данных}

Для обеспечения корректности анализа были выполнены следующие шаги верификации:

\begin{enumerate}
	\item \textbf{Пространственное покрытие}: все используемые коллекции фильтруются по координатам исследуемых участков с помощью метода \texttt{filterBounds()}.
	
	\item \textbf{Временной интервал}: фильтрация по дате осуществляется через \texttt{filterDate(start, end)}; используются интервалы по месяцам или сезонам.
	
	\item \textbf{Качество данных}:
	\begin{itemize}
		\item Sentinel-2 фильтруется по показателю \texttt{CLOUDY\_PIXEL\_PERCENTAGE < 20\%};
		\item исключаются значения \texttt{None} и выбросы по температуре и NDVI.
	\end{itemize}
	
	\item \textbf{Техническая пригодность}: все источники проверены на совместимость с API Google Earth Engine — коллекции успешно загружаются, ошибок доступа не зафиксировано.
\end{enumerate}

\section{Предварительный анализ данных}

\subsubsection*{NDVI (Normalized Difference Vegetation Index) \cite{Cherepanov2011}}

\begin{itemize}
	\item Источник: \texttt{COPERNICUS/S2\_SR}
	\item Формула:
	\[
	\text{NDVI} = \frac{B8 - B4}{B8 + B4},
	\]
	где $B8$ — ближний инфракрасный канал, $B4$ — красный канал. Расчёт в GEE осуществляется через функцию \texttt{normalizedDifference(['B8', 'B4'])}.
	\item Разрешение: 10 м
	\item Получение значений: среднее по точке через \texttt{reduceRegion(..., scale=10)}
\end{itemize}

\subsubsection*{Облачность}

\begin{itemize}
	\item Источник: \texttt{COPERNICUS/S5P/NRTI/L3\_CLOUD}
	\item Метрика: \texttt{cloud\_fraction}
	\item Разрешение: 1000 м
	\item Среднее значение рассчитывается на каждый временной интервал для каждой координаты
\end{itemize}

\subsubsection*{Температура воздуха}

\begin{itemize}
	\item Источник: \texttt{ECMWF/ERA5\_LAND/HOURLY}
	\item Переменная: \texttt{temperature\_2m}
	\item Разрешение: 1000 м
	\item Применяется усреднение по времени и пространству для каждой точки
\end{itemize}

\subsubsection*{Цифровая модель рельефа (DEM)}

\begin{itemize}
	\item Источник: \texttt{NASA/NASADEM\_HGT/001}
	\item Разрешение: $\sim$30 м
	\item Методика: извлечение значений высот через \texttt{sampleRegions()} по геометрии исследуемых точек
	\item Масштаб автоматически регулируется (от 30 до 500 м) для оптимизации количества точек
\end{itemize}

Выбранные спутниковые источники Copernicus, NASA и ECMWF являются авторитетными и проверенными платформами, обеспечивающими доступ к достоверным пространственно-временным данным. Верификация по охвату, времени, качеству и совместимости подтвердили пригодность данных для целей проекта. Предварительный анализ NDVI, облачности, температуры и высот показал, что структура и формат полученных данных соответствуют задачам курсового исследования, а применённые методы обеспечивают надёжную основу для последующего машинного обучения и статистической обработки.

	\backmatter %% Здесь заканчивается нумерованная часть документа и начинаются ссылки и
\chapter{Заключение}
Была сформулирована и проанализирована модель двумерного переноса.

Представлена программа для ЭВМ, которая вычисляет решения задачи.
В неё вошли метод лагранжевых частиц и конечно-разностный метод вверх по потоку.

Были проведены и проанализированы численные эксперименты с использованием разных схем решения системы с разными начальными условиями и полями.
\include{chapters/81-biblio}
	
	%% заключение
	
	
	%\appendix   % Тут идут приложения
	%\chapter{JSON тело запроса получения информации по полигону}
\label{cha:polygon}

\begin{figure}
	\centering
	\caption{Картинка в приложении. Страшная и ужасная.}
\end{figure}
	%\include{polygonJSONresponse}
\end{document}

%%% Local Variables:
%%% mode: latex
%%% TeX-master: t
%%% End:
