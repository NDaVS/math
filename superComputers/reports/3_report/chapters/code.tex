\chapter{JAVA код}
\section{Описание работы программы}


Данная программа демонстрирует использование \texttt{ExecutorService} и \texttt{ScheduledExecutorService} в языке Java для выполнения задачи с возможностью отмены по таймеру.  
Цель программы — сгенерировать массив случайных чисел, проверить наличие заданного числа и прервать выполнение задачи, если оно занимает слишком много времени.

\subsection{Основные параметры}
\begin{itemize}
	\item \texttt{size = 100} — размер массива случайных чисел;
	\item \texttt{target = 42} — целевое число, поиск которого выполняется;
	\item \texttt{timeout = 1\,s} — время ожидания до отмены задачи.
\end{itemize}

\subsection{Описание функционала}
\begin{enumerate}
	\item Определяется функциональный объект \texttt{containsNumber} типа \texttt{BiPredicate<Integer, Integer>}\cite{сьерра2012изучаем}, принимающий два аргумента:
	\begin{itemize}
		\item \texttt{n} — размер массива;
		\item \texttt{x} — искомое число.
	\end{itemize}
	Внутри создаётся массив случайных чисел, каждый элемент которого выбирается равномерно из диапазона $[0,100)$.  
	После генерации массив выводится на экран, и выполняется проверка:
	\[
	\text{result} = \exists i \in arr : i = x
	\]
	с использованием \texttt{Arrays.stream(arr).anyMatch(i -> i == x)}.
	
	\item Создаётся пул потоков:
	\begin{itemize}
		\item \texttt{ExecutorService executor} — выполняет основную задачу в отдельном потоке;
		\item \texttt{ScheduledExecutorService scheduler} — отвечает за планирование отмены по таймеру.
	\end{itemize}
	
	\item Определяется задача типа \texttt{Callable<Boolean>}, которая при вызове выполняет \texttt{containsNumber.test(size, target)} и возвращает булево значение.
	
	\item С помощью \texttt{executor.submit(task)} задача помещается в очередь на выполнение, а возвращаемое значение \texttt{Future<Boolean>} позволяет контролировать процесс.
	
	\item Планировщик \texttt{scheduler.schedule(...)} создаёт задачу, которая через одну секунду проверяет:
	\[
	\text{if (!future.isDone())} \Rightarrow \text{future.cancel(true)}
	\]
	Если основная задача не завершена — происходит её отмена.
	
	\item Основной поток вызывает \texttt{future.get()}, ожидая результат вычисления. Возможны три исхода:
	\begin{itemize}
		\item Если задача завершена успешно, выводится сообщение о наличии (или отсутствии) числа \texttt{target} в массиве;
		\item Если задача была отменена, возбуждается \texttt{CancellationException}, и выводится сообщение \texttt{"Задача была отменена"};
		\item При любых завершениях выполняется освобождение ресурсов: \texttt{executor.shutdown()} и \texttt{scheduler.shutdownNow()}.
	\end{itemize}
\end{enumerate}

\subsection{Код асинхронных методов и обрабочик}
\begin{verbatim}
	BiPredicate<Integer, Integer> containsNumber = (n, x) -> {
		Random rand = new Random();
		int[] arr = new int[n];
		
		for (int i = 0; i < n; i++) {
			arr[i] = rand.nextInt(100);
		}
		System.out.println("Сгенерированный массив: " 
		+ Arrays.toString(arr));
		return Arrays.stream(arr).anyMatch(i -> i == x);
	};
	
	ExecutorService executor = Executors.newSingleThreadExecutor();
	ScheduledExecutorService scheduler = 
		Executors.newSingleThreadScheduledExecutor();
	
	Callable<Boolean> task = () -> {
		//            Thread.sleep(3000);
		return containsNumber.test(size, target);
	};
	Future<Boolean> future = executor.submit(task);
	
	scheduler.schedule(() -> {
		if (!future.isDone()) {
			System.out.println("Отмена по таймеру");
			future.cancel(true);
		}
	}, 1, TimeUnit.SECONDS);
\end{verbatim}

\subsection{Результат работы}
На экране выводятся:
\begin{enumerate}
	\item Сгенерированный массив случайных чисел;
	\item Результат поиска:
	\[
	\text{``Число 42 найдено в массиве''} \quad \text{или} \quad \text{``не найдено''}
	\]
	\item При превышении времени ожидания (1 секунда) — сообщение:
	\[
	\text{``Отмена по таймеру'' и ``Задача была отменена''}
	\]
\end{enumerate}

\subsection{Ключевые концепции}
\begin{itemize}
	\item \textbf{Пул потоков (ExecutorService)} — средство управления асинхронными задачами;
	\item \textbf{Планировщик (ScheduledExecutorService)} — позволяет выполнять действия по таймеру;
	\item \textbf{Future} — объект, инкапсулирующий результат асинхронных вычислений и методы управления задачей (\texttt{cancel()}, \texttt{isDone()}, \texttt{get()}).
\end{itemize}



