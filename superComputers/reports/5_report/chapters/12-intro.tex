\chapter{Введение}

В современных вычислительных системах всё большее значение приобретают параллельные и асинхронные методы обработки данных \cite{zubkova2014}. 
Использование многопоточности позволяет выполнять независимые задачи одновременно, повышая общую эффективность программы и снижая время отклика.  

Java предоставляет мощные средства для организации параллельных вычислений, включая такие высокоуровневые компоненты, как \texttt{ExecutorService}, \texttt{Callable} и \texttt{Future} \cite{сьерра2012изучаем}.  
Данные инструменты позволяют запускать задачи в отдельных потоках, получать результаты их выполнения и контролировать процесс работы без необходимости вручную управлять потоками.  

В данной работе демонстрируется пример вычислительной задачи, выполняемой асинхронно: генерация массива случайных чисел и проверка наличия заданного элемента.  
Подобный подход может применяться для выполнения тяжёлых вычислений, анализа данных, поиска решений в больших пространствах или обработки потоков информации, не блокируя основной поток программы.
