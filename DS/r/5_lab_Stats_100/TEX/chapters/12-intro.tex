\chapter{Введение}
В наше время обработка данных является одним из самых востребованных областей IT области.
Методы позволяют находить  зависимости между различными показателями при обработке данных.

Эта работка направлена на применение полученых при обучениии навыков анализа.
В качестве датасета выступает Stat100\_Fall2018\_Survey02 за авторством Университета Иллинойса.

Целью работы является формирование двух моделей линейной регрессии и их сравнение по ряду параметров.

На основании цели были сформулированы следующие задачи:
\begin{enumerate}
	\item Загрузка и очистка датасета .
	\item Сформировать две модели линейной регрессии (базовую и расширенную).
	\item Построить визуализацю для каждой модели .
	\item Сделать выводы на основе полученных метрик и характеристик
\end{enumerate}