\chapter{Введение}
Анализ и обработка данных в современном мире занимают центральное место в сфере информационных технологий. Современные методы анализа позволяют выявлять закономерности и зависимости между различными характеристиками в данных, что делает их крайне востребованными в научных и прикладных задачах.

Цель данной работы — продемонстрировать применение методов линейной регрессии на реальных данных. В качестве объекта исследования выбран открытый датасет Stat100\_Fall2018\_Survey02, предоставленный Университетом Иллинойса. Этот набор данных содержит информацию, собранную в ходе учебного опроса, и хорошо подходит для целей демонстрационного анализа.

В рамках работы будут реализованы две модели линейной регрессии: базовая и расширенная. Сравнение этих моделей позволит выявить, насколько добавление дополнительных признаков влияет на качество предсказания.

Для достижения поставленной цели решаются следующие задачи:

\begin{enumerate}
	\item Загрузка и предварительная обработка датасета.
	\item Построение двух моделей линейной регрессии — базовой и расширенной.
	\item Визуализация результатов моделирования для наглядной интерпретации.
	\item Сравнительный анализ моделей по ключевым метрикам качества и интерпретация полученных результатов.
\end{enumerate}

В качестве инструмента анализа используется язык R\cite{metloff2019}. Для визуализации данных - библиотека ggplot2\cite{mastitsky2017}. Разделение данных произведено с помощью библиотеки caret\cite{caret2019}.