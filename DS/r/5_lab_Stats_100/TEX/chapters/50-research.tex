\chapter{Подготовка данных}
Для начала работу требуется загрузить найденный датасет. 
Но заголовки, заданные по умолчанию, содержать поясняющую информацию. 
Поэтому требуется отдельно загрузить данные и их заголовки.
Заранее все заголовки были перенесены в файд headers.txt.
Загрузка основной информации происходить с помощью кода, представленного в листинге \ref{lst:data}.

\lstinputlisting[language=R,caption=Считывание датасета ,label=lst:data]{listings/dataset.R}

Данные о размерности: (1458 ,35)
Первые пять записей представлены в таблице \ref{tab:truncated_table}
\begin{table}[ht]
	\centering
	\caption{Сводная таблица наблюдений: первые и последние 5 переменных (остальные обозначены как $\cdots$)}
	\scriptsize
	\setlength{\tabcolsep}{7pt}
	\begin{tabular}{|c|r r r r r c r r r r r|}
		\hline
		Obs & V1 & V2 & V3 & V4 & V5 & $\cdots$ & V31 & V32 & V33 & V34 & V35 \\
		\hline
		Obs1 & 1 & 1 & 1 & 0 & 2 & $\cdots$ & 0 & 100 & 7 & 1 & NA \\
		Obs2 & 2 & 1 & 1 & 0 & 3 & $\cdots$ & 0 & 100 & 9 & 1 & NA \\
		Obs3 & 3 & 1 & 1 & 1 & 2 & $\cdots$ & 0 & 100 & 3 & 1 & NA \\
		Obs4 & 4 & 1 & 1 & 0 & 2 & $\cdots$ & 15 & 0 & 1 & 1 & NA \\
		Obs5 & 5 & 0 & 0 & 1 & 2 & $\cdots$ & 0 & 50 & 5 & 1 & NA \\
		\hline
	\end{tabular}
	\label{tab:truncated_table}
\end{table}



Для обработки заголовки был использован следующий подход (листинг \ref{lst:headers} ):
\begin{itemize}
	\item считывание строки с заголовками;
	\item Использование регулярного выражения для удаления скобок и значений в них;
	\item Удаление лишних пробелов;
	\item Разделение заначений на элементы массива
\end{itemize}

\lstinputlisting[language=R,caption=Работа с заголовками, label=lst:headers]{listings/headers.R}
Длина полученного массива: 34 (на один меньше, чем количество столбцов датафрейма, т.к. последние значения там NULL)

После проделаных шагов, значения заголовков добавляются к датасету  (листинг \ref{lst:applying}).

\lstinputlisting[language=R,caption=Применение заголовков, label=lst:applying]{listings/appling_headers.R}
Размерность не изменилась: (1458, 35).

Первые пять строк полученного датасета представлены в таб. \ref{tab:students_summary}.

\begin{table}[ht]
	\centering
	\caption{Социально-академические характеристики студентов: только первые и последние 4 переменных}
	\scriptsize
	\setlength{\tabcolsep}{6pt}
	\begin{tabular}{|c|r r r r r c r r r r r|}
		\hline
		№   & Gender & Gender\_ID & Greek & Home\_Town & $\cdots$ & Work\_Hours & Tuition & Career & Section \\
		\hline
		1 & 1      & 1          & 1     & 0          & $\cdots$   & 0          & 100     & 7      & 1       \\
		2 & 2      & 1          & 1     & 0          & $\cdots$ & 0          & 100     & 9      & 1       \\
		3 & 3      & 1          & 1     & 1          & $\cdots$& 0          & 100     & 3      & 1       \\
		4 & 4      & 1          & 1     & 0          & $\cdots$ &15         & 0       & 1      & 1       \\
		5 & 5      & 0          & 0     & 1          & $\cdots$     & 0          & 50      & 5      & 1       \\
		6 & 6      & 1          & 1     & 0          & $\cdots$       & 0          & 90      & 5      & 1       \\
		\hline
	\end{tabular}
	\label{tab:students_summary}
\end{table}

Далее подготовим только нужные значения из датайрейма(\ref{lst:dd}).
\lstinputlisting[language=R,caption=Сформируем датафрейм нужных признаков, label=lst:dd]{listings/dd.R}
Описание данных:
\begin{enumerate}
	\item \textbf{Пол (Gender)}: 
	\begin{itemize}
		\item 0 = Мужской
		\item 1 = Женский
	\end{itemize}
	
	\item \textbf{Родной город (Home\_Town)}:  тип населенного пункта:
	\begin{itemize}
		\item 0 = Маленький город
		\item 1 = Средний город
		\item 2 = Большой город (пригород)
		\item 3 = Большой город (без пригородов)
	\end{itemize}
	
	\item \textbf{Часы вечеринок в неделю (Party\_Hours\_per\_week)}: среднее количество часов, проведенных на вечеринках в неделю.
	
	\item \textbf{Алкогольные напитки в неделю (Drinks\_per\_week)}: среднее количество алкогольных напитков, потребляемых за неделю.
\end{enumerate}


Преобразуем параметр города следующим образом: 0=Small Town+Medium City, 1=Big City. Также пол и города преобразуем в факторную переменную (листинг \ref{lst:towns}). 
\lstinputlisting[language=R,caption=Преобразование нужных признаков, label=lst:towns]{listings/towns.R}




Первые  пять строк полученного датасета представлены в таблице \ref{tab:behavior_town}
\begin{table}[ht]
	\centering
	\caption{Параметры студентов, связанные с поведением и происхождением}
	\begin{tabular}{|c|r|r|r|r|}
		\hline
		№   & Drinks\_per\_week & Party\_Hours\_per\_week & Gender & Home\_Town \\
		\hline
		1 & 1                 & 1                       & 1      & 1          \\
		2 & 0                 & 0                       & 1      & 1          \\
		3 & 10                & 6                       & 1      & 1         \\
		4 & 35                & 20                      & 1      & 1        \\
		5 & 25                & 12                      & 0      & 1        \\
		6 & 3                 & 8                       & 1      & 1          \\
		\hline
	\end{tabular}
	\label{tab:behavior_town}
\end{table}

Размерность полученного датафрейма: (1458, 4).

Сводная информаци по полученному датафрейму представлена в таб. \ref{tab:desc_stats}.

\begin{table}[ht]
	\centering
	\caption{Описательная статистика по ключевым переменным}
	\begin{tabular}{|l|r|r|r|r|}
		\hline
		Показатель     & Drinks\_per\_week & Party\_Hs\_per\_wk & Gender & Home\_Tn \\
		\hline
		Min            & 0.000             & 0.000                   & 0.0000 & 0.000      \\
		1st Quartile   & 0.000             & 1.000                   & 0.0000 & 1.000      \\
		Median         & 3.000             & 4.000                   & 1.0000 & 2.000      \\
		Mean           & 6.475             & 5.496                   & 0.6454 & 1.853      \\
		3rd Quartile   &10.000             & 8.000                   & 1.0000 & 3.000      \\
		Max            &50.000             &50.000                   & 1.0000 & 3.000      \\
		\hline
	\end{tabular}
	\label{tab:desc_stats}
\end{table}

Разделим полученый датафрейм на обучающую, валидационную  и тестовую выборки (листинг \ref{lst:samples})
\lstinputlisting[language=R,caption=Разделение на выборки, label=lst:samples]{listings/samples.R}

На этом подготовка данных завершена.