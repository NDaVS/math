\section{Введение}
В настоящее время методы классификации в области обработки больших данных находят применения в множестве сфер.
Одним из примеров можно привести применение классификаци в банковской системе: там она помогает в определение кредитоспособности заёмщиков. Кроме того, систему подобного рода можно настроить на решение других задач классификации, при условии, что обучающая выборка и данные будут достаточно репрезентативны и представлять из себя статистически значимые признаки.

Это, в свою очередь, предоставляет большие возможности для развития подобных систем в различных сферах повседневной жизни. Поэтому тема представляет большой интерес с точки зрения исследования и погружения в тему (возможно стоит поправить формулировку)

\subsection*{Цель работы}
Исспледовать методы классификации на основе предоставленного датасета \textbf{flsr\_moscow.txt}, разработать программу для классификации квартир на четыре класса по параметру площади (\textbf{totsp}) и сравнить полученные результаты для распределения квартир по количеству комнат.

\subsection*{Задачи}
\begin{itemize}
    \item Изучить теоретические основы методов классификации
    \item Проанализировать полученный датасет
    \item Построить матрицу путаницы
    \item провести ROC-анализ
    \item Сравнить методы логистической регрессии и SVM
    \item Визуализировать полученные результаты
\end{itemize}