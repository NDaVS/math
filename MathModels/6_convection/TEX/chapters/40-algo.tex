\chapter{Вычислительные эксперименты}
Рассмотрим задачу численного решения уравнения переноса скалярной величины \( u(x, y, t) \) в двумерной области \(\Omega \subset \mathbb{R}^2\). При отсутствии источников и стоков правая часть уравнения равна нулю (\(f = 0\)), и задача записывается в виде:

\begin{equation}
	\frac{\partial u}{\partial t} + \vec{v}(x, y, t) \cdot \nabla u = 0,
	\label{eq:advection}
\end{equation}
где \( \vec{v}(x, y, t) = (v_x, v_y) \) — заданное векторное поле скоростей, \( \nabla u = \left( \frac{\partial u}{\partial x}, \frac{\partial u}{\partial y} \right) \) — градиент функции \( u \).

Граничные условия принимаются периодическими. Начальное распределение \( u(x, y, 0) \) задается одной из аналитических функций: гауссовой функцией, индикатором круга и т.д.

Для экспериментов зададим следующие условия:
\begin{enumerate}
	\item Векторное поле $\Omega$ задано Декартовым произведением $[0,10] \times [0,10];$
	\item Временной интервал $t\in [0, 10]$;
	\item Шаг по времени $\Delta t = 0.01$;
	\item Все поля стационарные (не зависят от веремни);
	\item В случае, если точка в методе лагранжевых частиц выходит за пределы поля, более мы её не рассматриваем.
\end{enumerate}

\section{Аналитическое решение}

В случае отсутствия источников (\(f = 0\)) и известного поля скоростей \(\vec{v}(x,y,t)\), уравнение \eqref{eq:advection} допускает аналитическое решение вдоль траекторий частиц, заданных уравнением:

\begin{equation}
	\frac{d\vec{x}}{dt} = \vec{v}(\vec{x}, t), \quad \vec{x}(0) = \vec{x}_0,
	\label{eq:characteristics}
\end{equation}
где \(\vec{x}_0\) — начальная координата частицы. Тогда значение скалярной величины \(u\) вдоль характеристики сохраняется:

\begin{equation}
	u(\vec{x}(t), t) = u_0(\vec{x}_0).
	\label{eq:solution_along_characteristics}
\end{equation}

Это решение реализуется в методе лагранжевых частиц.
\section{Программа для ЭВМ}
В качестве языка программирования для рассчётов и
визуализации был выбран Python с использованием библиотек numpy
(вычисления) и matplotlib (визуализация).
\section{Вверх по потоку (upwind)}
\subsection{Численная схема конечных разностей}

Пространственная область \(\Omega = [x_{\min}, x_{\max}] \times [y_{\min}, y_{\max}]\) дискретизуется равномерной сеткой с \(n \times n\) узлами. Шаги по пространству:

\[
dx = \frac{x_{\max} - x_{\min}}{n - 1}, \quad dy = \frac{y_{\max} - y_{\min}}{n - 1}.
\]

Для аппроксимации производных используются односторонние разности в зависимости от направления скорости (схема upwind):

\[
\frac{\partial u}{\partial x} \approx
\begin{cases}
	\frac{u_{i,j} - u_{i-1,j}}{dx}, & v_x \geq 0 \\
	\frac{u_{i+1,j} - u_{i,j}}{dx}, & v_x < 0
\end{cases}, \quad
\frac{\partial u}{\partial y} \approx
\begin{cases}
	\frac{u_{i,j} - u_{i,j-1}}{dy}, & v_y \geq 0 \\
	\frac{u_{i,j+1} - u_{i,j}}{dy}, & v_y < 0
\end{cases}
\]

Временное интегрирование производится явным способом (метод Эйлера):

\[
u_{i,j}^{n+1} = u_{i,j}^{n} - \Delta t \left( v_x \frac{\partial u}{\partial x} + v_y \frac{\partial u}{\partial y} \right).
\]

Стабильность обеспечивается условием Куранта–Фридрихса–Леви (CFL) \cite{1964calculus}:

\begin{equation}
	\max \left( \frac{|v_x| \Delta t}{dx}, \frac{|v_y| \Delta t}{dy} \right) \leq 1.
	\label{eq:cfl_condition}
\end{equation}

\subsection{Описание экспериментов}

В ходе экспериментов используются различные поля скоростей:
\begin{itemize}
	\item \textbf{Спиральное течение (spiral)}:
	\begin{equation}
        v_x = (y-5) / 5,
		v_y = (-x + y) /5.
			\label{eq:vortex}
	\end{equation}

	
	\item \textbf{Модифицированное спиральное течение (sin-spiral)}:
	\begin{equation}
		v_x = (y - 5) / 5, \quad
		v_y = (-x + y) \sin(x )/ 5.
			\label{eq:shear}
	\end{equation}

	
	\item \textbf{Дивергентное поле (divergance)}:
	\begin{equation}
			v_x = -\pi \sin\left( \frac{2\pi x}{10} \right) \cos\left( \frac{\pi y}{10} \right), \quad
		v_y = 2\pi \sin\left( \frac{\pi y}{10} \right) \cos\left( \frac{2\pi x}{10} \right).
		\label{eq:div}
	\end{equation}

\end{itemize}

Начальные условия представлены в виде кругов или полос с шириной 1.

\subsection{Реализация алгоритма на Python}

Программа реализована на языке Python с использованием библиотек \texttt{numpy} и \texttt{matplotlib}. Структура кода включает:
\begin{itemize}
	\item \texttt{FluidSimulation} — класс с основной логикой метода конечных разностей;
	\item \texttt{VelocityField} — статические методы для задания различных полей скоростей;
	\item \texttt{InitialCondition} — начальные условия;
	\item \texttt{Experiment}, \texttt{ExperimentManager} — система организации и запуска экспериментов;
	\item визуализацию полей скоростей и решений.
\end{itemize}


Все течения зададим стационарными (не зависящими от $t$)
\subsection{Визуализация результатов}

Результаты каждого эксперимента сохраняются в виде изображений, иллюстрирующих:
\begin{itemize}
	\item исходное распределение скалярной величины \(u(x,y,0)\);
	\item поле скоростей (стримплот);
	\item распределение \(u(x,y,t)\) на каждом шаге.
\end{itemize}
\pagebreak
\subsection{Вихревое течение}
\begin{figure}
	\centering
	\begin{minipage}[b]{0.48\textwidth}
		\centering
	\includegraphics[width=\textwidth]{./../lab6_images_fd/exp_1:_spiral_flow_fd_velocity_field.png}
	\caption{Поле скоростей для спирального течения.}
	\label{fig:vortex_velocity}
	\end{minipage}
	\hfill
	\begin{minipage}[b]{0.48\textwidth}
		\centering
	\includegraphics[width=\textwidth]{./../lab6_images_fd/exp_1:_spiral_flow_t0.00.png}
	\caption{Скалярное поле \(u\) в начальный момент времени.}
	\label{fig:vortex_begin}
	\end{minipage}
\end{figure}

Поле на рис. \ref{fig:vortex_velocity} задаётся уравнением вихря (\ref{eq:vortex}).

В качестве начального распределения, используется круг с центром в точке (5 , 5) и радиусом 3 (\ref{fig:vortex_begin}).

Изображения промежуточных ходов представлены на рис. \ref{fig:exp_1_t2} - \ref{fig:exp_1_t8}

\begin{figure}
	\centering
	\begin{minipage}[b]{0.48\textwidth}
		\centering
		\includegraphics[width=\textwidth]{./../lab6_images_fd/exp_1:_spiral_flow_t2.00.png}
		\caption{Скалярное поле в момент времени $t=2$.}
		\label{fig:exp_1_t2}
	\end{minipage}
	\hfill
	\begin{minipage}[b]{0.48\textwidth}
		\centering
		\includegraphics[width=\textwidth]{./../lab6_images_fd/exp_1:_spiral_flow_t4.00.png}
		\caption{Скалярное поле в момент времени $t=4$.}
	\end{minipage}
	\vspace{0.5cm}
	
	\begin{minipage}[b]{0.48\textwidth}
		\centering
		\includegraphics[width=\textwidth]{./../lab6_images_fd/exp_1:_spiral_flow_t6.00.png}
		\caption{Скалярное поле в момент времени $t=6$.}
	\end{minipage}
	\hfill
	\begin{minipage}[b]{0.48\textwidth}
		\centering
		\includegraphics[width=\textwidth]{./../lab6_images_fd/exp_1:_spiral_flow_t8.00.png}
		\caption{Скалярное поле в момент времени $t=8$.}
		\label{fig:exp_1_t8}
	\end{minipage}
\end{figure}
\newpage
Финальное распределения  представлено на рис. \ref{fig:vortex_final}.

\begin{figure}[H]
	\centering
	\includegraphics[width=0.7\textwidth]{./../lab6_images_fd/exp_1:_spiral_flow_t10.00.png}
	\caption{Скалярное поле \(u\) в финальный момент времени ($t=10$).}
	\label{fig:vortex_final}
\end{figure}

Как можно видеть, начальное распределение расширилось по направлению поля. Края размылись из-за особенностей вычислительной схемы.
\newpage
\subsection{Течение по модифицированной спирали}
Поле на рис. \ref{fig:shear_velocity} задаётся уравнением поля (\ref{eq:shear})
\begin{figure}
	\centering
	\begin{minipage}[b]{0.48\textwidth}
		\centering
	\includegraphics[width=\textwidth]{./../lab6_images_fd/exp_2:_sin-spiral_flow_fd_velocity_field.png}
	\caption{Поле скоростей для течения модиф-ой спирали.}
	\label{fig:shear_velocity}
	\end{minipage}
	\hfill
	\begin{minipage}[b]{0.48\textwidth}
		\centering
	\includegraphics[width=\textwidth]{./../lab6_images_fd/exp_2:_sin-spiral_flow_t0.00.png}
	\caption{Скалярное поле \(u\) в начальный момент времени.}
	\label{fig:shear_begin}
	\end{minipage}
\end{figure}

В качетве начального распределения, используется диагональная  полоса  с левого верхнего угла к правому нижнему  и шириной 1 (рис. \ref{fig:shear_begin}).


Изображения промежуточных ходов представлены на рис. \ref{fig:exp_2_t=2} - \ref{fig:exp_2_t=8}.

\begin{figure}
	\centering
	\begin{minipage}[b]{0.48\textwidth}
		\centering
		\includegraphics[width=\textwidth]{./../lab6_images_fd/exp_2:_sin-spiral_flow_t2.00.png}
		\caption{Скалярное поле в момент времени $t=2$.}
		\label{fig:exp_2_t=2}
	\end{minipage}
	\hfill
	\begin{minipage}[b]{0.48\textwidth}
		\centering
		\includegraphics[width=\textwidth]{./../lab6_images_fd/exp_2:_sin-spiral_flow_t4.00.png}
		\caption{Скалярное поле в момент времени $t=4$.}
	\end{minipage}
	\vspace{0.5cm}
	
	\begin{minipage}[b]{0.48\textwidth}
		\centering
		\includegraphics[width=\textwidth]{./../lab6_images_fd/exp_2:_sin-spiral_flow_t6.00.png}
		\caption{Скалярное поле в момент времени $t=6$.}
	\end{minipage}
	\hfill
	\begin{minipage}[b]{0.48\textwidth}
		\centering
		\includegraphics[width=\textwidth]{./../lab6_images_fd/exp_2:_sin-spiral_flow_t8.00.png}
		\caption{Скалярное поле в момент времени $t=8$.}
				\label{fig:exp_2_t=8}
	\end{minipage}
\end{figure}

\newpage
Финальное распределения представлено на рис. \ref{fig:shear_final}.

\begin{figure}
	\centering
	\includegraphics[width=0.7\textwidth]{./../lab6_images_fd/exp_2:_sin-spiral_flow_t10.00.png}
	\caption{Скалярное поле \(u\) в финальный момент времени ($t=10$).}
	\label{fig:shear_final}
\end{figure}
В этом эксперименте также наблюдается перенос по направленю поля. Границы всё также размыты.
Справа снизу появилась аномальная область. Это можно объяснить особенностью вычислительной схемы.

\newpage
\subsection{Дивергентное течение}
Поле на рис. \ref{fig:div_velocity} задаётся уравнением поля (\ref{eq:div}).

\begin{figure}
	\centering
	\begin{minipage}[b]{0.48\textwidth}
		\centering
	\includegraphics[width=\textwidth]{./../lab6_images_fd/epx_3:_divergent_flow_fd_velocity_field.png}
	\caption{Поле скоростей для дивергентного течения.}
	\label{fig:div_velocity}
	\end{minipage}
	\hfill
	\begin{minipage}[b]{0.48\textwidth}
		\centering
	\includegraphics[width=\textwidth]{./../lab6_images_fd/epx_3:_divergent_flow_t0.00.png}
	\caption{Скалярное поле \(u\) в начальный момент времени.}
	\label{fig:div_begin}
	\end{minipage}
\end{figure}

В качетве начального распределения, используется горизонтальная полоса с координатой $y$ центральной линии равной 5 и шириной 1 (рис. \ref{fig:div_begin}).

Изображения промежуточных ходов представлены на рис. \ref{fig:exp_3_t2} - \ref{fig:exp_3_t8}.

\begin{figure}
	\centering
	\begin{minipage}[b]{0.48\textwidth}
		\centering
		\includegraphics[width=\textwidth]{./../lab6_images_fd/exp_3:_divergent_flow_t2.00.png}
		\caption{Скалярное поле в момент времени $t=2$.}
		\label{fig:exp_3_t2}
	\end{minipage}
	\hfill
	\begin{minipage}[b]{0.48\textwidth}
		\centering
		\includegraphics[width=\textwidth]{./../lab6_images_fd/exp_3:_divergent_flow_t4.00.png}
		\caption{Скалярное поле в момент времени $t=4$.}
	\end{minipage}
	\vspace{0.5cm}
	
	\begin{minipage}[b]{0.48\textwidth}
		\centering
		\includegraphics[width=\textwidth]{./../lab6_images_fd/exp_3:_divergent_flow_t6.00.png}
		\caption{Скалярное поле в момент времени $t=6$.}
	\end{minipage}
	\hfill
	\begin{minipage}[b]{0.48\textwidth}
		\centering
		\includegraphics[width=\textwidth]{./../lab6_images_fd/exp_3:_divergent_flow_t8.00.png}
		\caption{Скалярное поле в момент времени $t=8$.}
		\label{fig:exp_3_t8}
	\end{minipage}
\end{figure}

\newpage
Финальное распределение представлено на рис. \ref{fig:div_final}.
\begin{figure}
	\centering
	\includegraphics[width=0.7\textwidth]{./../lab6_images_fd/epx_3:_divergent_flow_t10.00.png}
	\caption{Скалярное поле \(u\) в финальный момент времени ($t=10$).}
	\label{fig:div_final}
\end{figure}

В этом эксперименте виден недостаток вычислительной схемы - при высокой турбулентности начальное распределение расплывается.


\newpage

\section{Метод Лагранжевых частиц}
\subsection{Описание численного метода}

В отличие от метода конечных разностей, лагранжев подход отслеживает движение частиц, несущих скалярную величину \( u \), по траекториям, определяемым полем скоростей. Основная идея заключается в решении задачи Коши для системы ОДУ:

\[
\frac{d \mathbf{x}}{dt} = \mathbf{v}(\mathbf{x}, t), \quad \mathbf{x}(0) = \mathbf{x}_0,
\]
где \( \mathbf{x}(t) = (x(t), y(t)) \) — положение частицы во времени, а \( \mathbf{v} = (v_x, v_y) \) — заданное поле скоростей.

Частицы не взаимодействуют и перемещаются независимо друг от друга. Величина \( u \) сохраняется вдоль траектории:

\[
\frac{du}{dt} = 0 \Rightarrow u(\mathbf{x}(t), t) = u(\mathbf{x}_0, 0).
\]

Для численного интегрирования используется метод Рунге–Кутты 4-го порядка (RK4):

\begin{equation}
	\begin{aligned}
		\mathbf{k}_1 &= \mathbf{v}(\mathbf{x}_n), \\
		\mathbf{k}_2 &= \mathbf{v}(\mathbf{x}_n + \tfrac{1}{2} \Delta t \mathbf{k}_1), \\
		\mathbf{k}_3 &= \mathbf{v}(\mathbf{x}_n + \tfrac{1}{2} \Delta t \mathbf{k}_2), \\
		\mathbf{k}_4 &= \mathbf{v}(\mathbf{x}_n + \Delta t \mathbf{k}_3), \\
		\mathbf{x}_{n+1} &= \mathbf{x}_n + \frac{\Delta t}{6} (\mathbf{k}_1 + 2\mathbf{k}_2 + 2\mathbf{k}_3 + \mathbf{k}_4).
	\end{aligned}
	\label{eq:rk4}
\end{equation}

Для отрисовки скалярных полей используется кубическая интерполяция.

\subsection{Реализация на Python}

Программа реализована на языке Python с использованием библиотек \texttt{numpy}, \texttt{matplotlib}, \texttt{scipy}. Структура кода включает:
\begin{itemize}
	\item \texttt{ParticleSet} — класс, отвечающий за координаты и значения частиц;
	\item \texttt{VelocityField} — статические методы для различных полей скоростей;
	\item \texttt{Integrator} — численные методы интегрирования траекторий (RK4);
	\item визуализацию движения частиц и их распределений;
	\item построение плотности \( u(x, y, t) \) через биннинг в сетку.
\end{itemize}

\subsection{Описание экспериментов}
В экспериментах используются те же три стационарных поля скоростей (\ref{eq:vortex}), (\ref{eq:shear}), (\ref{eq:div}).
Количество точек для моделирования - 1000.
Начальные положения частиц и параметры моделирования задаются аналогично. 
Все точки инициализируются с $u = 1$.

\subsection{Визуализация результатов}

Для каждого эксперимента строятся:
\begin{itemize}
	\item Карта начального положения и значений частиц;
	\item График поля скоростей;
	\item Промежуточные распределение частиц с учётом перемещения;
	\item Реконструированная плотность скалярной величины на сетке.
\end{itemize}
\newpage
\subsection{Спиральное течение (Lagrangian)}
Поле используем то же, что и в эксперименте с разделёнными разностями (рис. \ref{fig:vortex_velocity},  задаётся уравнением спирального течения (\ref{eq:vortex}).

В качестве начального распределения, используется круг с центром в точке (5 , 5) и радиусом 3 (\ref{fig:lg_vortex_begin}).
\begin{figure}
	\centering
	\includegraphics[width=0.7\textwidth]{./../lab6_images/exp_1:_spiral_flow_t0.00.png}
	\caption{Начальное распределение частиц (спиральное поле).}
	\label{fig:lg_vortex_begin}
\end{figure}

Изображения промежуточных ходов представлены на рис. \ref{fig:exp_1l_t2} - \ref{fig:exp_1l_t8}. Также добавлены траектории 50 случайных точек.

\begin{figure}
	\centering
	\begin{minipage}[b]{0.48\textwidth}
		\centering
		\includegraphics[width=\textwidth]{./../lab6_images/exp_1:_spiral_flow_t2.00.png}
		\caption{Скалярное поле в момент времени $t=2$.}
		\label{fig:exp_1l_t2}
	\end{minipage}
	\hfill
	\begin{minipage}[b]{0.48\textwidth}
		\centering
		\includegraphics[width=\textwidth]{./../lab6_images/exp_1:_spiral_flow_t4.00.png}
		\caption{Скалярное поле в момент времени $t=4$.}
	\end{minipage}
	\vspace{0.5cm}
	
	\begin{minipage}[b]{0.48\textwidth}
		\centering
		\includegraphics[width=\textwidth]{./../lab6_images/exp_1:_spiral_flow_t6.00.png}
		\caption{Скалярное поле в момент времени $t=6$.}
	\end{minipage}
	\hfill
	\begin{minipage}[b]{0.48\textwidth}
		\centering
		\includegraphics[width=\textwidth]{./../lab6_images/exp_1:_spiral_flow_t8.00.png}
		\caption{Скалярное поле в момент времени $t=8$.}
		\label{fig:exp_1l_t8}
	\end{minipage}
\end{figure}

\newpage
Финальное распределение представлено на рис. \ref{fig:lg_vortex_finall}.
\begin{figure}
	\centering
	\includegraphics[width=0.65\textwidth]{./../lab6_images/exp_1:_spiral_flow_t10.00.png}
	\caption{Финальное распределение частиц.}
	\label{fig:lg_vortex_finall}
\end{figure}

Траектории всех точек представлены на рис. \ref{fig:lg_vortex_tr}. 

\begin{figure}[H]
	\centering
	\includegraphics[width=0.65\textwidth]{./../lab6_images/exp_1:_spiral_flow_trajectories.png}
	\caption{Траектория частиц.}
	\label{fig:lg_vortex_tr}
\end{figure}

\newpage
\subsection{Течение модифицироанной спирали (Lagrangian)}

Поле используем то же, что и в эксперименте с разделёнными разностями (рис. \ref{fig:shear_velocity},  задаётся уравнением (\ref{eq:shear}).

В качестве начального распределения, используется диагональная  полоса из левого верхнего угла в правый нижний  с шириной 1 (рис. \ref{fig:lg_shaer_begin}). 
\begin{figure}
	\centering
	\includegraphics[width=0.7\textwidth]{./../lab6_images/exp_2:_sin-spiral_flow_t0.00.png}
	\caption{Начальное распределение частиц.}
	\label{fig:lg_shaer_begin}
\end{figure}

Изображения промежуточных ходов представлены на рис. \ref{fig:exp2_t2} - \ref{fig:exp2_t8}.  Также добавлены траектории 50 случайных точек.

\begin{figure}
	\centering
	\begin{minipage}[b]{0.48\textwidth}
		\centering
		\includegraphics[width=\textwidth]{./../lab6_images/exp_2:_sin-spiral_flow_t2.00.png}
		\caption{Скалярное поле в момент времени $t=2$.}
		\label{fig:exp2_t2}
	\end{minipage}
	\hfill
	\begin{minipage}[b]{0.48\textwidth}
		\centering
		\includegraphics[width=\textwidth]{./../lab6_images/exp_2:_sin-spiral_flow_t4.00.png}
		\caption{Скалярное поле в момент времени $t=4$.}
	\end{minipage}
	\vspace{0.5cm}
	
	\begin{minipage}[b]{0.48\textwidth}
		\centering
		\includegraphics[width=\textwidth]{./../lab6_images/exp_2:_sin-spiral_flow_t6.00.png}
		\caption{Скалярное поле в момент времени $t=6$.}
	\end{minipage}
	\hfill
	\begin{minipage}[b]{0.48\textwidth}
		\centering
		\includegraphics[width=\textwidth]{./../lab6_images/exp_2:_sin-spiral_flow_t8.00.png}
		\caption{Скалярное поле в момент времени $t=8$.}
				\label{fig:exp2_t8}
	\end{minipage}
\end{figure}

\newpage
Финальное распределение представлено на рис. \ref{fig:lg_shaer_finall}.
\begin{figure}
	\centering
	\includegraphics[width=0.65\textwidth]{./../lab6_images/exp_2:_sin-spiral_flow_t10.00.png}
	\caption{Финальное распределение частиц.}
	\label{fig:lg_shaer_finall}
\end{figure}

Траектории всех точек представлены на рис. \ref{fig:lg_shaer_tr}.

\begin{figure}[H]
	\centering
	\includegraphics[width=0.65\textwidth]{./../lab6_images/exp_2:_sin-spiral_flow_trajectories.png}
	\caption{Траектории частиц.}
	\label{fig:lg_shaer_tr}
\end{figure}

\newpage
\subsection{Дивергентное течение (Lagrangian)}
Поле используем то же, что и в эксперименте с разделёнными разностями (рис. \ref{fig:div_velocity},  задаётся уравнением (\ref{eq:div}).

В качестве начального распределения, используется горизонтальная полоса с центральной линией на координате $y=5$ и шириной 1 (рис. \ref{fig:lg_div_begin}). 
\begin{figure}
	\centering
	\includegraphics[width=0.7\textwidth]{./../lab6_images/exp_3:_divergent_flow_t0.00.png}
	\caption{Начальное распределение частиц.}
	\label{fig:lg_div_begin}
\end{figure}

Изображения промежуточных ходов представлены на рис. \ref{fig:exp3_2} - \ref{fig:exp3_8}. Также добавлены траектории 50 случайных точек.

\begin{figure}
	\centering
	\begin{minipage}[b]{0.48\textwidth}
		\centering
		\includegraphics[width=\textwidth]{./../lab6_images/exp_3:_divergent_flow_t2.00.png}
		\caption{Скалярное поле в момент времени $t=2$.}
		\label{fig:exp3_2}
	\end{minipage}
	\hfill
	\begin{minipage}[b]{0.48\textwidth}
		\centering
		\includegraphics[width=\textwidth]{./../lab6_images/exp_3:_divergent_flow_t4.00.png}
		\caption{Скалярное поле в момент времени $t=4$.}
	\end{minipage}
	\vspace{0.5cm}
	
	\begin{minipage}[b]{0.48\textwidth}
		\centering
		\includegraphics[width=\textwidth]{./../lab6_images/exp_3:_divergent_flow_t6.00.png}
		\caption{Скалярное поле в момент времени $t=6$.}
	\end{minipage}
	\hfill
	\begin{minipage}[b]{0.48\textwidth}
		\centering
		\includegraphics[width=\textwidth]{./../lab6_images/exp_3:_divergent_flow_t8.00.png}
		\caption{Скалярное поле в момент времени $t=8$.}
				\label{fig:exp3_8}
	\end{minipage}
\end{figure}

\newpage
Финальное распределение представлено на рис. \ref{fig:lg_div_finall}.
\begin{figure}
	\centering
	\includegraphics[width=0.7\textwidth]{./../lab6_images/exp_3:_divergent_flow_t10.00.png}
	\caption{Финальное распределение частиц.}
	\label{fig:lg_div_finall}
\end{figure}

Траектории всех точек представлены на рис. \ref{fig:lg_div_tr}.

\begin{figure}[H]
	\centering
	\includegraphics[width=0.7\textwidth]{./../lab6_images/exp_3:_divergent_flow_trajectories.png}
	\caption{Траектории частиц.}
	\label{fig:lg_div_tr}
\end{figure}

\newpage


\section{Выводы}

Метод конечных разностей позволяет качественно решать задачу переноса при условии соблюдения критерия CFL. Использование upwind-схем позволяет избежать осцилляций, но при этом может вызывать численное диффузионное размытие. Метод легко масштабируется и хорошо подходит для моделирования задач с заданным полем скоростей.

Метод лагранжевых частиц обеспечивает высокую точность отслеживания переноса вещества или других скалярных величин в заданном поле скоростей. Вместо аппроксимации функции на фиксированной сетке, как в методе конечных разностей, метод отслеживает движение частиц по траекториям, определяемым полем скоростей. Это позволяет избежать численного диффузионного размытия и сохранять чёткие границы. Метод особенно эффективен при моделировании переноса в сильно неоднородных или турбулентных потоках.

Также отметим, что скорость вычилений этих методов кратно разные: если метод вверх по потоку занимает в среднем 54 сек. на эксперимент, то метод лагранжевых частиц - уже 23.5 сек.