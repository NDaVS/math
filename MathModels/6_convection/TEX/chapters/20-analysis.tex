\chapter{Анализ модели и аналитическое решение двумерного уравнения переноса}

\section{Математическая постановка задачи}

Рассмотрим задачу переноса скалярной величины $u(x, y, t)$ в двумерной области $\Omega \subset \mathbb{R}^2$ при отсутствии источников и стоков. Модель описывается линейным уравнением переноса  в частных производных первого порядка:

\begin{equation}
	\frac{\partial u}{\partial t} + v_x(x, y) \frac{\partial u}{\partial x} + v_y(x, y) \frac{\partial u}{\partial y} = 0,
	\label{eq:transport_eq}
\end{equation}
где $v_x(x, y), v_y(x, y)$ — компоненты заданного векторного поля скорости переноса $\vec{v}(x, y)$.

\textbf{Начальное условие:}
\begin{equation}
	u(x, y, 0) = u_0(x, y), \quad (x, y) \in \Omega.
	\label{eq:initial_condition}
\end{equation}

\textbf{Граничные условия} задаются на входящих границах области, то есть там, где вектор скорости направлен внутрь области:

\begin{equation}
	u(x, y, t) = g(x, y, t), \quad (x, y) \in \partial \Omega^{-}, \quad t > 0,
\end{equation}
где $\partial \Omega^{-} = \{ (x, y) \in \partial \Omega \mid \vec{v}(x, y) \cdot \vec{n}(x, y) < 0 \}$ и $\vec{n}$ — внешняя нормаль к границе.

\section{Классификация уравнения и анализ характеристик}

Уравнение переноса \eqref{eq:transport_eq} относится к классу гиперболических уравнений первого порядка. Оно описывает чистое перемещение вещества без учета диффузии или реакций.

Анализ уравнения удобно производить с использованием метода характеристик. Вдоль характеристик, т.е. кривых в пространстве $(x, y, t)$, решение $u$ сохраняется неизменным.

Пусть $(x(t), y(t))$ — характеристика, тогда:

\begin{equation}
	\frac{dx}{dt} = v_x(x, y), \quad \frac{dy}{dt} = v_y(x, y), \quad \frac{du}{dt} = 0.
	\label{eq:characteristics}
\end{equation}

Таким образом, $u$ вдоль траектории движения частиц остается постоянным, и решение можно выразить через начальное распределение $u_0$.

\section{Аналитическое решение при постоянной скорости}

Рассмотрим частный случай, когда поле скорости постоянно:
\[
\vec{v} = (a, b), \quad a, b \in \mathbb{R}.
\]

В этом случае уравнение \eqref{eq:transport_eq} принимает вид:

\begin{equation}
	\frac{\partial u}{\partial t} + a \frac{\partial u}{\partial x} + b \frac{\partial u}{\partial y} = 0.
	\label{eq:const_velocity}
\end{equation}

Характеристики представляют собой прямые линии:

\[
x(t) = x_0 + at, \quad y(t) = y_0 + bt.
\]

Поскольку $u$ сохраняется вдоль этих линий, решение имеет вид:

\begin{equation}
	u(x, y, t) = u_0(x - at, y - bt).
	\label{eq:analytical_solution}
\end{equation}

\textbf{Пример.} Пусть
\[
u_0(x, y) = \exp\left(-\alpha[(x - x_0)^2 + (y - y_0)^2]\right),
\]
где $\alpha > 0$ — параметр ширины, $(x_0, y_0)$ — центр гауссианы. Тогда

\begin{equation}
	u(x, y, t) = \exp\left(-\alpha[(x - at - x_0)^2 + (y - bt - y_0)^2]\right),
\end{equation}
что соответствует движению начального профиля с постоянной скоростью $(a, b)$.

\section{Выводы по аналитическому анализу}

\begin{itemize}
	\item Уравнение переноса описывает перемещение профиля функции без изменения формы (в отсутствии источников).
	\item Решение можно выразить в аналитическом виде в случае постоянной скорости.
	\item В общем случае решение строится вдоль характеристик, что важно для построения численных схем.
	\item Важным элементом анализа является определение направлений входа потока — это определяет постановку граничных условий.
\end{itemize}
