\chapter{Анализ математической модели}
\section{Аналитическое решение}
Рассмотрим общее линейное уравнение первого порядка:

\[
\frac{\partial u}{\partial t} + v \frac{\partial u}{\partial x} = f(x,t).
\]

Рассмотрим характеристические кривые $x(t)$, вдоль которых уравнение превращается в обыкновенное дифференциальное:

\[
\frac{dx}{dt} = v \quad \Rightarrow \quad x = vt + \xi,
\]
где $\xi$ — параметр, определяющий начальную точку характеристики при $t=0$: $x(0) = \xi$.

На этой характеристике уравнение сводится к:

\[
\frac{d}{dt} u(x(t), t) = f(x(t), t).
\]

Интегрируя по $t$ от $0$ до $t$:

\[
u(x(t), t) = u(\xi, 0) + \int_0^t f(x(s), s)\, ds.
\]

Поскольку $x(s) = vs + \xi = x - v(t - s)$ (так как $x = vt + \xi$), получаем:

\[
u(x,t) = \varphi(x - vt) + \int_0^t f(x - v(t - s), s)\, ds.
\]

Это решение корректно, если характеристика проходит через начальную линию $t=0$, то есть если $x - vt \ge 0$.

\section*{Учет граничного условия при $x < vt$}

Если $x - vt < 0$, то характеристика не пересекает ось $t=0$, а входит в область из граничной точки $x = 0$. Тогда необходимо использовать граничное условие.

Для этого найдем момент времени, когда характеристика, проходящая через $(x,t)$, пересекает ось $x=0$:

\[
x = v t_0 \quad \Rightarrow \quad t_0 = t - \frac{x}{v}.
\]

Соответствующее значение $u$ на $x=0$ и $t_0 = t - \frac{x}{v}$: $u(0, t_0) = \psi(t - \frac{x}{v})$.

Тогда вдоль характеристики:

\[
u(x,t) = \psi\left(t - \frac{x}{v}\right) + \int_{t - \frac{x}{v}}^t f(x - v(t - s), s)\, ds.
\]

\section*{Полное аналитическое решение}

Таким образом, общее решение задачи \eqref{eq:sys} имеет вид:

\[
u(x,t) =
\begin{cases}
	\varphi(x - vt) + \displaystyle\int_0^t f(x - v(t - s), s)\, ds, & \text{если } x \ge vt, \\
	\psi\left(t - \dfrac{x}{v} \right) + \displaystyle\int_{t - \frac{x}{v}}^t f(x - v(t - s), s) \, ds, & \text{если } x < vt.
\end{cases}
\]

\section{Численное решение}
Система (\ref{eq:sys}) является системой обыкновенного дифференциального уравнения в частных производных.
Для численного решения применяется методы конечных разностей\cite{Turchak2003}. Введём равномерную сетку:
\[
x_i = i \Delta x, \quad t^j = j \Delta t, \quad i = 0, \dots, N_x, \quad j = 0, \dots, N_t,
\]
где $\Delta x = \frac{L}{N_x}$, $\Delta t = \frac{T}{N_t}$. Обозначим $u_i^j = u(x_i, t^j)$ и $f_{i}^i = f (x_{i}, t^i)$.

Будем рассматривать три схемы решения: явную\cite{Turchak2003}, неявную\cite{Turchak2003} и "вверх по потоку" \cite{Patankar1984}.
В качетве профилья волны выберем следующую функцию:
\[
	\phi(x) = 4\exp(-100 x^4).
\]
Для простоты используем нулевые граничные условия и равномерную среду ($f_{i}^j = 0$). 
Скорость $v$ примем 1. Расстояние $L$ равняется 5. Соответственно, максимальное значение времи тоже равно 5.
Шаг по пространству возьмём 0.05, а по времени - 0.001. Такой выбор удовлетворяет условиям устойчивости всех трёх методов (\ref{eq:ust1} - \ref{eq:ust2}) и не требует больших вычислительных ресурсов.





