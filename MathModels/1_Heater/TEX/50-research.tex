\chapter{Построение математической модели}
\section{Модель без термо регулятора}
Основной характеристикой нагревательного прибора является температура. При включенном нагревателе она изменяется со временем. Нас интересует зависимость изменения температуры ($[T]$ = K) от веремни ($[t]=c$): $T(t)$.

Предположим, что нагреватель состоит из одного материала, температура окружающей среды постоянная и равна $T_{env}$. Также отметим, что масса окружающей среды намного больше массы нагревательного прибора (паяльника): $m_{env} >> m_H$ .

Процесс нагревания описыватеся изменением количеством внутренней энергии тела ($\Delta Q$, $[Q]=\text{Дж}$)  от изменении температуры ($\Delta T$):
\begin{equation}
	\Delta Q = cm\Delta T,
	\label{eq:heat_energy}
\end{equation}
где $c$ - удельная теплоёмкость тела $\left(\frac{\text{Дж}}{\text{кг}\cdot\text{К}}\right)$, m - масса нагревателя (кг).

Нагревательный прибор использует электрический ток для увеличения внутренней энергии:
\begin{equation}
	\Delta Q_1 = P \Delta t,
	\label{eq:electrical_energy}
\end{equation}
где $P$ - мощность (Вт).

На изменение внутренней энергии также влияют входящие и исходящие тепловые потоки.
На единицу площади за единицу времени исходящий поток будет изменять энергию на величину $-kT$, 
а входящий - на величину $kT_{env}$, где $k$ - коэффициент теплопередачи, характерынй для данной конструкции нагревательного прибора $\left(\frac{\text{Вт}}{\text{м}^2\text{К}}\right)$. С учётом этих явлений, внутренняя энергия будет изменяться на следующую величину:
\begin{equation}
	\Delta Q_2 = -kS(T-T_{env})\Delta t.
	\label{eq:heat_transfer}
\end{equation}

Кроме этих явлений, согласно закону Стефана-Больцмана, любое тело, нагретое выше абсолютного нуля за единицу веремени на единицу площади излучает энергию равную $-\sigma T^4$, где $\sigma \approx 5.68 \cdot 10 ^{-8} \left(\frac{\text{Вт}}{\text{м}^2\cdot\text{К}^2}\right) $ - постоянная Стефана-Больцмана. Аналогично, излучение поступает из кружающей среды, равное $\sigma T^4_{env}$.
Тогда, измненение внутренней энергии, вызванного этим процессом, равно:
\begin{equation}
	\Delta Q_3 = -\sigma S (T^4 - T^4_{env}) \Delta t.
	\label{eq:stefan_boltzmann}
\end{equation}

Суммируя все потоки энергии, получаем уравнение теплового баланса (см. \ref{eq:heat_energy}, \ref{eq:electrical_energy}, \ref{eq:heat_transfer}, \ref{eq:stefan_boltzmann}):
 
 \begin{equation}
 	cm\Delta T = P\Delta t - k S (T-T_{env})\Delta t - \sigma S (T^4 - T^4_{env})\Delta t .
 	\label{eq:temp_balance}
 \end{equation}
 
 Разделим, обе части уравнения (\ref{eq:temp_balance}) на $cm\Delta t$ и совершим предельный переход $\Delta t \rightarrow 0$:
 \begin{equation}
 	\frac{dT}{dt} = \frac{P - kS(T-T_{env}) - \sigma S (T^4 - T^4_{env})}{cm}
 	\label{eq:dif_temp_balance}
 \end{equation}
 
 Таким образом, мы получили дифференциальное уравнение теплового баланса, которое описывает поведение температуры нагрвателя. 
 Для нахождения единственного достаточно ввести начальное условие: $T(0) = T_0$.
 \section{Модель с терморегулятором}
 
 Для предотвращения перегрева нагревателя, целесообразно установить терморегулятор, которые будет выключать нагреватель при достижении максимальной температуры. 
 Для этого достаточно ввести функцию, которая будет отключать нагреватель, когда температура больше максимально установленной ($T_{max}$), и включать, при достижении минимальной установленной температуры($T_{min})$.
 \begin{equation}
 	I(T, T_{min}, T_{max}) = 
 	\begin{cases}
 		1, & T < T_{min} \\
 		0, & T > T_{max}
 	\end{cases}.
 	\label{eq:switcher}
 \end{equation}
 
 Добавляя (\ref{eq:switcher}) в (\ref{eq:dif_temp_balance}) получим:
  \begin{equation}
 	\frac{dT}{dt} = \frac{P\cdot I(T, T_{min}, T_{max}) - kS(T-T_{env}) - \sigma S (T^4 - T^4_{env})}{cm}
 	\label{eq:dif_temp_balance_with_switcher},
 \end{equation}

 
 
 
