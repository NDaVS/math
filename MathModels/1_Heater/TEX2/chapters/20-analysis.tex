\chapter{Анализ математической модели}
Сперва найдём точки равновесия уравнения (\ref{eq:dif_temp_balance}).

Заметим, что удельная теплоёмкость ($c$) и масса ($m$) нагревателя находятся в знаменателе.
Это значит, что эти параметры не влияют на точки равновесия, но регулируют скорость изменения температуры в самом уравнении.

В начальный момент времени уравнение будет иметь следующий вид:

 \begin{equation}
	\frac{dT}{dt} = \frac{P}{cm} > 0.
	\label{eq:start_point_dif_temp_balance}
\end{equation}

На основе (\ref{eq:start_point_dif_temp_balance}) можно сделать вывод, что температура будет увеличиваться. 
Со временем, отрицательные слагаемые будут увеличиваться по модулю, что вызовет замедление увеличения скорости роста температуры. 

Это уменьшение скорости роста температуры, будет продолжаться до тех пор, пока отрицательные слагаемые, отвечающие за изменения внутренней энергии (\ref{eq:heat_transfer}, \ref{eq:stefan_boltzmann}), не уравновесят увеличение внутренней энергии от работы электрического тока (\ref{eq:heat_energy}). Достигнутая температура будет максимальной и постоянной до тех пор, пока нагреватель будет включен.

Отметим, что достигнутая максимальная температура будет единственной. Это связано с тем, что дальнейший прирост внутренней энергии компенсируется соответствующим ростом по модулю отрицательных слагаемых.

\section{Вычисление точек покоя}
Для вычисления точек покоя выберем параметры нагревательного элемента.
В качестве него возьмём паяльник с радиусом 0.003м и длиной 0.05м.
\[
	P=35\text{Вт}, m = 0.25\text{кг}, c=375\frac{\text{Дж}}{\text{кг}\cdot\text{К}},k = 2,
\]

\[
	 S = 2\pi r  h = 0.00094 \text{м}^2,T_{env}= 296 \text{К}
\]

Точные значения точек равновесия для этих параметров будут такие:

 \[
T_1 = -917, T_2 = 895.53,  T_{3,4} = 10.74 \pm 906.4i.
\]

Получили только одно положительное значение. Проверим его, использовав метод первого приближения. Обозначив правую часть дифференциального уравнения за R, получим:
\begin{equation}
	\frac{dR}{dT} = - 0.0017
	\label{eq:first_check}.
\end{equation}
Значение отрицательное, следовательно, положение устойчиво (система возвращается в равновесие при малых возмущениях). 
