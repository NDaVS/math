\chapter{Постановка задачи}
Целью данной работы является построение математической модели Лотки-Вольтерра, характеризующую взаимодействие двух видов типа "хищник-жертва". 

Для построения модели необходимо учитывать следующие параметры:

\begin{itemize}
	\item Численность жертв \( x(t) \) — популяция жертв в момент времени \( t \).
	\item Численность хищников \( y(t) \) — популяция хищников в момент времени \( t \).
	\item Скорость размножения жертв \( \alpha \) — коэффициент, определяющий, как быстро популяция жертв увеличивается в отсутствие хищников.
	\item Смертность хищников \( \gamma \) — коэффициент, определяющий, как быстро популяция хищников уменьшается в отсутствие жертв.
	\item Коэффициент смертности жертв от хищников \( \beta \) — показывает, как взаимодействие с хищниками влияет на численность жертв.
	\item Коэффициент конверсии пойманной жертвы в новых хищников \( \delta \) — определяет, как количество пойманных жертв увеличивает популяцию хищников.
\end{itemize}

\section{Формализация}
Для анализа взаимодействия популяций необходимо вывести систему дифференциальных уравнений, описывающих динамику численности жертв и хищников во времени. Система будет основываться на предположении, что обе популяции изолированы и взаимодействуют только друг с другом.

При отсутствии хищников жертвы размножаются неограниченно.
Хищники при отсутствии жертв вымирают.


Решение модели позволит проанализировать динамику изменения популяций в изолированной системе, выявить закономерности взаимодействия между видами и предсказать возможные сценарии развития экосистемы.

\chapter{Построение математической модели}

Основная идея модели заключается в том, что численность хищников зависит от доступности жертв, а популяция жертв ограничивается воздействием хищников \cite{bratush2010}.

Будем рассматривать взаимодействие популяций во времени. 
Из формализации задачи имеем: $x(t)$ - популяция жертв, а $y(t)$ — популяция хищников. 
Единицы измерения в данной системе предлагается опустить, измеряя популяции некоторым числом особей. 
Модель обладает физическим смыслом при $x(t) \ge 0, y(t) \ge 0$.

Предположим, что популяции изолированы. 
В таком случае популяция жертв будет увеличиваться, и скорость её роста также будет расти пропорционально числу особей в популяции.

Популяция хищников, напротив, со временем будет уменьшаться: в отсутствие пропитания особы могут гибнуть из-за нехватки ресурсов. 
Заметим, что абсолютное число погибающих хищников будет пропорционально их численности.

На основании этих рассуждений можно построить следующую модель:

\begin{equation}
	\begin{cases}
		\frac{dx}{dt} = \alpha x, \\
		\frac{dy}{dt} = -\gamma y,
	\end{cases}
	\label{eq:basic_model}
\end{equation}
где $x$ — численность жертв, $y$ — численность хищников, $\alpha > 0$ — скорость размножения жертв, $\gamma > 0$ — смертность хищников.

Теперь рассмотрим, как взаимодействие популяций будет влиять на их численность.

Изменение численности популяций зависит от количества особей обоих видов. 
Чем больше численность хищников в замкнутой системе, тем быстрее будет уменьшаться популяция жертв. 
В то же время увеличение численности популяции жертв повлечёт за собой рост численности хищников, так как появляется избыток питания, что позволяет существовать большему количеству хищников.

Таким образом, можно сделать вывод, что скорость изменения обеих популяций будет зависеть от численности обоих видов: увеличение количества хищников влечёт уменьшение числа жертв, уменьшение числажертв — уменьшение числа хищников и наоборот.

Учитывая эти рассуждения, дополним систему (\ref{eq:basic_model}) новыми составляющими:

\begin{equation}
	\begin{cases}
		\frac{dx}{dt} = \alpha x - \beta x y,\\
		\frac{dy}{dt} = \delta x y - \gamma y,
	\end{cases}
	\label{eq:model}
\end{equation}
где $\beta > 0$ — коэффициент смертности жертв от встречи с хищником, $\delta > 0$ — коэффициент конверсии пойманной жертвы в новых хищников (прирост за счёт избытка пищи).

Таким образом, построенная модель конкуренции является системой двух обыкновенных дифференциальных уравнений. Для получения единственного решения необходимо построить задачу Коши (добавить начальные условия):

\begin{equation}
	\begin{cases}
		x(0) = x_0,\\
		y(0) = y_0.
	\end{cases}
	\label{eq:terms}
\end{equation}

Эта система демонстрирует колебания численности обеих популяций, которые могут напоминать реальные циклы в природе, например взаимодействие рыси и зайца. Регуляторами взаимодействий являются коэффициенты $\alpha, \ \beta, \ \delta, \ \gamma$. 

Однако модель является упрощённой и не учитывает многие факторы, такие как насыщение хищников, изменения среды и взаимодействие других видов.