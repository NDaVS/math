\chapter{Построение математической модели}
Основная идея модели заключается в том, что численность хищников зависит от доступности жертв, а популяция жертв ограничивается воздействием хищников. 

Будем рассматривать взаимодействие популяций во времени. Скажем, что $x(t)$ - отражает популяцию жерт, $y(t)$ - популяцию хищников. Единицы измерения в данной системе предлагается опустить, измеряя популяции некоторым числом особей. Модель обладает физическим смыслом при $x(t) > 0, y(t) > 0$.

Предположим, что популяции изолированы. В таком случае, популяция жерт будет увеличиваться, и скорость роста также будет рости пропорционально числу особей в популяции. 
Популяция хищников, напротив, со временем будет уменьшаться - в отсутствии пропитания, особи могут гибнуть от конкуренции внутри популяции. Заметим, что скорость смертности будет уменьшаться со временем пропорционально количеству особей в популяции.

На основании этих рассуждений можно построить следующую модель:


	\begin{equation}
		\begin{cases}
			\frac{dx}{dt} = \alpha x \\
			\frac{dy}{dt} = -\gamma y
		\end{cases},
		\label{eq:basic_model}
	\end{equation}
где $x$ - численность жертв, $y$ - численность хищников, $\alpha > 0$ - скорость размножения жертв,  $\gamma > 0$ - смертность хищников.

Теперь рассмотрим как взаимодействие популяций будет влиять на их "объём".

Скорость изменения будет зависить от объёма обоих популяций. Чем больше численность хищников в замкнутой системе, тем скорее будет уменьшаться популяция жертв.
В тоже время, увеличение численности популяции жерт повлечёт за собой увеличение числа хищников, т.к. появляется избыток питания, что позволяет существовать большему количеству хищников.

Таким образом, можно сделать вывод, что скорость изменения обоих популяций будет изменяться пропорционально объёму обоих популяций: увеличение количества хищников влечёт уменьшение количества жерт, уменьшение количества жерт влечёт уменьшение количества хищников и наоборот.

Учитывая эти рассуждения дополним систему (\ref{eq:basic_model}) новыми составляющими:
\begin{equation}
	\begin{cases}
		\frac{dx}{dt} = \alpha x - \beta x y\\
		\frac{dy}{dt} = \delta x y  - \gamma y
	\end{cases},
	\label{eq:model}
\end{equation}
где $\beta > 0$ - коэффициент смертности от встречи с хищником, $\delta > 0$ - коэффициент конверсии пойманной жертвы в новых хищников (прирост за счёт избытка пищи).

Таким образом, потроенная модель конкуренции является системой двух обыкновенных дифференциальных уравнений.
Для получения единственного решения необходимо построить задчу Коши (добавить начальные условия):
\begin{equation}
	\begin{cases}
		x(0) = x_0\\
		y(0) = y_0
	\end{cases}
	\label{eq:terms}
\end{equation}

Эта система демонстрирует периодические колебания численности обеих популяций, которые могут напоминать реальные циклы в природе, например, взаимодействие рыси и зайца. Однако модель является упрощённой и не учитывает многие факторы, такие как насыщение хищников, изменения среды и конкуренцию внутри видов.


 
 
 
