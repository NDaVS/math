\section{Введение}
Взаимодействие хищников и жертв — один из ключевых процессов, определяющих структуру и динамику экосистем. Исследование этих взаимоотношений важно не только для фундаментальной экологии, но и для решения прикладных задач, таких как управление популяциями управления популяциями, сохранения биоразнообразия, прогнозирования последствий антропогенного воздействия и разработки устойчивых стратегий рыболовства и сельского хозяйства.

Классическая модель Лотки-Вольтерра, предложенная в начале XX века, стала основой для изучения динамики популяций.
Однако в реальных экосистемах взаимодействия между видами гораздо сложнее, чем это описывает базовая модель. Изменения среды обитания, конкуренция за ресурсы, сезонные колебания и эволюционные адаптации могут существенно влиять на устойчивость и цикличность популяций. 

В данной работе изучается классическая модель Лотки-Вольтерра без учёта дополнительных факторов.
