\section{Введение}
Взаимодействие хищников и жертв — один из ключевых процессов, определяющих структуру и динамику экосистем. Исследование этих взаимоотношений важно не только для фундаментальной экологии, но и для решения прикладных задач: управления популяциями, сохранения биоразнообразия, прогнозирования последствий антропогенного воздействия и разработки устойчивых стратегий рыболовства и сельского хозяйства.

Классическая модель Лотки-Вольтера, предложенная в начале XX века, стала основой для изучения динамики популяций. Однако в реальных экосистемах взаимодействия между видами гораздо сложнее, чем описано в базовой модели. Изменения среды обитания, конкуренция за ресурсы, сезонные колебания и эволюционные адаптации могут существенно влиять на устойчивость и цикличность популяций. В связи с этим актуальной задачей остается развитие и уточнение математических моделей, способных более точно описывать реальные процессы в природе.

Современные исследования направлены на расширение классической модели за счет учета пространственной структуры популяций, временных задержек, стохастических факторов и нелинейных эффектов. Такие подходы позволяют не только глубже понять механизмы экосистемных взаимодействий, но и разрабатывать научно обоснованные методы управления природными ресурсами и защиты исчезающих видов.