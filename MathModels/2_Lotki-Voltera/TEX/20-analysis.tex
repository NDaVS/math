\chapter{Анализ математической модели}
Очевидным является тривиальное решение (0,0).

Найдём точки равновесия уравнения (\ref{eq:model}).
 \begin{equation}
	\begin{cases}
		\frac{dx}{dt} = 0 \\ 
		\frac{dy}{dt} = 0
	\end{cases}
	\Rightarrow
	\begin{cases}
			\alpha x - \beta x y = 0\\
			\delta x y  - \gamma y = 0
	\end{cases}
	\Rightarrow
	\begin{cases}
		x_{bal} = \frac{\gamma}{\delta}\\
		y_{bal} = \frac{\alpha}{\beta}
	\end{cases}
	.
	\label{eq:balance_points}
\end{equation}

Теперь проанализируем устойчивость решений, применив метод первого приближения. 
Для этого построим матрицу Якоби (\textbf{$J$}), найдём её cобственные значения, подставив точки равновесия, и по ним укажем тип устойчивости.

\begin{equation}
	J = \begin{pmatrix}
		\alpha - \beta y & - \beta x \\
		\delta y & - \gamma + \delta x
		\label{eq:jacobi}
	\end{pmatrix}
\end{equation}
Подставляем тривиальное решение:
\begin{equation}
	J\big|_{(0,0)} = 
	\begin{pmatrix}
		\alpha & 0 \\
		0 & -\gamma
	\end{pmatrix}
	\Rightarrow \lambda_1 = -\gamma , \lambda_2 = \alpha
	\label{eq:EV_triv}
\end{equation}

Заметим, что собственные значения вещественны, но различны по знаку. 
Это означает, что точка $(0,0)$ - является седловой точкой.

Подставим нетривиальную точку равновесия:
\begin{equation}
	J\big|_{(x_{bal},y_{bal})} = 
	\begin{pmatrix}
		0 &  - \frac{\beta \gamma}{\delta} \\
		\frac{\delta \alpha}{\beta} & 0
	\end{pmatrix}
	\Rightarrow \lambda_{1,2} = \pm i \sqrt{\alpha\gamma}
	\label{eq:EV_bal}
\end{equation}

Собственные значения в точке равновесия $(x_bal, Y_bal)$ полностью мнимые. 
Следовательно точка является неасимптотически устойчивой. Изменение величин в ней не будет происходить, хотя на удалении от неё будут появляться циклы

Теперь найдём первый интеграл. Для этого поделим первое уравнение на второе и разделим переменные:
\begin{equation}
	\frac{dx}{dy} = \frac{\alpha x - \beta yx}{\delta xy - \gamma y} 
	\Rightarrow
	\left(\frac{\alpha}{y} - \beta\right) dy + \left(\frac{\gamma}{x} - \delta\right)dy = 0.
	\label{eq:diff_FI}
\end{equation}
Интегрируя, получим следующий результат:
\begin{equation}
	\alpha \ln(|y|) - \beta y + \gamma \ln(|x|) - \delta x = \text{const}.
	\label{eq:first_int}
\end{equation}

Выражение (\ref{eq:first_int}) показывает, что соотношение величин будет постоянным с течением времени. Следовательное, зная, что объёмы популяций меняются, в какой-то момент соотношения вренутся к начальным значениям.
