\chapter{Анализ математической модели}

Очевидным является тривиальное решение $(0,0)$.

Найдём точки равновесия уравнения (\ref{eq:model}):

\begin{equation} 
	\begin{cases} 
		\frac{dx}{dt} = 0 \\ 
		\frac{dy}{dt} = 0 
	\end{cases} 
	\Rightarrow 
	\begin{cases} 
		\alpha x - \beta x y = 0\\ 
		\delta x y - \gamma y = 0 \end{cases} 
		\Rightarrow 
		\begin{cases} 
			x_{bal} = \frac{\gamma}{\delta}\\ 
			y_{bal} = \frac{\alpha}{\beta} 
		\end{cases}. 
		\label{eq:balance_points} 
\end{equation}

Теперь проанализируем устойчивость решений, применив метод первого приближения.
Для этого построим матрицу Якоби (\textbf{$J$}), найдём её собственные значения, подставив точки равновесия, и по ним укажем тип устойчивости.

\begin{equation} 
	J = \begin{pmatrix} \alpha - \beta y & - \beta x \\ \delta y & - \gamma + \delta x 
	\end{pmatrix}. \label{eq:jacobi} \end{equation}

Подставляем тривиальное решение:

\begin{equation} J\big|_{(0,0)} = \begin{pmatrix} \alpha & 0 \\ 0 & -\gamma \end{pmatrix} \Rightarrow \lambda_1 = \alpha, \quad \lambda_2 = -\gamma. \label{eq:EV_triv} \end{equation}

Заметим, что собственные значения вещественны, но различны по знаку.
Это означает, что точка $(0,0)$ является седловой точкой.

Теперь подставим нетривиальную точку равновесия:

\begin{equation}
	J\big|_{(x_{bal}, y_{bal})} =
	\begin{pmatrix}
		\alpha - \beta \frac{\alpha}{\beta} & - \beta \frac{\gamma}{\delta} \\
		\delta \frac{\alpha}{\beta} & -\gamma + \delta \frac{\gamma}{\delta}
	\end{pmatrix} = 
	\begin{pmatrix}
		0 & - \frac{\beta \gamma}{\delta} \\
		\frac{\delta \alpha}{\beta} & 0
	\end{pmatrix}.
\end{equation}

Найдём её собственные значения, решая характеристическое уравнение:

\begin{equation}
	\det \left( J - \lambda I \right) = 0.
\end{equation}

Вычтем $\lambda I$ из $J$:

\begin{equation}
	J - \lambda I =
	\begin{pmatrix}
		-\lambda & - \frac{\beta \gamma}{\delta} \\
		\frac{\delta \alpha}{\beta} & -\lambda
	\end{pmatrix}.
\end{equation}

Найдём определитель:

\begin{equation}
	\det
	\begin{pmatrix}
		-\lambda & - \frac{\beta \gamma}{\delta} \\
		\frac{\delta \alpha}{\beta} & -\lambda
	\end{pmatrix}
	= (-\lambda) \cdot (-\lambda) - \left( - \frac{\beta \gamma}{\delta} \cdot \frac{\delta \alpha}{\beta} \right),
\end{equation}

\begin{equation}
	\lambda^2 - \left( - \frac{\beta \gamma}{\delta} \cdot \frac{\delta \alpha}{\beta} \right) = 0,
\end{equation}

\begin{equation}
	\lambda^2 - \frac{\beta \gamma \delta \alpha}{\delta \beta} = 0,
\end{equation}

\begin{equation}
	\lambda^2 - \alpha \gamma = 0.
\end{equation}

Решая квадратное уравнение:

\begin{equation}
	\lambda_{1,2} = \pm i \sqrt{\alpha\gamma}.
\end{equation}

Собственные значения в точке равновесия $(x_{bal}, y_{bal})$ чисто мнимые.
Следовательно, точка является неасимптотически устойчивой. Точка равновесия не является устойчивым фокусом, но малые отклонения приведут к замкнутым траекториям, соответствующим периодическим колебаниям численности популяций.

Теперь найдём первый интеграл системы. Для этого поделим первое уравнение на второе и разделим переменные:

\begin{equation} \frac{dx}{dy} = \frac{\alpha x - \beta x y}{\delta x y - \gamma y} \Rightarrow \left(\frac{\alpha}{y} - \beta\right) dy + \left(\frac{\gamma}{x} - \delta\right)dx = 0. \label{eq:diff_FI} \end{equation}

Интегрируя, получим:

\begin{equation} \alpha \ln |y| - \beta y + \gamma \ln |x| - \delta x = \text{const}. \label{eq:first_int} \end{equation}

Выражение (\ref{eq:first_int}) показывает, что соотношение численности популяций остаётся постоянным во времени.
Следовательно, хотя популяции изменяются, в какой-то момент их соотношение вернётся к первоначальному значению, что приводит к цикличному поведению системы.