\chapter{Анализ математической модели}
\section{Аналитическое решение}
Очевидным точным решением является смещенный профиль волны.
\[
	u(x,t) = \phi(x_{0} - vt)
\]

\section{Численное решение}
Система (\ref{eq:sys}) является системой обыкновенного дифференциального уравнения в частных производных.
Для численного решения применяется методы конечных разностей\cite{Turchak2003}. Введём равномерную сетку:
\[
x_i = i \Delta x, \quad t^j = j \Delta t, \quad i = 0, \dots, N_x, \quad j = 0, \dots, N_t,
\]
где $\Delta x = \frac{L}{N_x}$, $\Delta t = \frac{T}{N_t}$. Обозначим $u_i^j = u(x_i, t^j)$.

Будем рассматривать три схемы решения: явную\cite{Turchak2003}, неявную\cite{Turchak2003} и "вверх по потоку" \cite{Patankar1984}.
В качетве профилья волны выберем следующую функцию:
\[
	\phi(x) = 4\exp(-100 x^4).
\]
Для простоты используем нулевые граничные условия. 
Скорость $c$ примем 1. Расстояние $L$ равняется 5. Соответственно, максимальное значение времи тоже равно 5.
Шаг по пространству возьмём 0.05, а по времени - 0.001. Такой выбор удовлетворяет условиям устойчивости всех трёх методов (\ref{eq:ust1} - \ref{eq:ust2}) и не требует больших вычислительных ресурсов.





