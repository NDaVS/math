\section{Введение}
Одним из ключевых процессов в природе являются течения в жидких средах. 
Эти течения играют важную роль в переносе веществ и энергии из одной точки в другую. 
Примером могут служить течения в реках, морях и океанах, 
которые способствуют не только транспортировке питательных веществ, но и формированию экосистем.

Кроме того, аналогичные процессы происходят и в газообразной среде. 
В атмосфере воздух также движется, создавая ветры, которые переносят влагу, пыль и другие частицы. 
Эти воздушные течения влияют на климатические условия, распределение тепла и осадков, 
а также на миграцию животных и распространение семян растений.

Рассмотрим подобный перенос в двумерном пространстве-время.