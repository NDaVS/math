\section{Введение}
В нашем мире все процессы зависят от точки зрения наблюдателя. Разные наблюдатели могут видеть одни и те же явления по-разному. Например, если находиться в равномерно движущемся поезде и бросить теннисный мяч вниз, он будет отскакивать от пола, описывая прямолинейную траекторию для пассажира. Однако для наблюдателя, стоящего на земле и не движущегося вместе с поездом, траектория мяча будет выглядеть как парабола.

Аналогично, в случае движения на вращающейся поверхности, например, когда человек идет по карусели или вблизи полюса Земли, восприятие движения будет различаться в зависимости от положения наблюдателя.

Рассмотрим такое движение на вращяющейся поверхности диска и поверхности  Земли.