\chapter{Анализ математической модели}
\subsection*{Тип системы}
Данная система представляет собой систему линейных дифференциальных уравнений второго порядка. Если $\phi$ — постоянная, коэффициенты уравнений также постоянны. Если же $\phi = \phi(t)$, система становится с переменными коэффициентами.

\subsection*{Приведение к матричной форме}
Введём вектор состояния:

\[
\mathbf{r} = \begin{bmatrix} x \\ y \end{bmatrix}, \quad
\mathbf{v} = \begin{bmatrix} \dot{x} \\ \dot{y} \end{bmatrix}.
\]

Тогда уравнение можно записать как систему первого порядка:

\[
\begin{bmatrix} \dot{v}_x \\ \dot{v}_y \end{bmatrix}
=
\begin{bmatrix} 0 & 2\omega \cos(\phi) \\ -2\omega \cos(\phi) & 0 \end{bmatrix}
\begin{bmatrix} v_x \\ v_y \end{bmatrix}.
\]

Матрица системы:

\[
A = \begin{bmatrix} 0 & 2\omega \cos(\phi) \\ -2\omega \cos(\phi) & 0 \end{bmatrix}
\]

является антисимметричной, что указывает на вращательную динамику.

\subsection*{Собственные числа системы}
Собственные числа матрицы $A$ находятся из характеристического уравнения:

\[
\det \begin{bmatrix} -\lambda & 2\omega \cos(\phi) \\ -2\omega \cos(\phi) & -\lambda \end{bmatrix} = 0.
\]

Вычисляя определитель:

\[
\lambda^2 + 4\omega^2 \cos^2(\phi) = 0.
\]

Решая квадратное уравнение, получаем:

\[
\lambda = \pm 2\omega \cos(\phi) i.
\]

Так как собственные значения чисто мнимые, система имеет колебательное поведение.

\subsection*{Динамическое поведение}
Так как $\lambda$ чисто мнимые, движение будет периодическим. Возможны два случая:
\begin{itemize}
	\item Если $\phi$ — постоянная, решение представляет собой вращательное движение.
	\item Если $\phi = \phi(t)$ изменяется, возможны переходные режимы и сложные траектории.
\end{itemize}


\subsection*{Вывод}
\begin{itemize}
	\item Система описывает вращательное движение.
	\item Решения имеют колебательный характер (круговые или эллиптические траектории).
	\item Если $\phi$ меняется со временем, возможны сложные нелинейные эффекты.
\end{itemize}
