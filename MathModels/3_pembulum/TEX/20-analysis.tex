\chapter{Анализ математической модели}
Для анализа будем использовать линейную модель (\ref{eq:lin_model}).
Из курса дифференциальных уравнений известен вид решения:
$$\theta(t) = C_1 \sin(\omega t) + C_2 \sin(\omega t)$$.
Тогда частное решение будет иметь равняться:
$$\theta(t) = \frac{\theta_1}{\omega} \sin(\omega t) + \theta_0 \cos(\omega t).$$
Заметим, что при начальных значениях $\theta(0) = \pi n, \ n \in \mathbb{Z}, \ \dot{\theta}(0) = 0$ маятник будет покоиться.

Используя тригонометрическую идентичность, мы можем выразить комбинацию синуса и косинуса в виде одного синуса. Для этого применим формулу синуса суммы:

\[
\sin(x + y) = \sin x \cos y + \cos x \sin y.
\]

Таким образом, мы можем записать:

\[
\alpha(t) = \alpha_0 \cos(\omega t) + a_1 \omega \sin(\omega t).
\]

Теперь мы можем выразить это в виде одной синусоидальной функции:

\begin{equation}
	\alpha(t) = \rho \sin(\omega t + \phi)
	\label{eq:sin_value}
\end{equation}
где:

- \(\rho\) — амплитуда, которая определяется как:

\[
\rho = \sqrt{\alpha_0^2 + (a_1 \omega)^2}
\]

- \(\phi\) — фаза, которая определяется через соотношения:

\[
\rho \sin \phi = \alpha_0,
\]
\[
\rho \cos \phi = a_1 \omega.
\]

Как можно видеть из (\ref{eq:sin_value}), резальтатом будет являться синусоида с некоторым смещением, зависящим от начальных параметров.
