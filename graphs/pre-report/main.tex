\documentclass[14pt]{article}

\usepackage[T2A]{fontenc} % Кодировка шрифта
\usepackage[utf8]{inputenc} % Кодировка ввода
\usepackage[english,russian]{babel} % Языковые настройки
\usepackage{graphicx} % Для вставки изображений
\usepackage{amsmath} % Для использования математических формул
\usepackage{amssymb}
\usepackage{cancel}
\usepackage{amsfonts} % Для использования математических символов и шрифтов
\usepackage{titlesec} % Для настройки заголовков разделов
\usepackage{titling} % Для настройки титульной страницы
\usepackage{geometry} % Для настройки размеров страницы
\usepackage{pgfplots}
\usepackage{type1cm}
\fontsize{16}{16}\selectfont

\pgfplotsset{compat=1.9}

\title{Расчет сопротивления между вершинами в графе}
\author{Вершинин Данил Алексеевич }
\date{\today}

\begin{document}
	
	\maketitle
	
	\section{Введение}
	В данной работе представлен расчет сопротивления между двумя вершинами произвольного графа с использованием законов Кирхгофа и метода решения системы линейных уравнений. Граф задан его матрицей смежности, и каждое ребро графа имеет единичное сопротивление.
	
	\section{Методика расчета}
	
	\subsection{Исходные данные}
	Матрица смежности графа:
	\[
	A = \begin{bmatrix}
		0 & 1 & 0 & 0 \\
		1 & 0 & 1 & 1 \\
		0 & 1 & 0 & 1 \\
		0 & 1 & 1 & 0
	\end{bmatrix}
	\]
	
	\subsection{Определение матрицы степеней}
	Матрица степеней \(D\) определяется исходя из количества рёбер, инцидентных каждой вершине:
	\[
	D = \begin{bmatrix}
		1 & 0 & 0 & 0 \\
		0 & 3 & 0 & 0 \\
		0 & 0 & 2 & 0 \\
		0 & 0 & 0 & 2
	\end{bmatrix}
	\]
	
	\subsection{Матрица Лапласа}
	Матрица Лапласа \(L\) вычисляется как разность \(D\) и \(A\):
	\[
	L = \begin{bmatrix}
		1 & -1 & 0 & 0 \\
		-1 & 3 & -1 & -1 \\
		0 & -1 & 2 & -1 \\
		0 & -1 & -1 & 2
	\end{bmatrix}
	\]
	
	\subsection{Расчет сопротивления между вершинами 1 и 4}
	
	Для расчета сопротивления между вершиной 1 (источник) и вершиной 4 (сток) модифицируем матрицу Лапласа, удалив строки и столбцы, соответствующие вершине 4.
	
	\[
	L' = \begin{bmatrix}
		1 & -1 & 0 \\
		-1 & 3 & -1 \\
		0 & -1 & 2
	\end{bmatrix}
	\]
	
	\subsection{Решение системы линейных уравнений}
	Система уравнений для определения потенциалов при токе 1 Ампер:
	\[
	\begin{bmatrix}
		1 & -1 & 0 \\
		-1 & 3 & -1 \\
		0 & -1 & 2
	\end{bmatrix}
	\begin{bmatrix}
		V_1 \\
		V_2 \\
		V_3
	\end{bmatrix}
	= \begin{bmatrix}
		1 \\
		0 \\
		0
	\end{bmatrix}
	\]
	
	Решив данную систему, мы находим, что:
	\[
	V_1 = \frac{5}{3}
	\]
	
	Следовательно, сопротивление между вершинами 1 и 4 равно \( \frac{5}{3} \) Ом.
	
	\section{Заключение}
	Методика, представленная в данной работе, демонстрирует использование матрицы Лапласа для расчета электрического сопротивления в модели, представленной графом. Результаты расчетов позволяют точно определить сопротивление между выбранными вершинами.
	
\end{document}
