\documentclass{article}

\usepackage[T2A]{fontenc} % Кодировка шрифта
\usepackage[utf8]{inputenc} % Кодировка ввода
\usepackage[english,russian]{babel} % Языковые настройки
\usepackage{graphicx} % Для вставки изображений
\usepackage{amsmath} % Для использования математических формул
\usepackage{amssymb}
\usepackage{cancel}
\usepackage{tikz}
\usepackage{amsfonts} % Для использования математических символов и шрифтов
\usepackage{titlesec} % Для настройки заголовков разделов
\usepackage{titling} % Для настройки титульной страницы
\usepackage{geometry} % Для настройки размеров страницы
\usepackage{pgfplots}
\pgfplotsset{compat=1.9}

% Настройка заголовков разделов
\titleformat{\section}
{\normalfont\Large\bfseries}{\arabic{section}}{1em}{}
\titleformat{\subsection}
{\normalfont\large\bfseries}{}{1em}{}

% Настройка титульной страницы
\setlength{\droptitle}{-3em} % Отступ заголовка
\title{\vspace{-1cm}Резистивное расстояние}
\author{Вершинин Данил Алексеевич}
\date{\today}

% Настройка размеров страницы
\geometry{a4paper, margin=2cm}

\begin{document}
	
	% Автоматическая генерация оглавления (см. далее)
	\maketitle
	%\tableofcontents
	%\chapter{Задачи}
	\section{Введение}
		Теория графов часто применяется для исследования электрических цепей различной топологии. Одной ее из задач является нахождение сопротивления между двумя эквивалентными узлами электрической сети.
	Данная задача рассматривается в контексте теории графов, т.к. электрическую цепь можно интепретировать связным графом, где узлом графа является узел сети, 
	а весом ребра между двумя узлами графа -- сопротивление между двумя узлами сети, которые являются прообразами узлов графа. С введением графовой интерпретации можно задать метрику для нахождения 
	расстояния между двумя произвольными узлами графа -- резистивное расстояние\cite{RD}. Резистивное расстояние между двумя веришинами простого связного графа G равно сопротивлению между двумя эквивалентными 
	точками электрической цепи, построенной путем замены ребра графа на сопротивление в 1 ом.
	В тоже время, для электрической цепи применим первый закон Киргхофа : "алгебраическая сумма токов в узле электрической цепи равна нулу". Этот закон также можно интерпретировать и на графы (см. теоретическое обоснование)
	
	Связь между Законами Киргхофа резистивным расстоянием начали исследовать Douglas J. Klein (Texas A\&M) и  M. Randic (Drake University)\cite{RD2} несколько десятилетий назад. Хоть их исследования были направлены в области химии, но они нашли отклик и в математической области. Кроме того, существует тесная связь с методом случайых блужданий в графе.
	
	В данной работе представлен алгоритм нахождения резистивного расстояния между двумя произвольными узлами сети для дальнейшей интерпритации и сравнения результата со случайными блужданиями в графе для поиска кратчашего пути.
	Таким образом, данная работа представляет собой начало исследования поиска оптимального пути в графе через интепретацию в электрическую цепь.
	
	\section{Теоретическое обоснование\cite{ss}}
	
	Предположим, что нам дан граф G(V,E)  с матрицей смежности M:
	\[M = \begin{pmatrix}
		0 & 1 & 0 & 0 \\
		1 & 0 & 1 & 1 \\
		0 & 1 & 0 & 1 \\
		0 & 1 & 1 & 0 \\
	\end{pmatrix}, \eqno (1)\]
	
	 Найдем матрицу степеней вершин $P$ этого графа. Для этого в каждой i-ой строке ($i  = \overline{1,n}$, где $n$ - длина строки или столбца матрицы смежности) сложим ее элементы и полученный результат запишем в новую нулевую матрицу P в качестве элемента $p_{i,i}$. Полученная матрица является матрицей степеней вершин для нашего графа G.
	\[P = \begin{pmatrix}
		1 & 0 & 0 & 0 \\
		0 & 3 & 0 & 0 \\
		0 & 0 & 2 & 0 \\
		0 & 0 & 0 & 2 \\
	\end{pmatrix}\] 
	 
	Вычитая из P матрицу M получим так называемую матрицу Киргхофа K, где по главной диагонали стоят степени вершин графа, а 
	остальные элементы являются сопротивлением до связных узлов (0 -- связи нет, -1 -- связь есть). Данную матрицу можно интепретировать, как систему уравнений Киргхофа для каждой из вершин. Для каждой строки выполнятеся первый закон Киргхофа.
	\[K = \begin{pmatrix}
		1 & -1 & 0 & 0 \\
		-1 & 3 & -1 & -1 \\
		0 & -1 & 2 & -1 \\
		0 & -1 & -1 & 2 \\
	\end{pmatrix}\] 
	
	
	Теперь можно рассматривать эту матрицу, как матрицу коэффициентов системы уравнений первого закона Киргхофа. 
	\newline
	$$Av=b,$$
	где $A$ -- матрица системы уравнений Киргхофа, $v$ -- напряжение в узлах графа, $b$ -- результирующее напряжение (с начальной вершиной в которой результируещее напряжение равно 1, в конечную с результирующем напряжением равным -1).
	
	Строки являются линейно зависимыми. Поэтому в чистом виде система не разрешима. Но для ее приближённого решения можно использовать один из частных случаев псевдообратной матрицы.
	Для этого используется матрица Мура-Пенроуза (псевдообратную матрицу), которая, как и обратная матрица, является частным случаем псевдо обратной матрицы. Достаточно провести ледующие действия:
	
	$$Av=b$$
	Слева каждую часть уравнения следует умножить на транспонированную матрицу A, тем самым квадратная матрица останется квадратной, но при этом произведения матриц не будет вырожденным, что позволит найти обратную матрицу для этого произведения.
	$$A^TAv=A^Tb$$
	Теперь, домножим слева обе части на обратную для $A^TA$ матрицу:
	$$(A^TA)^{-1}A^TAv=(A^TA)^{-1}A^Tb$$
	В левой части останется только вектор напряжений, т.к. произведение $(A^TA)^{-1}A^TA$ даст единичную матрицу $I$, правую часть отавим без изменений.
	$$v=(A^TA)^{-1}A^Tb =A^+b$$
	
	Таким образом, получен вектор напряжений на узлах. Для определения резистивного расстояни между двумя узлами достаточно вычесть из напряжения первого узла, напряжение второго и это (по модулю) будет являеться резистивным расстоянием между этими вершинами графа. Что в интепретации электрических цепей является сопротивлением между этими точками (узлами сети).
	
	
	\subsection{Пример 1}
	Пусть дана матрица смежности M графа G
		\[M = \begin{pmatrix}
		0 & 1 & 0 & 0 \\
		1 & 0 & 1 & 1 \\
		0 & 1 & 0 & 1 \\
		0 & 1 & 1 & 0 \\
	\end{pmatrix}, \eqno (1)\]

	\begin{figure}[h]
		\centering
		\begin{tikzpicture}[main/.style = {draw, circle}] 
			\node[main] (1) {$1$}; 
			\node[main] (2) [right of=1] {$2$};
			\node[main] (3) [right of=2] {$3$}; 
			\node[main] (4) [below of=2] {$4$};
			
			\draw (2) -- (4);
			\draw (3) -- (4);
			\draw (1) -- (2);
			\draw (2) -- (3);
			\draw (2) -- (1);
		\end{tikzpicture}
		\caption{Граф G с четырьмя вершинами, построенный по матрице смежности M}
	\end{figure}
	Найдём матрицу степеней вершин:
	$$P =\begin{pmatrix}
		1& 0& 0& 0&\\
		0& 3& 0& 0&\\
		0& 0& 2& 0&\\
		0& 0& 0& 2&\\
	\end{pmatrix}$$
	Вычтем из матрицы степеней матрицу смежности:
	$$P-M = L = \begin{pmatrix}
		1& -1& 0& 0&\\
		-1& 3& -1& -1&\\
		0& -1& 2& -1&\\
		0& -1& -1& 2&\\
		
	\end{pmatrix}$$
	Полученную матрицу можно интепретировать, как коэффициенты системы уравнений Киргхофа для неориентированного графа:
	$$Lv=b$$
	Теперь, по описанному выше алгоритму получаем псевдообратную матрицу:
	$$L^+ = (L^TL)L^T$$
	Опуская вычисления, получим:
	$$L^+=\begin{pmatrix}
		0.6875   &  -0.0625  &   -0.3125    & -0.3125\\
		-0.0625  &    0.1875  &   -0.0625   &  -0.0625 \\
		-0.3125   &  -0.0625   &   0.35416667 & 0.02083333 \\
		-0.3125   &  -0.0625    &  0.02083333 & 0.35416667
	\end{pmatrix}$$
	Найдём вектор напряжений v:
	$$v = L^+b = \begin{pmatrix}
		0.6875   &  -0.0625  &   -0.3125    & -0.3125\\
		-0.0625  &    0.1875  &   -0.0625   &  -0.0625 \\
		-0.3125   &  -0.0625   &   0.35416667 & 0.02083333 \\
		-0.3125   &  -0.0625    &  0.02083333 & 0.35416667
	\end{pmatrix} \begin{pmatrix}
		1\\0\\0\\-1
	\end{pmatrix}$$
	
	$v = (1 \  0 \ -1/3 \ -2/3)^T$
	Найдём разность потенциалов первого и четвертого узла:
	$$1+\frac{2}{3} =  \frac{5}{3}$$
	
	Проверка прямым расчётом:\newline
	От узла 1 к узлу 2:  1\newline
	От узла 2 к 4: $\left(\frac{1}{1} +  \frac{1}{1+1}\right)^{-1} = \frac{2}{3}$\newline
	Итог, от 1 к 4: $1+\frac{2}{3} = \frac{5}{3}$\newline
	Метод сработал.
	
	\section{Пример 2}
	Нам дана следующая матрица смежности:
	$$M=\begin{pmatrix}
		0& 1& 1 &1\\
		1& 0& 1 &1\\
		1& 1& 0 &1\\
		1& 1& 1 &0
	\end{pmatrix}$$
	
	\begin{figure}[h]
		\centering
		\begin{tikzpicture}[main/.style = {draw, circle}] 
			\node[main] (1) {$1$}; 
			\node[main] (2) [above of=1] {$2$};
			\node[main] (3) [below right of=1] {$3$}; 
			\node[main] (4) [below left of=1] {$4$};
			
			\draw (2) -- (4);
			\draw (3) -- (4);
			\draw (1) -- (4);
			\draw (1) -- (3);
			\draw (2) -- (3);
			\draw (2) -- (1);
		\end{tikzpicture}
		\caption{Графа G с четырьмя вершинами, построенный по матрице смежности М}
	\end{figure}
	
	Найдём матрицу степеней вершин:
	$$P =\begin{pmatrix}
		3& 0& 0& 0\\
		0& 3& 0& 0\\
		0& 0& 3& 0\\
		0& 0& 0& 3\\
		
	\end{pmatrix}$$
	Вычтем из матрицы степеней матрицу смежности:
	$$P-M = L = \begin{pmatrix}
		3& -1& -1& -1\\
		-1& 3& -1& -1\\
		-1& -1& 3& -1\\
		-1& -1& -1& 3\\
		
	\end{pmatrix}$$
	Полученную матрицу можно интепретировать, как коэффициенты системы уравнений Киргхофа для неориентированного графа:
	$$Lv=b$$
	Теперь, по описанному выше алгоритму получаем псевдообратную матрицу:
	$$L^+ = (L^TL)L^T$$
	Опуская вычисления, получим:
	$$L^+=\begin{pmatrix}
		0.1875 & -0.0625 & -0.0625 & -0.0625\\
		-0.0625 & 0.1875 & -0.0625 & -0.0625\\
		-0.0625 & -0.0625 & 0.1875 & -0.0625\\
		-0.0625 & -0.0625 & -0.0625 & 0.1875\\
		
	\end{pmatrix}$$
	Найдём вектор напряжений v:
	$$v = L^+ b = \begin{pmatrix}
		0.1875 & -0.0625 & -0.0625 & -0.0625\\
		-0.0625 & 0.1875 & -0.0625 & -0.0625\\
		-0.0625 & -0.0625 & 0.1875 & -0.0625\\
		-0.0625 & -0.0625 & -0.0625 & 0.1875\\
		
	\end{pmatrix} \begin{pmatrix}
		1\\
		0\\
		0\\
		-1
	\end{pmatrix}$$
	$$v = (0.25\ 0\ 0\ -0.25)^T$$
	Найдём разность потенциалов в первом узле и в четвёртом: $$0.25 - (-0.25) = 0.5$$
	Ответ: 0.5
	
	
	\section{Вывод}
	В данной работе исследовался метод нахождения резистивного расстояния между узлами графа с использованием матрицы Кирхгофа и псевдообратной матрицы Мура-Пенроуза. Было показано, что данный метод позволяет эффективно рассчитывать сопротивление между узлами электрической цепи, интерпретированной как связный граф. Проведенные примеры подтвердили корректность алгоритма и его применимость для решения задач на графах.
	
	Основные выводы работы:
	\begin{enumerate}
		\item Электрическая цепь можно представить в виде графа, где узлы сети являются вершинами графа, а сопротивления между узлами - весами рёбер.
		\item Матрица Кирхгофа, построенная на основе матриц смежности и степеней вершин, позволяет формализовать задачи теории электрических цепей в терминах линейной алгебры.
		\item Псевдообратная матрица Мура-Пенроуза эффективно применяется для решения систем линейных уравнений, возникающих в процессе расчета резистивного расстояния.
		\itemРезультаты, полученные с использованием этого метода, соответствуют классическим расчетам, что подтверждает его точность и надежность.
	\end{enumerate}
	

	Таким образом, предложенный метод можно использовать для анализа сложных электрических цепей и для поиска оптимальных путей в графах, что открывает новые возможности для применения теории графов в различных областях науки и техники.
	\bibliographystyle{plain}
	
	\bibliography{biba}
\end{document}