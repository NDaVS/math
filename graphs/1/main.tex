\documentclass{article}

\usepackage[T2A]{fontenc} % Кодировка шрифта
\usepackage[utf8]{inputenc} % Кодировка ввода
\usepackage[english,russian]{babel} % Языковые настройки
\usepackage{graphicx} % Для вставки изображений
\usepackage{amsmath} % Для использования математических формул
\usepackage{amssymb}
\usepackage{cancel}
\usepackage{tikz}
\usepackage{amsfonts} % Для использования математических символов и шрифтов
\usepackage{titlesec} % Для настройки заголовков разделов
\usepackage{titling} % Для настройки титульной страницы
\usepackage{geometry} % Для настройки размеров страницы
\usepackage{pgfplots}
\pgfplotsset{compat=1.9}

% Настройка заголовков разделов
\titleformat{\section}
{\normalfont\Large\bfseries}{\arabic{section}}{1em}{}
\titleformat{\subsection}
{\normalfont\large\bfseries}{}{1em}{}

% Настройка титульной страницы
\setlength{\droptitle}{-3em} % Отступ заголовка
\title{\vspace{-1cm}Резистивное расстояние}
\author{Вершинин Данил Алексеевич}
\date{\today}

% Настройка размеров страницы
\geometry{a4paper, margin=2cm}

\begin{document}
	
	% Автоматическая генерация оглавления (см. далее)
	\maketitle
	%\tableofcontents
	%\chapter{Задачи}
	\section{Введение}
	 Для решения задчи нахождения резистивного расстояния строится матрица Киргхофа (лапласиан) по первому закону Киргхофа. Так, элементы главной диагонали указывают количество входящего тока, а ненулевые элементы - направления исходящего тока.
	 
	 
	 У нас есть матрица, построенная по первому закону Киргхофа для неориентированного графа. Она имеет квадратную форму. Кроме того, она вырожденная. Следовательно мы не сможем найти точного решения, т.к. нам мешаются лишние элементы. 
	 
	 
	 Я нашёл два метода решения:
	 
	 1.    Можно вычеркнуть конечную строку (заземлить её). В таком случае минор 
	 матрицы будет невырожденным и можно решить эту систему, полкчив ответ в интересующем элементе вектора-ответа.
	 
	 
	 2.   Для задач подобного рода существует метод приближёного решения - через псевдообратную матрицу. Этот метод позволяет минимизировать отклонение от ответа (по итогу получается, что ответ в точности совпадёт с эксперементальными результатами; Этого можно добиться дополнительными преобразованиями)
	 
	 
	 Здесь будет описан только второй метод, т.к. первый является следствием из него.
	 
	 
	 \section{Описание Алгоритма\cite{RD}}
	 
	 
	 Изначально имеется матричное уравнение следующего вида:
	 $$Av=b,$$
	  где $A$ - матрица системы уравнений Киргхофа, $v$ - напряжение в узлах графа, $b$ - результирующее напряжение (с начальной вершиной =1, в конечную =-1)
	 
	 Обычный метод решения тут не применим. Решается данное матричное уравнение через матрицу Мура-Пенроуза (псевдообратную матрицу). Достаточно провести ледующие действия:
	 
	 $$Av=b$$
	 $$A^TAv=A^Tb$$
	 $$v=(A^TA)^{-1}A^Tb =A^+b$$
	 
	 Интерсное замечание: Такой метод решения достаточно популярен. Например, модобным образом считаются коэффициенты многомерного случая линейной регрессии.
	 
	 Теперь остаётся только вычислить эту матрицу, умножить на вектор и найти разность потенциалов узлов.
	 Таким образом, мы получим сопротивление между двумя вершинами графа.
	 \section{Пример 1}
	 Нам дана следующая матрица смежности:
	 $$M=\begin{pmatrix}
	 	0& 1& 0 &0\\
	 	1& 0& 1 &1\\
	 	0& 1& 0 &1\\
	 	0& 1& 1 &0
	 \end{pmatrix}$$
	  \begin{figure}[h]
	 	\centering
	 	\begin{tikzpicture}[main/.style = {draw, circle}] 
	 		\node[main] (1) {$1$}; 
	 		\node[main] (2) [right of=1] {$2$};
	 		\node[main] (3) [right of=2] {$3$}; 
	 		\node[main] (4) [below of=2] {$4$};
	 		
	 		\draw (2) -- (4);
	 		\draw (3) -- (4);
	 		\draw (1) -- (2);
	 		\draw (2) -- (3);
	 		\draw (2) -- (1);
	 	\end{tikzpicture}
	 	\caption{Пример графа с четырьмя вершинами, построенный по матрице смежности М}
	 \end{figure}
	 \newpage
	 Найдём матрицу степеней вершин:
	 $$G =\begin{pmatrix}
	 	1& 0& 0& 0&\\
	 	0& 3& 0& 0&\\
	 	0& 0& 2& 0&\\
	 	0& 0& 0& 2&\\
	\end{pmatrix}$$
	  Вычтем из матрицы степеней матрицу смежности:
	$$G-M = L = \begin{pmatrix}
		1& -1& 0& 0&\\
		-1& 3& -1& -1&\\
		0& -1& 2& -1&\\
		0& -1& -1& 2&\\
		
	\end{pmatrix}$$
	Полученную матрицу можно интепретировать, как коэффициенты системы уравнений Киргхофа для неориентированного графа:
	$$Lv=b$$
	Теперь, по описанному выше алгоритму получаем псевдообратную матрицу:
	$$L^+ = (L^TL)L^T$$
	Опуская вычисления, получим:
	$$L^+=\begin{pmatrix}
		0.6875   &  -0.0625  &   -0.3125    & -0.3125\\
		-0.0625  &    0.1875  &   -0.0625   &  -0.0625 \\
		-0.3125   &  -0.0625   &   0.35416667 & 0.02083333 \\
		-0.3125   &  -0.0625    &  0.02083333 & 0.35416667
	\end{pmatrix}$$
	Найдём вектор напряжений v:
	$$v = L^+b = \begin{pmatrix}
		0.6875   &  -0.0625  &   -0.3125    & -0.3125\\
		-0.0625  &    0.1875  &   -0.0625   &  -0.0625 \\
		-0.3125   &  -0.0625   &   0.35416667 & 0.02083333 \\
		-0.3125   &  -0.0625    &  0.02083333 & 0.35416667
	\end{pmatrix} \begin{pmatrix}
	1\\0\\0\\-1
	\end{pmatrix}$$
	
	$v = (1 \  0 \ -1/3 \ -2/3)^T$
	Найдём разность потенциалов первого и четвертого узла:
	$$1+\frac{2}{3} =  \frac{5}{3}$$
	
	Проверка прямым расчётом:\newline
	От узла 1 к узлу 2:  1\newline
	От узла 2 к 4: $\left(\frac{1}{1} +  \frac{1}{1+1}\right)^{-1} = \frac{2}{3}$\newline
	Итог, от 1 к 4: $1+\frac{2}{3} = \frac{5}{3}$
	 \section{Пример 2}
	 Нам дана следующая матрица смежности:
	 $$M=\begin{pmatrix}
	 	0& 1& 1 &1\\
	 	1& 0& 1 &1\\
	 	1& 1& 0 &1\\
	 	1& 1& 1 &0
	 \end{pmatrix}$$
	 
	 \begin{figure}[h]
	 	\centering
	 	\begin{tikzpicture}[main/.style = {draw, circle}] 
	 		\node[main] (1) {$1$}; 
	 		\node[main] (2) [above of=1] {$2$};
	 		\node[main] (3) [below right of=1] {$3$}; 
	 		\node[main] (4) [below left of=1] {$4$};
	 		
	 		\draw (2) -- (4);
	 		\draw (3) -- (4);
	 		\draw (1) -- (4);
	 		\draw (1) -- (3);
	 		\draw (2) -- (3);
	 		\draw (2) -- (1);
	 	\end{tikzpicture}
	 	\caption{Пример графа с четырьмя вершинами, построенный по матрице смежности М}
	 \end{figure}
	 \newpage
	  Найдём матрицу степеней вершин:
	  $$G =\begin{pmatrix}
	  	3& 0& 0& 0&\\
	  	0& 3& 0& 0&\\
	  	0& 0& 3& 0&\\
	  	0& 0& 0& 3&\\
	  	
	  \end{pmatrix}$$
	  Вычтем из матрицы степеней матрицу смежности:
	  $$G-M = L = \begin{pmatrix}
	  	3& -1& -1& -1&\\
	  	-1& 3& -1& -1&\\
	  	-1& -1& 3& -1&\\
	  	-1& -1& -1& 3&\\
	  	
	  \end{pmatrix}$$
	  Полученную матрицу можно интепретировать, как коэффициенты системы уравнений Киргхофа для неориентированного графа:
	  $$Lv=b$$
	  Теперь, по описанному выше алгоритму получаем псевдообратную матрицу:
	  $$L^+ = (L^TL)L^T$$
	  Опуская вычисления, получим:
	  $$L^+=\begin{pmatrix}
	  	0.1875 & -0.0625 & -0.0625 & -0.0625\\
	  	-0.0625 & 0.1875 & -0.0625 & -0.0625\\
	  	-0.0625 & -0.0625 & 0.1875 & -0.0625\\
	  	-0.0625 & -0.0625 & -0.0625 & 0.1875\\
	  	
	  \end{pmatrix}$$
	  Найдём вектор напряжений v:
	  $$v = L^+ b = \begin{pmatrix}
	  	0.1875 & -0.0625 & -0.0625 & -0.0625\\
	  	-0.0625 & 0.1875 & -0.0625 & -0.0625\\
	  	-0.0625 & -0.0625 & 0.1875 & -0.0625\\
	  	-0.0625 & -0.0625 & -0.0625 & 0.1875\\
	  	
	  \end{pmatrix} \begin{pmatrix}
	  1\\
	  0\\
	  0\\
	  -1
	  \end{pmatrix}$$
	  $$v = (0.25\ 0\ 0\ -0.25)^T$$
	  Найдём разность потенциалов в первом узле и в четвёртом: $$0.25 - (-0.25) = 0.5$$
	  Ответ: 0.5
	  
	 
	 \section{Код на python (Позже перенесу на Octave)}
	 
	 
	 \begin{verbatim}
	 	import numpy as np
	 	import networkx as nx
	 	import numpy as np
	 	import matplotlib.pyplot as plt
	 	a, b = 1, 4 # номера вершин, между которыми будем искать сопротивление.
	 	
	 	
	 	#Матриица смежности заданного графа
	 	P0= [[0, 1, 1, 1],
	 	[1, 0, 1, 1],
	 	[1, 1, 0, 1],
	 	[1, 1, 1, 0,],]
	 	#Рисуем граф
	 	G = nx.DiGraph(np.matrix(P0))
	 	nx.draw(G, with_labels=True, node_size=300, arrows=False)
	 	plt.show()
	 	#Добавляем степени вершин
	 	P1 = np.array(P0)
	 	if P1[0][0] == 0:
	 	P1 = np.array(P0) * -1
	 	for i, row in enumerate(P0):
	 	d = sum(row)
	 	P1[i][i] =d
	 	
	 	L = P1
	 	print(L)
	 	#Получили матрицу Киргофа (Матричный лапласиан)
	 	# Находим элементы псевдообратной матрицы
	 	L_plus = np.linalg.pinv(L)
	 	print(L_plus)
	 	h = np.array([0]*L_plus[0].size)
	 	h[a-1] = 1
	 	h[b-1] = -1
	 	
	 	# Находим резистивное расстояние между узлами 1 и 4
	 	answer_v = np.dot(L_plus, h)
	 	answer = answer_v[a-1] - answer_v[b-1] 
	 	
	 	print(f"Резистивное расстояние между узлами {a} и {b}:", answer)
	 \end{verbatim}
	 \bibliographystyle{plain}
	 
	\bibliography{biba}
\end{document}