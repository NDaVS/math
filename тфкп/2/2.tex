\documentclass{article}
\usepackage{amsmath}
\usepackage{amsthm}
\usepackage[T2A]{fontenc} % Кодировка шрифта
\usepackage[utf8]{inputenc} % Кодировка ввода
\usepackage[english,russian]{babel} % Языковые настройки
% Определение стилей для теорем, лемм и замечаний
\newtheorem{theorem}{Теорема}[section]
\newtheorem{lemma}[theorem]{Лемма}
\newtheorem{corollary}[theorem]{Следствие}
\newtheorem{proposition}[theorem]{Утверждение}

\theoremstyle{definition}
\newtheorem{definition}[theorem]{Определение}
\newtheorem{example}[theorem]{Пример}

\theoremstyle{remark}
\newtheorem{remark}[theorem]{Замечание}

\begin{document}
	\tableofcontents
	
	\section{Примеры теорем и лемм}
	
	\begin{theorem}
		\label{theorem:example}
		Это пример теоремы. Текст теоремы может содержать математические формулы, например, \( E = mc^2 \).
	\end{theorem}
	
	\begin{proof}
		Это пример доказательства теоремы \ref{theorem:example}. Доказательства обычно содержат последовательные логические шаги.
	\end{proof}
	
	\begin{lemma}
		Это пример леммы. Леммы часто используются для доказательства теорем.
	\end{lemma}
	
	\begin{corollary}
		Это пример следствия, которое вытекает из теоремы или леммы.
	\end{corollary}
	
	\begin{proposition}
		Это пример утверждения, которое можно доказать аналогично теореме или лемме.
	\end{proposition}
	
	\begin{definition}
		Это пример определения. Определения нужны для уточнения понятий и терминов.
	\end{definition}
	
	\begin{example}
		Это пример примера. Примеры помогают иллюстрировать определения и теоремы.
	\end{example}
	
	\begin{remark}
		Это пример замечания. Замечания содержат дополнительные комментарии или наблюдения.
	\end{remark}
	
	$\mathbf{x}$
	
\end{document}
