\documentclass{article}

\usepackage[T2A]{fontenc} % Кодировка шрифта
\usepackage[utf8]{inputenc} % Кодировка ввода
\usepackage[english,russian]{babel} % Языковые настройки
\usepackage{graphicx} % Для вставки изображений
\usepackage{amsmath} % Для использования математических формул
\usepackage{amssymb}
\usepackage{cancel}
\usepackage{amsfonts} % Для использования математических символов и шрифтов
\usepackage{titlesec} % Для настройки заголовков разделов
\usepackage{titling} % Для настройки титульной страницы
\usepackage{geometry} % Для настройки размеров страницы
\usepackage{pgfplots}
\pgfplotsset{compat=1.9}

% Настройка заголовков разделов
\titleformat{\section}
{\normalfont\Large\bfseries}{\arabic{section}}{1em}{}
\titleformat{\subsection}
{\normalfont\large\bfseries}{}{1em}{}

% Настройка титульной страницы
\setlength{\droptitle}{-3em} % Отступ заголовка
\title{\vspace{-1cm}ИДЗ №1}
\author{Вершинин Данил Алексеевич}
\date{\today}

% Настройка размеров страницы
\geometry{a4paper, margin=2cm}

\begin{document}
	
	% Автоматическая генерация оглавления (см. далее)
	\maketitle
	%\tableofcontents
	%\chapter{Задачи}
	\section{Вычислить интеграл $\oint\limits_{|z-1|=3} \frac{ze^z}{\sin z}dz$}
	\subsection{Решение:}
	\[\oint\limits_{|z-1|=3} \frac{ze^z}{\sin z}dz - \text{Отношение голоморфных функций}\]
	\text{Найдём особые точки:}
	\[\sin z = 0 \Rightarrow z = \pi n, n \in \mathbb{Z}\]
	\text{В круге $|z-1|=3 $ - две особые точки: $z=0, z = \pi$}
	\[(ze^z)' = e^z + ze^z \bigg|_{z=0} = 1 \ne 0\]
	\[(\sin z)'\ = \cos z \bigg|_{z=0} = 1 \ne 0\]
	\text{Следовательно, 0 - нуль первого порядка, следовательно УОТ, следователь вычет в ней равен 0}
	Точка $\pi$ - нуль первого порядка знаменателя, числитель в ней не обращается в 0.
	\[ze^z \bigg|_{z=\pi} = \pi e^ pi \ne 0; \ (\sin z )' = \cos z \bigg|_{z=\pi} = -1 \ne 0\]
	Следовательно, $z=\pi$ - полюс первого порядка, вычет вычисляется по формуле:
	\[\underset{z=\pi}{\text{Res}} \frac{ze^z}{\sin z } = \frac{zez^z}{\cos z}\bigg|_{z=\pi} = -\pi e^\pi\]
	Поэтому интеграл равен:
	\[\oint\limits_{|z-1|=3} \frac{ze^z}{\sin z}dz = 2\pi i \cdot (-\pi e^\pi) =  -2\pi^2 e^\pi i\]
	
	\subsection{Ответ:}
	$ -2\pi^2 e^\pi i$
	
	\section{Вычислить интеграл $\oint\limits_{|z|=1}\frac{e^{2z}-z}{z^2} dz$}
	
	\subsection{Решение:}
	\[\oint\limits_{|z|=1}\frac{e^{2z}-z}{z^2} dz\]
	Заметим, что \[\oint\limits_{|z|=1}\frac{e^{2z}-z}{z^2} dz = \oint\limits_{|z|=1}\left(\frac{e^{2z}}{z^2} - \frac{1}{z}\right) dz = \oint\limits_{|z|=1}\frac{e^{2z}}{z^2}dz - \oint\limits_{|z|=1}\frac{1}{z}dz\]
	\[\underset{z=0}{\text{Res}} \frac{1}{z} = 1\]
	\[\lim\limits_{z\rightarrow0} \frac{z^2 e^{2z}}{z^2} = 1 \text{ - видим, что 0 - полюс второго порядка}\]
	Разложим $\frac{e^{2z}}{z^2}$ в ряд Лорана в окрестности точки 0:
	\[\frac{1}{z^2}\left(1 + 2z + \frac{4z^2}{2!} + \dots\right) = \frac{1}{z^2} + \frac{2}{z} + \frac{4}{2!} + \dots\]
	Отсюда видим, что вычет равен 2. Следовательно интеграл равен:
	\[\oint\limits_{|z|=1}\frac{e^{2z}-z}{z^2} dz = 2\pi i (2 - 1) = 2\pi i\]
	\subsection{Ответ: $2\pi i$}
	
	\section{Вычислить интеграл $\oint\limits_{|z|=0,05} \frac{e^{iz} - 1 -\sin 4z}{z^3 \sh 16 \pi z}dz$}
	\subsection{Решение:}
	\[\oint\limits_{|z|=0,05} \frac{e^{iz} - 1 -\sin 4z}{z^3 \sh 16 \pi z}dz\]
	Числитель и знаменатель голоморфные функции всюду в$\mathbb{C} \Rightarrow$ особые точки - это нули знаменателя, при условии $|z| < 0,05$
	\[z^3\sh16\pi z \iff z^3 = 0 \ \lor \ \sh 16 \pi z = 0 \]
	Последнее перепишем:
	\[\sh 16 \pi z = 0 \Rightarrow e^{16\pi z} - e^{-16\pi z} = 0 \Rightarrow e^{16\pi z} = e^{-16\pi z} \Rightarrow e^{32\pi z} = 1 \Rightarrow 32\pi z = 2\pi n, n \in \mathbb{Z}\]
	Следующая оценка покажет, что в $|z| < 0,05$ попадает тольо одна точка ($z=0$)
	\[n \ne 0 \Rightarrow \left|\frac{\pi}{16}n\right| \ge \left|\frac{\pi}{16}\right| > \frac{3}{16} > \frac{1}{20}\]
	Итак, вычисления показывают, что:
	\[\oint\limits_{|z|=0,05} \frac{e^{iz} - 1 -\sin 4z}{z^3 \sh 16 \pi z}dz = 2\pi i \underset{z=0}{\text{Res}}\frac{e^{iz} - 1 -\sin 4z}{z^3 \sh 16 \pi z}\]
	Заметим, что $z=0$ - является нулём первого порядка для числителя:
	\[e^{iz}-1\sin4z \big|_{z=0} = 0; \ (e^{iz} -1 -\sin 4z)'\big|_{z=0} = (ize^{iz} -4\cos 4z)\big|_{z=0} = i -4 \ne 0\]
	и нуль четвертого порядка для знаменателя, т.к. он является $z^3$ (3-го порядка) и $\sh16\pi z$ (первый порядок); $(\sh16\pi z)'\big|_{z=0} \ne 0$\newline
	Поэтому, для дроби $z=0$ - полюс 3-го порядка. \newline Вычет равен:
	\[\underset{z=0}{Res}\frac{e^{iz} - 1 -\sin 4z}{z^3 \sh 16 \pi z} = \frac{1}{2}\left(\frac{e^{iz} - 1 -\sin 4z}{z^3 \sh 16 \pi z}\right)^{(2)}\]
	Брать вторую производную достаточно затратное занятие. Найдём вычет через ряд Лорана
	$z=0$ - полюс 3-го порядка, следовательно разложение будет иметь вид:
	\[f(z) = \frac{C_{-3}}{z^3} + \frac{C_{-2}}{z^2} + \frac{C_{-1}}{z} + C_0 + \dots \big| \cdot \sh16\pi z\]
	\[e^{iz} -1 -\sin4z = \sh 16 \pi z (C_{-3} + C_{-2}z + C_{-1}z^2 + C_{0}z^3 + \dots)\]
	Разложим левую часть:
	\[e^{iz} -1 -\sin4z = iz + \frac{(iz)^2}{2!} + \frac{(iz)^3}{3!} + \dots - (4z - \frac{(4z)^3}{3!}) = (i-4)z + \frac{(iz)^2}{2} + \frac{4^3 - i}{6}z^3 + \dots\]
	Справа получили разложение:
	\[\sh16\pi z (C_{-3} +\dots) = (16\pi z + \frac{(16\pi z)^3}{6} + \dots)(C_{-3} + C_{-2}z + C_{-1}z^2) + C_{0}z^3 + o(z^4) = \]
	\[=16\pi z C_{-3} + 16 \pi C_{-2}z^2 + \left[\frac{(16\pi)^3}{6}C_{-3} + 16\pi C_{-1}\right]z^3 + \dots\]
	Приравниваем коэффициенты:
	\[\begin{Bmatrix}
		i-4 = 16\pi C_{-3} & \Rightarrow &C_{-3} =  \frac{i-4}{16\pi} \\
		\frac{1}{2} = 16\pi C_{-2} & \Rightarrow & C_{-2} = \frac{1}{32\pi}\\
		\frac{4^3-i}{6} = \frac{(16\pi)^3}{6}C_{-3} + 16\pi C_{-1} & \Rightarrow & C_{-1} = \left(\frac{4^3 - i}{6} - \frac{(16\pi)^3}{6}C_{-3}\right)\frac{1}{16\pi}
		
	\end{Bmatrix}\]
	Таким образом:
	\[\oint\limits_{|z|=0,05} \frac{e^{iz} - 1 -\sin 4z}{z^3 \sh 16 \pi z}dz = 2\pi i \cdot \frac{1}{16\pi}\left(\frac{4^3-i}{6} - \frac{(16\pi)^3}{6}\cdot\frac{(i-4)}{16\pi}\right) = \frac{i}{8\pi}\left(\frac{4-i - (16\pi)^2(i-4)}{6}\right)\]
	\subsection{Ответ:$\frac{i}{8\pi}\left(\frac{4-i - (16\pi)^2(i-4)}{6}\right)$}
	
	
	\section{Вычислить интеграл $\oint\limits_{|z+3|=2}\left(z \sh \frac{i}{z+3} - \frac{4 \sh \frac{\pi i z}{4}}{(z+2)^2z}\right) dz$}
	\subsection{Решение:}
	Интеграл суммы равен сумме интегралов, поэтому:
	\[\oint\limits_{|z+3|=2}\left(z \sh \frac{i}{z+3} - \frac{4 \sh \frac{\pi i z}{4}}{(z+2)^2z}\right) dz = \oint\limits_{|z+3|=2}z \sh \frac{i}{z+3}dz - \oint\limits_{|z+3|=2} \frac{4 \sh \frac{\pi i z}{4}}{(z+2)^2z} dz\]
	Рассмотрим первый интеграл. Очевидно, что подынтегральное выражение имеет в $\mathbb{C}$ только одну изолированную особую точку:  $z=-3$ -  центр окружности, по которой интегрируем:
	\[\oint\limits_{|z+3|=2}z \sh \frac{i}{z+3}dz\]
	Применяя основную теорему о вычетах, получим:
	\[\oint\limits_{|z+3|=2}z \sh \frac{i}{z+3}dz = 2\pi i \underset{z=-3}{\text{Res}}z\sh\frac{i}{z+3}\]
	Разложим подынтегральную функцию в ряд лорана в окрестности точки $z=-3$:
	\[z\sh\frac{i}{z+3} = (-3 + (z+3))\left(\frac{i}{z+3} + \frac{i^3}{3!(z+3)^3} + \frac{i^5}{5!(z+3)^5} +\dots\right) = i - \frac{3i}{z+3} - \frac{i}{3!(z+3)^2} + \frac{3i}{3!(z+3)^3} + \dots\]
	Отсюда видим, что разложение содержит бесконечно много ненулевых коэффициентов при отрицательных степенях $(z+3)$, значит $z=-3$ - СОТ. \newline 
	Кроме того, вычет в этой точке 
	\[\underset{z=-3}{\text{Res}}z\sh\frac{i}{z+3} = -3i\]
	Значит,
	\[\oint\limits_{|z+3|=2}z \sh \frac{i}{z+3}dz = 6\pi\]
	Теперь займёмся вторым интегралом:
	\[\oint\limits_{|z+3|=2} \frac{4 \sh \frac{\pi i z}{4}}{(z+2)^2z} dz\]
	Особыми точками подынтегральной функции являются нули знаменателя $z=0$ и $z=-2$. Но точка $z=0$ лежит вне окружности $|z+3|=2$, поэтому по основной теореме Коши о вычетах имеем:
	\[\oint\limits_{|z+3|=2} \frac{4 \sh \frac{\pi i z}{4}}{(z+2)^2z} dz = 2\pi i \underset{z=-2}{\text{Res}}\frac{4 \sh \frac{\pi i z}{4}}{(z+2)^2z} \]
	Но $z=-2$ полюс второго порядка, так как является, очевидно, нулем второго порядка для знаменателя, а числитель в ней $\ne 0$. По
	формуле для вычисления вычета в полюсе порядка 2
	\[\underset{z=-2}{\text{Res}}\frac{4 \sh \frac{\pi i z}{4}}{(z+2)^2z} = \lim\limits_{z\rightarrow -2} \left(\frac{4 \sh \frac{\pi i z}{4}}{z}\right)' = 4\lim\limits_{z\rightarrow-2}\frac{\frac{\pi i}{4}\ch\frac{\pi i z}{4}z - \sh\frac{\pi i z}{4}}{z^2} = \frac{\pi i }{4}\ch\left(-\frac{\pi i}{2}\right) - \sh \left(-\frac{\pi i}{2}\right) =\]
	\[=\frac{\pi i }{4}\cos\left(-\frac{\pi}{2}\right) - i\sin \left(-\frac{\pi}{2}\right) = i\]
	Поэтому
	\[\oint\limits_{|z+3|=2} \frac{4 \sh \frac{\pi i z}{4}}{(z+2)^2z} dz = 2\pi i \underset{z=-2}{\text{Res}}\frac{4 \sh \frac{\pi i z}{4}}{(z+2)^2z} = -2\pi\]
	\subsection{Ответ:}
	\[\oint\limits_{|z+3|=2}\left(z \sh \frac{i}{z+3} - \frac{4 \sh \frac{\pi i z}{4}}{(z+2)^2z}\right) dz = 8\pi\]
	
	\section{Вычислить интеграл $\int\limits_{0}^{2\pi} \frac{dt}{2\sqrt{6}\sin t - 5}$}
	\subsection{Решение:}
	Проведём замену $e^{it} = z\Rightarrow dt = dz/iz$. Будут справедливы формулы:
	
	\[\sin t = \frac{e^{iz} - e^{-iz}}{2i} = \frac{1}{2i}\left(z - \frac{1}{z}\right)\]
	\[\cos t = \frac{e^{iz} + e^{-iz}}{2} = \frac{1}{2}\left(z + \frac{1}{z}\right)\]
	После замены получаем
	\[\int\limits_{0}^{2\pi} \frac{dt}{2\sqrt{6}\sin t - 5} = \oint \limits_{|z|=1} \frac{dz}{\left(\cancel{2} \sqrt{6} \frac{1}{\cancel{2}i}\left(z - \frac{1}{z}\right) -5\right)iz} = \oint \limits_{|z|=1} \frac{dz}{\sqrt{6}z \left(z - \frac{1}{z}\right) -5iz} = \oint \limits_{|z|=1} \frac{dz}{\sqrt{6}z^2 - \sqrt{6} -5iz}\]
	очевидно, что нули наменателя являются полюсами первого порядка для подынтегрального выражения. Найдём их:
	\[\sqrt{6}z^2 - \sqrt{6} -5iz =0 \Rightarrow z_{1,2} = \frac{5i \pm \sqrt{-25 + 24}}{2\sqrt{6}} = \begin{cases}
		\frac{i\sqrt{6}}{2} \\
		\frac{i\sqrt{6}}{3}
	\end{cases}\]
	В круге $|z|=1$ лежит только корень $\frac{i\sqrt{6}}{3}$. Поэтому, искомый интеграл вычисялется так:
	\[\oint \limits_{|z|=1} \frac{dz}{\sqrt{6}z^2 - \sqrt{6} -5iz} = 2\pi i \underset{z=\frac{i\sqrt{6}}{3}}{\text{Res}} \frac{1}{\sqrt{6}z^2 - \sqrt{6} -5iz} = 2\pi i \underset{z=\frac{i\sqrt{6}}{3}}{\text{Res}} \frac{1}{\sqrt{6}\left(z - \frac{i\sqrt{6}}{2}\right)\left(z - \frac{i\sqrt{6}}{3}\right)} = \frac{2\pi i}{\sqrt{6}\left(z - \frac{i\sqrt{6}}{2}\right)}\Big|_{z = \frac{i\sqrt{6}}{3}}  =\]
	\[= \frac{2 \pi i }{\sqrt{6}\left(\frac{i\sqrt{6}}{3} - \frac{i\sqrt{6}}{2}\right)} = -2\pi\]
	\subsection{Ответ:}
	\[\int\limits_{0}^{2\pi} \frac{dt}{2\sqrt{6}\sin t - 5} = -2\pi\]
	
	
	\section{Вычислить интеграл $\int\limits_{0}^{2\pi} \frac{dt}{\left(\sqrt{7} + \sqrt{5}\cos t\right)^2}dt$}
	\subsection{Решение:}
	Произведём замену $e^{it} = z \Rightarrow dt = dz/iz $
	\[\cos t = \frac{1}{2}\left(z + \frac{1}{z}\right)\]
	\[\int\limits_{0}^{2\pi} \frac{dt}{\left(\sqrt{7} + \sqrt{5}\cos t\right)^2}dt = \oint\limits_{|z|=1}\frac{dz}{\left(\sqrt{7} + \frac{\sqrt{5}}{2}\left(z + \frac{1}{z}\right)\right)^2iz} = \frac{4}{i}\oint\limits_{|z|=1}\frac{zdz}{\left(\sqrt{5}z^2 + 2\sqrt{7}z + \sqrt{5}\right)^2}\]
	Последний интеграл вычисляем по основной теореме Коши о вычетах. Для этого найдём нули знаменателя:
	\[\sqrt{5}z^2 + 2\sqrt{7}z + \sqrt{5} = 0 \Rightarrow z_{1,2} = \frac{-2\sqrt{7} \pm \sqrt{28 - 20}}{2\sqrt{5}} = \begin{cases}
		\frac{-\sqrt{7} -\sqrt{2}}{\sqrt{5}}\\
		\frac{-\sqrt{7} +\sqrt{2}}{\sqrt{5}}
	\end{cases}\]
	Первый из корней лежит вне круга $|z|<1$, поэтому:
	\[\frac{4}{i}\oint\limits_{|z|=1}\frac{zdz}{\left(\sqrt{5}z^2 + 2\sqrt{7}z + \sqrt{5}\right)^2} = \frac{4}{i}\oint\limits_{|z|=1} \frac{zdz}{5\left(z+\frac{\sqrt{7}+\sqrt{2}}{\sqrt{5}}\right)^2\left(z+\frac{\sqrt{7}-\sqrt{2}}{\sqrt{5}}\right)^2} = \frac{8\pi}{5}\underset{z=-\frac{\sqrt{7} - \sqrt{2}}{\sqrt{5}}}{\text{Res}} \frac{z}{\left(z+\frac{\sqrt{7}+\sqrt{2}}{\sqrt{5}}\right)^2\left(z+\frac{\sqrt{7}-\sqrt{2}}{\sqrt{5}}\right)^2}\]
	Очевидно, для выражение под знаком вычета точка $z =-\frac{\sqrt{7} - \sqrt{2}}{\sqrt{5}}$ является полюсом второго порядка, поэтому вычет вычисляется по формуле для полюса порядка n (n = 2)
	\[\underset{z=-\frac{\sqrt{7} - \sqrt{2}}{\sqrt{5}}}{\text{Res}} \frac{z}{\left(z+\frac{\sqrt{7}+\sqrt{2}}{\sqrt{5}}\right)^2\left(z+\frac{\sqrt{7}-\sqrt{2}}{\sqrt{5}}\right)^2} = \lim\limits_{z \rightarrow-\frac{\sqrt{7} - \sqrt{2}}{\sqrt{5}}} \left(\frac{z}{\left(z+\frac{\sqrt{7}+\sqrt{2}}{\sqrt{5}}\right)^2}\right)' = \lim\limits_{z \rightarrow-\frac{\sqrt{7} - \sqrt{2}}{\sqrt{5}}} \frac{-5\sqrt{5}x + 5\sqrt{7} + 5\sqrt{2}}{(\sqrt{5}x + \sqrt{7} + \sqrt{2})^3} = \]
	\[=\frac{5\sqrt{14}}{16}\]
	Окончательно получаем
	\subsection{Ответ:}
	\[\int\limits_{0}^{2\pi} \frac{dt}{\left(\sqrt{7} + \sqrt{5}\cos t\right)^2}dt = \frac{\sqrt{14}}{2}\pi\]
	
	\section{Вычислить интеграл $\int\limits_{-\infty}^{+\infty} \frac{x^2+1}{\left(x^2 + x + 1\right)^2}dx$}
	\subsection{Решение:}
	\subsection{Ответ:}
	
	\section{Вычислить интеграл $\int\limits_{-\infty}^{+\infty} \frac{\sin 2x}{\left(x^2 - 2x + 10\right)}dx$}
	\subsection{Решение:}
	\subsection{Ответ:}
	
	
	\section{Найти оригинал по заданному изображению $\frac{5p}{(p+2)(p^2-2p+2)}$}
	\subsection{Решение:}
	\subsection{Ответ:}
	\newpage
	\section{Найти решение дифференциального уравнения, удовлетворяющее условию $y'' + 2' = \frac{1}{\ch^2t}$}
	\subsection{Решение:}
	\subsection{Ответ:}
	
	\section{Операционным методом решить задачу Коши $2y'' + 3y' + y = 3e^t, \ y(0) = 0, \ y'(0)=1$}
	\subsection{Решение:}
	\subsection{Ответ:}
	
	\section{Найти решение системы дифференциальных уравнений, удовлетворяющее заданному начальному условию $\begin{cases}
			x'=y+3,& x(0) =1 \\
			y'=x+2,& y(0)=0
		\end{cases}$}
	\subsection{Решение:}
	\subsection{Ответ:}
\end{document}