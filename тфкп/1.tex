\documentclass{article}

\usepackage[T2A]{fontenc} % Кодировка шрифта
\usepackage[utf8]{inputenc} % Кодировка ввода
\usepackage[english,russian]{babel} % Языковые настройки
\usepackage{graphicx} % Для вставки изображений
\usepackage{amsmath} % Для использования математических формул
\usepackage{amssymb}
\usepackage{amsfonts} % Для использования математических символов и шрифтов
\usepackage{titlesec} % Для настройки заголовков разделов
\usepackage{titling} % Для настройки титульной страницы
\usepackage{geometry} % Для настройки размеров страницы
\usepackage{pgfplots}
\pgfplotsset{compat=1.9}

% Настройка заголовков разделов
\titleformat{\section}
{\normalfont\Large\bfseries}{\arabic{section}}{1em}{}
\titleformat{\subsection}
{\normalfont\large\bfseries}{}{1em}{}

% Настройка титульной страницы
\setlength{\droptitle}{-3em} % Отступ заголовка
\title{\vspace{-1cm}ИДЗ №1}
\author{Вершинин Данил Алексеевич}
\date{\today}

% Настройка размеров страницы
\geometry{a4paper, margin=2cm}

\begin{document}
	
	% Автоматическая генерация оглавления (см. далее)
	\maketitle
	%\tableofcontents
	%\chapter{Задачи}
	\section{Найти все значения корня: $\sqrt[3]{8i}$}
	\subsection{Решение:}
	\[w= 8i\]
	Найду модуль и аргумент:
	\[\rho = |8i| = \sqrt{64} = 8 \\ \]
	\[\phi = \arctg\left(\frac{8}{0}\right) = \arctg(+\infty) = \frac{\pi}{2}\]
	\[2 = 8(\cos\left(\frac{\pi}{2}\right) + i \sin\left(\frac{\pi}{2}\right))\]
	\[z_k = 2 \left(\cos\left(\frac{\pi }{2\cdot 3} + \frac{2\pi k}{3}\right) + i \sin\left(\frac{\pi }{2\cdot 3} + \frac{2\pi k}{3}\right)\right) = 2 \left(\cos\left(\frac{\pi }{6} + \frac{2\pi k}{3}\right) + i \sin\left(\frac{\pi }{6} + \frac{2\pi k}{3}\right)\right)\]
	\[z_0 = 2 \left(\cos\left(\frac{\pi}{6}\right) + i \sin\left(\frac{\pi}{6}\right)\right) = \sqrt{3} + i\]
	\[z_1 = 2 \left(\cos\left(\frac{\pi}{6} + \frac{2\pi}{3}\right) + i \sin\left(\frac{\pi}{6} + \frac{2\pi}{3}\right)\right) = 2 \left(\cos\left(\frac{5\pi}{6}\right) + i \sin\left(\frac{5\pi}{6}\right)\right) = - \sqrt{3} + i\]
	\[z_2 = 2 \left(\cos\left(\frac{\pi}{6} + \frac{4\pi}{3}\right) + i \sin\left(\frac{\pi}{6} + \frac{4\pi}{3}\right)\right) = 2 \left(\cos\left(\frac{3\pi}{2}\right) + i \sin\left(\frac{3\pi}{2}\right)\right) = 2i\]
	\subsection{Ответ:}
	\[z_0 = \sqrt{3} + i\]
	\[z_1 = - \sqrt{3} = i\]
	\[z_3 = 2i\]
	
	\section{Представить в алгебраической форме: $\sh(2-\pi i)$}
	\subsection{Решение:}
	\[\sh(2-\pi) = \sh(2)\ch(\pi i) - \ch(2)\sh(\pi i) = -\sh(2)\]
	\[\ch(\pi i) = \cos(\pi) = -1\]
	\[\sh(i\pi) = i \sin(\pi) = 0\]
	\subsection{Ответ:}
	\[-\sh(2)\]
	
	\section{Представить в алгебраической форме: $\text{arcth}(1+ \sqrt{3}i)$}
	\subsection{Решение:}
	\[\text{arcth}(1+ \sqrt{3}i)\]
	\[\text{arcth}(z) = w \Rightarrow \cth(w) = z\]
	\[\frac{e^w + e^{-w}}{e^w - e^{-w}} = z\]
	\[\frac{e^w + 1/e^w}{e^w - 1/e^w} = \frac{e^{2w}+1}{e^{2w}-1} = z \Rightarrow e^{2w} + 1 = z \left(e^{2w} -1\right)\]
	\[e^{2w}(1-z) +z +1 = 0\]
	\[e^{2w}(1-z) + (z+1) = 0 \Rightarrow e^{2w} = \frac{z+1}{z-1}\]
	\[2w = \text{Ln}\left(\frac{z+1}{z-1}\right) = \ln \left|\frac{z+1}{z-1}\right| + i\text{Arg}\left(\frac{z+1}{z-1}\right) \]
	\[w=  \frac{1}{2} \left(\ln \left|\frac{z+1}{z-1}\right| + i\left(\text{arg}\left(\frac{z+1}{z-1}\right) +2\pi n\right)\right), \ n \in \mathbb{Z}\]
	\[\frac{z+1}{z-1} = \frac{2 + \sqrt{3}i}{\sqrt{3}i}\cdot \frac{-\sqrt{3}i}{-\sqrt{3}i} = \frac{3 - 2\sqrt{3}i}{3} = 1 - \frac{2\sqrt{3}i}{3}\]
	\[\left|\frac{z+1}{z-1} \right| = \left|1 - \frac{2\sqrt{3}i}{3}\right| = \sqrt{1 + \frac{4}{3}} = \sqrt{\frac{7}{3}}\]
	\[\text{arg}\left(1 -\frac{2\sqrt{3}i}{3}\right) = \arctg\left(-\frac{2}{\sqrt{3}}\right) = -\arctg\left(\frac{2}{\sqrt{3}}\right)\]
	\[w = \frac{1}{2} \left(\ln\sqrt{\frac{7}{3}} + i \left(-\arctg\left(\frac{2}{\sqrt{3}}\right) + 2\pi n\right)\right), \ n \in \mathbb{Z}\]
	\subsection{Ответ:}
	\[\frac{1}{2} \left(\ln\sqrt{\frac{7}{3}} + i \left(-\arctg\left(\frac{2}{\sqrt{3}}\right) + 2\pi n\right)\right), \ n \in \mathbb{Z}\]
	
	\section{Представить в алгебраической форме: $(-3i)^i$}
	\subsection{Решение:}
	\[(-3i)^i = e^{i\text{Ln}(-3i)}\]
	\[\text{Ln}(-3i) = \ln |-3i| + i\text{Arg}(-3i) = \ln(3) + i\left(\frac{\pi}{2} + 2\pi k\right), k \in \mathbb{Z}\]
	\[e^{i\text{Ln}(-3i)} = e ^ {-\left(\frac{\pi}{2} + 2\pi k\right) + i\ln(3)} = e ^ {-\left(\frac{\pi}{2} + 2\pi k\right)}\left(\cos\left(\ln(3)\right) + i\sin\left(\ln(3)\right)\right), k \in \mathbb{Z}\]
	\subsection{Ответ:}
	\[e ^ {-\left(\frac{\pi}{2} + 2\pi k\right)}\left(\cos\left(\ln(3)\right) + i\sin\left(\ln(3)\right)\right), k \in \mathbb{Z}\]
	
	\section{Представить в алгебраической форме: $\text{Ln}(2+2\sqrt{3}i)$}
	\subsection{Решение:}
	\[\text{Ln}(2+2\sqrt{3}i) = \ln|2 + 2\sqrt{3}i| + i\text{Arg}(2+2\sqrt{3}i) = \ln 4 + i\left(\frac{\pi}{3} + 2\pi k\right), k \in \mathbb{Z}\]
	\subsection{Ответ:}
	\[\ln 4 + i\left(\frac{\pi}{3} + 2\pi k\right), k \in \mathbb{Z}\]
	
	\section{Вычертить область, заданную неравествами: $\\D = \{z : |z - 1 + i| \ge 1, Re(z) < 1, Im(z) \le 1 \}$}
	\subsection{Ответ:}
	\begin{tikzpicture}
	
		\filldraw[draw=cyan, line width=1pt, fill=cyan!20] (-3, 0) -- (1,0) arc (90:270:1)--(1,-3)--(-3,-3)--(-3,0);
		\draw[blue!40] (1,-1) circle (1); % заливка круга с центром в (1,1) и радиусом 1
		\draw[thick, ->] (-3,0) -- (3,0) node[below] {$\Re(z)$};
		\draw[thick, ->] (0,-3) -- (0,2) node[left] {$\Im(z)$};
		\draw[thick] (1,-1)--(2, -1) node[midway,below]{1} ;
		\draw (-3, 1) -- (3,1);
		
		\draw[dashed] (1,2) -- (1,0); % линия Im(z) = 1
		\draw[dashed] (1,-2) -- (1, -3);
		
	\end{tikzpicture}
	
	
	
	\section{Определить вид пути и в случае, когда он проходит через точку $\infty$, исследовать его поведение в этой точке: $z = \frac{4}{\ch(4t)} + i2\th(4t)$}
	\subsection{Решение:}
	\subsection{Ответ:}
	
	\section{Восстановить голоморфную в окрестности точки $z_0$ функцию $f(z)$ по известной действительной части $u(x,y)$ или мнимой $v(x,y)$ и начальному значению $f(z_0)$: $u = x^2 - y^2 + x, \ f(0) = 0$}
	\subsection{Решение:}
	\[u = x^2 - y^2 + x\]
	\[\frac{\delta u}{\delta x} = 2x+1\]
	\[\frac{\delta u}{\delta y} = -2y\]
	\[\begin{cases}
		\frac{\delta^2 u}{\delta x ^2} = 2\\
		\frac{\delta ^ 2 г}{\delta y^2} = -2
	\end{cases} \Rightarrow 2 -2 = 0 \text{- следовательно, условие Лапласа выполнено}\]
	 \[\begin{cases}
	 	\frac{\delta u }{\delta x} = \frac{\delta v}{\delta y} \\
	 	\frac{\delta u} {\delta y} = - \frac{\delta v}{\delta x}
	 \end{cases} \Rightarrow
	 \begin{cases}
	 	\frac{\delta v}{\delta y} = 2x+1 \\
	 	\frac{\delta v}{\delta x} = 2y
	 \end{cases}\]
	 \[v = 2yx + \phi (y) \ \Rightarrow \frac{\delta v}{\delta y} =2x + \phi'(y) \Rightarrow \phi' (y) = 1 \Rightarrow \phi(y) = y+C\]
	 \[f(x,y) = u(x,y) + iv(x,y) = x^2 - y^2 +x + i (2xy + y + C)  = x + iy + x^2 - y^2 + 2ixy + iC =()*)\]
	 \[z = x + iy \Rightarrow z ^2 = x^2 - y^2 +2ixy\]
	 \[(*) = z + z^2 + iC\]
	 \[f(0) = 0 \Rightarrow C = 0\]
	 \[f(z) = z^2 + z\]
	\subsection{Ответ:}
	
	\section{Вычислить интеграл от функции комплексной переменной по данному пути: $\int_{AB} z Im z^2 dz; \ AB - \text{отрезок прямой } z_A = 0, \ z_B = 1 + i$}
	\subsection{Решение:}
	\subsection{Ответ:}
	\newpage
	
	\section{Найти радиус сходимости степенного ряда: $\sum\limits_{n=1}^{\infty}(\cos^2(n))\cdot z^n$}
	\subsection{Решение:}
	\subsection{Ответ:}
	
	\section{Найти все лорановское разложение данной функции в 0 и в $\infty$: $f(z) = \frac{5z+100}{50z - 5z^2 - 2 z^3}$}
	\subsection{Решение:}
	\subsection{Ответ:}
	
	\section{Найти все лорановское разложение данной функции по степеням $z - z_0$: $f(z) = \frac{z+3}{z^2 - 1},\  z_0 = -2 -2i $}
	\subsection{Решение:}
	\subsection{Ответ:}
	
	\section{Данную функцию разложить в ряд Лорана в окрестности точки $z_0$: $f(z)=ze^{\frac{z}{z-5}}, z_0 = 5$}
	\subsection{Решение:}
	\subsection{Ответ:}
	
	\section{Определить тип особой точки $z = 0$ для данной функции: $f(z) = \frac{\ch(5z) - 1}{e^z - 1 -z}$}
	\subsection{Решение:}
	\subsection{Ответ:}
	
	
	\section{Для данной функции найти все изолированные особые точки и определить их тип: $f(z) = \ctg(\pi z)$}
	\subsection{Решение:}
	\subsection{Ответ:}
	
\end{document}
