\chapter{Разработка программного продукта}
В качестве фреймворка для разработки фронтенд части был выбран Vue.ts и Flask - для бэка. Такой выбор обеспечит быстроту реализации и легкость поддержки в виду минимального набора инструментов и большого количества библиотек.

\section{Фронтэнд}
Разработку фронтэнд части можно разделить на несколько этапов:
\begin{itemize}
	\item Настройка карты и взаимодействие с ней
	\item Формирование основных слоёв и фунцкий обработки данных
	\item Создание стандартизированных запросов к бэку.
	\item Визуализация полученных резальтатов.
\end{itemize}

\subsection*{Карта}
В качестве библиотеки для данной задачи была выбрана OpenLayers, которая предоставляет бесплатный доступ к картам различной проекции.
Для работы была выбрана (ПРОЕКЦИЯ) проекция, которая является наиболее привычной и часто встречающийся в других ГИС системах. Её особенность заключается в сохранении форм при проектирвоании сферы на плоскость. Но такой подход вынуждает отказаться от привычной системы широты и долготы. Поэтому, был написана обёртка для библиотечного  конвертера координат, который преобразует координаты в нужный формат (проекции).

\subsection*{Основные слои и функции}
В качестве UI библиотеки используется библиотека CoreUI, которая бесплатно предоставляет большой набор инструментов для формирования интерфейсов.

Для взаимодействия с картой был создан виджет рисования, который активирует режим рисования и позволяет выбрать, какой объект планируется нарисовать: точку или полиго. В зависимости от выбранного режима, при окончании рисования будет возвращён объект, содержащий координаты и проекцию. Для взаимодействия с ними были реализованы два класса Точка и Полигон, хранящие в себя числовой идентификатор, координаты и текущую проекцию. Кроме того, одним из полей является название объекта, т.к. при работе может возникать множество однотипных объектов, которые требуется отличать.

В качестве хранилища используется store, который подключается библиотекой (что-то с ананасом). Хранилище хоть и является частью библиотеки, но требует написания используемых методов. Такими методами стали: получение всех объектов данного класса, фильтрация по идентификатору, добавление и удаление объектов.

Для отображения сохранённых объектов используется виджет хранения, который включает в себя сворачивающиеся окна по каждому классу объектов. Внутри находится список из имён соответсвующих объектов с кнопкойпоказа координат.

Добавлен виджет поиска по городам. Он представляет собой текстовой поле с кнопкой поиска. При удачном запросе, видимаю область карты перемещается на полученные координаты. Скорость регилируется программно.

Также добавлен функциональный виджет, который представляет собой список кнопок, отвечающих за открытие окна с соответсвующей фичей. На данный момент присутсвует четыре метода анализа: высота, NDVI, облачность и температура.

(Добавить картинку и пронумеровать объекты на ней)

Виджет высот представляет собой набор точек для выбора. При нажати, высвечиваются координаты. Для получения высоты требуется нажать на соответсвующую кнопку. При успешном запросе, выысота быдет выведена в соответсвующее поле. (В данный момент разработана система моделирования поверхности по полигону, но ещё не внедрена)

Виджеты для оставльных метрик идентичны. Каждый представляет собой окно, разделённое на две части: графики и настройки данных. Настройка происходит путём множественного выбора точек и временных интервалов. Добавлен выбор типа графиков для иллюстрации данных. Присутсвует окно детального анализа показателей: амплитуда, минимальные и максимальные значения, отклонения и средние.
Отличие функций только в выборе получаемых данных.

\subsection*{Формирование запросов}
Каждая фича фронта требует получение данных из различных источников: это может быть обращение на получение координат города по его названию или получения температуры во Владивостоке в период лета 2023 года с делением этого интервала на 20 частей. 
Для этого следует стандартизовать завпросы, которые предполагают обмен информацией с бэкэндом. В качестве инструмента формирования запросов выступает openapi-ts - библиотека, которая автоматически формирует запросы по описанию, записанному в yaml файле. Кроме того был написан специальный хэндлер, который обрабаьывает ответы бэкэнд стороны и приводит данные к нужному формату, указанному во всё том же yaml файле.

Каждый запрос отсылается на эндпоинт соответствующией фичи. Общий адрес (host:port/metrics). 

Запрос на получение координат города формируется из query-параметра - названия города. Ожидает получить две координаты - широту и долготу.

Запрос на получение высоты (точечный) - формирует тело запроса из следующего JSON занчения : (ВСТАВИТЬ ЗНАЧЕНИЯ). Ожидает получить одно значение - высоту в формате (ВСТАВАИТЬ ФОРМАТ).

Запросы на получение температуры формирует тело из поля feature  и координат. Ожидает получить списоки температуры и дат, одинокового размера. (ВСТАВИТЬ)

Запросы на получение облочности и NDVI индекс строятся аналогично.

\subsection*{Визуализация}
Для точечной визуализации множественных показателей используется CoreUI, которая также предоставляет возможность отрисовки графиков с возмодностью отключения одного из показателей.
В Данный момент также разработана система отрисовки данных, полученных с помощью обработки полигонов. Но ещё не интегрирована.

\subsection*{Промежуточный вывод}
Таким образом, подобная система способна дать наглядное ознакомление с ключевыми параметрами региона. 
Возможность выбора произвольной точки также делает получение данных наиболее релевантным.