\chapter{Разработка программного продукта}

В качестве фреймворков для реализации клиентской и серверной частей программного продукта были выбраны, соответственно, Vue.js\cite{VueJS2023} с использованием TypeScript и Flask \cite{Flask2023}. Данный выбор обоснован высокой скоростью разработки, простотой поддержки и широким набором готовых библиотек, что обеспечивает эффективность при реализации и дальнейшем масштабировании проекта.

\section{Фронтенд}

Разработка пользовательского интерфейса (см. приложение \ref{cha:MainUI}) была организована поэтапно:

\begin{itemize}
	\item Инициализация и настройка карты, организация взаимодействия с пользователем;
	\item Формирование основных слоёв отображения и реализация функций обработки пространственных данных;
	\item Стандартизация и реализация API-запросов к серверной части;
	\item Визуализация полученных данных и аналитических результатов.
\end{itemize}

\subsection*{Карта}

В качестве основной библиотеки для отображения и взаимодействия с картографическими данными была выбрана OpenLayers \cite{OpenLayers2023}, предоставляющая широкие возможности работы с картами различных проекций. В рамках проекта использовалась проекция EPSG:3857 — прямоугольная система координат, основанная на проекции Меркатора \cite{Borovik2010}, построенной по системе параметров WGS84\cite{EPSG4326}, которая является одной из наиболее распространённых в современных ГИС-системах. Данная проекция обеспечивает сохранение формы объектов при отображении сферической поверхности на плоскости, что, однако, требует преобразования координат из географической (широта и долгота) в проекционную систему. Для решения этой задачи была реализована обёртка над встроенными средствами библиотеки по преобразованию координат, обеспечивающая автоматическую конвертацию при необходимости.

\subsection*{Основные слои и функции}

Для построения пользовательского интерфейса применяется библиотека CoreUI\cite{CoreUI2023}, предоставляющая обширный набор компонентов и шаблонов.

Для взаимодействия с картой реализован специализированный виджет рисования, позволяющий пользователю в интерактивном режиме создавать объекты типа «точка» или «полигон». По завершении рисования формируется объект, содержащий координаты и сведения о проекции. В целях дальнейшей обработки были реализованы два класса: \textbf{Точка} и \textbf{Полигон}, каждый из которых содержит числовой идентификатор, координаты, текущую проекцию и пользовательское наименование объекта — что особенно важно при наличии большого количества однотипных элементов.

Для хранения и управления объектами используется централизованное хранилище (store), подключаемое с использованием библиотеки pinia \cite{Pinia2023}. Несмотря на наличие базовой реализации, для корректной работы были разработаны пользовательские методы: получение всех объектов определённого класса, фильтрация по идентификатору, добавление новых элементов и их удаление.

Для визуализации сохранённых данных используется отдельный компонент — \textit{виджет хранения}, включающий в себя разворачивающиеся панели по типам объектов. Внутри отображаются списки с названиями элементов и кнопками для отображения соответствующих координат.

Дополнительно реализован \textit{виджет поиска по городам}, представляющий собой текстовое поле с кнопкой запуска запроса. При успешной обработке запроса карта автоматически перемещается к заданным координатам, причём скорость анимации регулируется программно.

Также создан \textit{функциональный виджет}, включающий набор кнопок, активирующих интерфейс соответствующего аналитического модуля. В текущей версии доступны четыре функции анализа: высота, NDVI, облачность и температура.

%(Добавить картинку и пронумеровать объекты на ней)

\textit{Виджет высот} реализован в виде выбора точек на карте. При клике отображаются координаты, а для получения значения высоты нужно дополнительно нажать соответствующую кнопку. Результат запроса отображается в выделенном поле. (В настоящий момент разработан функционал моделирования поверхности по полигону, однако его интеграция пока не завершена.)

Виджеты для других метрик построены по аналогичному принципу. Каждый представляет собой окно, разделённое на две части: область отображения графиков и блок настройки параметров. Пользователь может выбрать несколько точек и временных интервалов. Кроме того, доступен выбор типа графика для отображения данных. Реализован модуль детального анализа, включающий расчёт амплитуды, минимальных и максимальных значений, стандартного отклонения и среднего значения. Основное отличие между виджетами заключается в типе обрабатываемых данных.

\subsection*{Формирование запросов}

Каждый функциональный модуль фронтенда требует получения данных из внешних источников, будь то координаты населённого пункта по его названию или температурные данные по определённому региону за лето 2023 года, с разбивкой периода на 20 равных интервалов. Для обеспечения унифицированного взаимодействия с бэкендом была реализована система стандартизации API-запросов.

Основным инструментом генерации запросов выступает утилита openapi-ts \cite{OpenAPI_TS2023}, автоматически формирующая интерфейсы на основе спецификаций, описанных в yaml-файле. Также был разработан специализированный хэндлер, обеспечивающий обработку ответов и приведение данных к формату, указанному в спецификации.

Все запросы отправляются на эндпоинты, соответствующие определённой функциональности. Базовый адрес — \texttt{host:port/api}.

\begin{itemize}
	\item Запрос координат города формируется через query-параметр (название города) и ожидает в ответе координаты: широту и долготу(см. приложение \ref{cha:city}).
	
	\item Запрос высоты (точечный) формируется в виде тела запроса, содержащего JSON-объект (см. приложение \ref{cha:city}). Ожидаемый ответ — числовое значение высоты в формате: \texttt{"elevation": 279}.
	
	\item Запрос температуры содержит поля, отвечающие за временной интервал, количество интервалов и координаты (см. приложение \ref{cha:point}); ожидаемый ответ — списки значений температуры и дат одинаковой длины(см. приложение \ref{cha:pointresponse}). 
	
	\item Запросы по облачности и NDVI строятся по аналогичной схеме.
\end{itemize}

\subsection*{Визуализация}

Для визуализации множественных точечных данных применяется функционал библиотеки CoreUI, обеспечивающий построение интерактивных графиков с возможностью отключения отдельных серий данных.

Дополнительно реализована система визуализации, основанная на обработке данных, полученных по полигонам. В настоящее время данная система находится на стадии тестовой интеграции.

\section{Бэкенд}

Как отмечалось ранее, серверная часть приложения реализована с использованием фреймворка Flask и выполняет функции прокси-сервера. Основная задача заключается в получении запросов от клиентской части, их обработке и взаимодействии с библиотекой GEE\cite{GoogleEarthEngine2023}.

Для полноценной работы с GEE потребовалась предварительная настройка API в облачной консоли, создание пользователя с соответствующими учётными данными, используемыми для авторизации при обращении к сервису.

Исключением является модуль поиска по городам, где используется внешний API — \texttt{OpenWeatherMap}. По названию города отправляется запрос на сторонний сервер, результатом которого являются координаты населённого пункта. Доступ к API осуществляется по пользовательскому ключу, полученному при регистрации, а фронтенду передаются только координаты.

Для обработки данных дистанционного зондирования с использованием библиотеки \texttt{ee} создаются объекты соответствующего типа (точка или полигон). После этого выбирается нужный датасет, и производится фильтрация данных по пространственным и временным параметрам. Затем формируется запрос к серверам GEE, и полученные данные преобразуются в формат, пригодный для отправки клиенту. На этом этапе задача бэкенда завершается.

\subsection*{Промежуточный вывод}

Таким образом, разработанная система предоставляет пользователю интуитивно понятный и наглядный доступ к ключевым параметрам выбранного региона. Возможность работы как с произвольными точками, так и с полигонами существенно повышает точность и репрезентативность получаемой информации.
