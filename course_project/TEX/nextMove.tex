\chapter{Дальнейшие планы}

На следующем этапе развития программного продукта планируется поэтапная доработка функциональности и расширение аналитических возможностей сервиса.

В первую очередь будет реализована поддержка параметров второй очереди, таких как количество осадков, типы почв, наличие и распределение водных и инфраструктурных ресурсов. Расширение спектра анализируемых показателей позволит повысить полноту оценки территорий и расширить спектр прикладных сценариев использования платформы.

Параллельно будет осуществлена разработка личного кабинета пользователя и панели администратора. Внедрение механизмов аутентификации и индивидуального хранения пользовательских данных обеспечит персонализированный подход к взаимодействию с системой и повысит общий уровень удобства (UX). Дополнительно, реализация внешнего хранилища и использование механизмов кэширования призваны снизить задержки при повторной обработке запросов, что критически важно для обеспечения высокой производительности веб-приложения.

Следующим направлением станет запуск маркетинговой кампании, направленной на привлечение пользователей из ключевых целевых групп. Запланированы мероприятия по сбору и анализу пользовательской обратной связи с целью последующей адаптации интерфейсов и функциональности к реальным потребностям аудитории.

В более долгосрочной перспективе планируется интеграция механизмов анализа новостных источников (агрегаторов) с целью автоматического выявления значимых событий, таких как криминогенные инциденты, экологические угрозы и другие антропогенные риски. Также рассматривается возможность внедрения инструментов на базе больших языковых моделей для реализации интеллектуального поиска и подбора территорий на основании текстового описания, предоставленного пользователем. Это позволит осуществлять более гибкий и человекоцентричный подход к анализу и интерпретации пространственной информации.