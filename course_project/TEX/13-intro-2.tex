\section{Цель и задачи проекта}

\textbf{Целью} данного проекта является разработка цифрового сервиса для анализа территорий с использованием данных дистанционного зондирования Земли (ДЗЗ) и сопутствующих геопространственных источников. Сервис предназначен для наглядного представления и оценки ключевых характеристик выбранной местности с учётом климатических, экологических и геофизических параметров.

Для достижения поставленной цели сформулированы следующие \textbf{задачи}:

\begin{itemize}
	\item Определение переченя параметров анализа (климат, рельеф, загрязнения, водные ресурсы и др.) и установка приоритетность их реализации на раннем этапе разработки;
	\item Выделуние сферы деятельности и группы людей для применения будущего сервиса.
	\item Поиск и верификация актуальных источников данных, включая открытые спутниковые платформы (например, Copernicus, NASA EarthData, USGS Earth Explorer);
	\item Предварительный анализ датасетов;
	\item Проектирование архитектуры программного решения, включая бэкенд, интерфейс пользователя и модуль интеграции с внешними API;
	\item Разработка минимально жизнеспособный продукт (MVP), демонстрирующий основные функции анализа и визуализации данных по выбранной территории.
\end{itemize}
