\chapter{Заключение}
На основании поставленной цели и сформулированных задач можно сделать вывод о комплексном и системном подходе к разработке цифрового сервиса для пространственного анализа территорий на основе данных дистанционного зондирования. Проект охватывает ключевые этапы — от концептуального планирования и выбора приоритетных характеристик анализа до проектирования архитектуры программного решения и разработки его минимально жизнеспособной версии. Такой подход позволяет не только обеспечить функциональную полноту и масштабируемость создаваемой системы, но и на ранних этапах ориентироваться на потребности конечных пользователей, включая специалистов в области экологии, градостроительства, сельского хозяйства и управления рисками.

Разработка сервиса с возможностью визуализации пространственных данных, полученных из надёжных открытых источников (таких как Copernicus или NASA EarthData), обеспечивает высокую достоверность анализа и потенциал для дальнейшего расширения функциональности. Реализация данного проекта создаёт прочную основу для формирования цифрового инструмента, способного повысить эффективность принятия решений в задачах оценки состояния окружающей среды, мониторинга изменений и планирования устойчивого развития территорий.
