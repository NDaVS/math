\chapter{Введение}
\section{Актуальность}
С древнейших времён человечество стремилось обосноваться в наиболее благоприятных регионах.
Однако в силу большого разнообразия климатических зон, географических условий и изменчивости экологических показателей, задача выбора оптимального места остаётся сложной и многомерной. 
В современном мире этот выбор осложняется глобальными изменениями климата, урбанизацией, ростом загрязнения окружающей среды и увеличением мобильности населения, включая рост числа цифровых кочевников — удалённых работников, не привязанных к конкретной географической локации.

В наше время использование цифровых инструментов становится обыденностью. Подобный инструмент позволил бы агрегировать, визуализировать и анализировать разнородные геоданные для поддержки принятия решений. 

Создание сервиса, наглядно отображающего ключевые характеристики выбранного региона, отвечает вызовам времени и может использоваться в широком спектре задач:
\begin{itemize}
	\item Обычными пользователями — для подбора комфортных для жизни регионов с учётом климата, экологии и инфраструктуры;
	\item Цифровыми кочевниками — для сравнения условий в разных регионах мира по заданным критериям;
	\item Научными организациями — для изучения изменений климата;
	\item Экологическими фондами — для мониторинга загрязнений и охраны окружающей среды;
	\item Недропользователями и геологоразведочными компаниями — для оценки перспективности территорий с точки зрения наличия полезных ископаемых.
\end{itemize}

Таким образом, разработка такого сервиса является актуальной и востребованной задачей, лежащей на пересечении информационных технологий, географии, экологии.

\section{Цель и задачи проекта}

\textbf{Целью} данного проекта является разработка цифрового сервиса для анализа территорий с использованием данных дистанционного зондирования Земли и сопутствующих геопространственных источников. Сервис предназначен для наглядного представления и оценки ключевых характеристик выбранной местности с учётом климатических, экологических и геофизических параметров.

Для достижения поставленной цели сформулированы следующие \textbf{задачи}:

\begin{itemize}
	\item Определение перечня параметров анализа (климат, рельеф, загрязнения, водные ресурсы и др.) и установление приоритетности их реализации на раннем этапе разработки;
	\item Выделение сферы деятельности и группы людей для применения будущего сервиса.
	\item Поиск и верификация актуальных источников данных, включая открытые спутниковые платформы (например, Copernicus, NASA EarthData, USGS Earth Explorer);
	\item Предварительный анализ датасетов;
	\item Проектирование архитектуры программного решения, включая бэкенд, интерфейс пользователя и модуль интеграции с внешними API;
	\item Разработка минимально жизнеспособного продукта (MVP), демонстрирующий основные функции анализа и визуализации данных по выбранной территории.
\end{itemize}
