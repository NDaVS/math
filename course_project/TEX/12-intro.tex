\section{Актуальность}
Жизнь человека напрямую зависит от факторов окружающей среды: рельефа, температуры, влажности воздуха, наличия водоёмов, уровня загрязнения и других природных и антропогенных характеристик. 
Изучение и учёт этих факторов играют критическую роль при выборе места жительства, ведения хозяйства и планирования экономической деятельности.

С древнейших времён человечество стремилось обосноваться в наиболее благоприятных регионах.
Однако в силу большого разнообразия климатических зон, географических условий и изменчивости экологических показателей, задача выбора оптимального места остаётся сложной и многомерной. 
В современном мире этот выбор осложняется глобальными изменениями климата, урбанизацией, ростом загрязнения окружающей среды и увеличением мобильности населения, включая рост числа цифровых кочевников — удалённых работников, не привязанных к конкретной географической локации.

Современные реалии требуют наличия цифровых инструментов, которые позволили бы агрегировать, визуализировать и анализировать разнородные геоданные для поддержки принятия решений. 
Создание сервиса, наглядно отображающего ключевые характеристики выбранного региона, отвечает вызовам времени и может использоваться в широком спектре задач:
\begin{itemize}
	\item Обычными пользователями — для подбора комфортных для жизни регионов с учётом климата, экологии и инфраструктуры;
	\item Цифровыми кочевниками — для сравнения условий в разных регионах мира по заданным критериям;
	\item Научными организациями — для изучения изменений климата;
	\item Экологическими фондами — для мониторинга загрязнений и охраны окружающей среды;
	\item Недропользователями и геологоразведочными компаниями — для оценки перспективности территорий с точки зрения наличия полезных ископаемых.
\end{itemize}

Таким образом, разработка такого сервиса является актуальной и востребованной задачей, лежащей на пересечении информационных технологий, географии, экологии.