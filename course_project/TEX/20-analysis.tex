\chapter{Поиск целевой аудитории}

Разрабатываемый сервис ориентирован на широкий спектр пользователей, заинтересованных в пространственно-временном анализе окружающей среды. Основные целевые группы включают:

\begin{itemize}
	\item Людей, рассматривающих возможность смены места жительства;
	\item Туристов и цифровых кочевников;
	\item Экологов и аналитиков;
	\item Геологоразведчиков и специалистов по природопользованию.
\end{itemize}

Рассмотрим подробнее мотивацию и потребности каждой из этих групп.

\section*{Граждане, планирующие смену региона проживания}

Эта категория пользователей ищет оптимальное место для жизни, учитывая индивидуальные предпочтения по климату, экологическим условиям и геоморфологии местности. Традиционно подобный выбор требует анализа разрозненных источников информации и не всегда сопровождается наличием объективных статистических данных. Предлагаемый сервис агрегирует климатические, экологические и пространственные показатели в единой интерактивной системе, позволяя сравнивать различные регионы по заданным критериям и формировать обоснованные решения.

\section*{Туристы и цифровые кочевники}

Для этой группы важна возможность гибкой смены места пребывания с сохранением качества жизни и рабочих условий. Пользователи, стремящиеся покинуть перегруженные города в пользу более спокойных и благоприятных районов, могут использовать платформу для подбора оптимальных локаций. Сервис предоставляет данные по шумовому фону, климату, зелёным зонам и другим характеристикам, позволяющим найти комфортное место для проживания, работы и отдыха.

\section*{Экологи и пространственные аналитики}

Платформа может служить инструментом для мониторинга и анализа экологических параметров, таких как динамика растительного покрова, уровни загрязнённости, сезонные колебания NDVI и прочее. При партнёрстве с научными и исследовательскими организациями возможно расширение функциональности под конкретные задачи. Пользователи могут направлять запросы на включение дополнительных индикаторов или источников данных, что делает систему адаптивной к различным исследовательским сценариям.

\section*{Геологоразведчики и специалисты по недропользованию}

Для специалистов, занятых поиском полезных ископаемых и оценкой перспективности территорий, сервис может предложить инструменты дистанционного анализа на основе спутниковых снимков, цифровых моделей рельефа и исторических геологических данных. Возможность оперативного выявления аномалий, анализа минералогического состава поверхности и оценки изменений ландшафта открывает перспективы для более точного планирования полевых работ.

Таким образом, создаваемая платформа охватывает различные слои профессиональной и гражданской аудитории, предоставляя гибкие инструменты пространственного анализа для принятия решений на основе открытых спутниковых и экологических данных.
