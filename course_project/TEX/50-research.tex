\chapter{Определение перечня параметров}
\section{Выбор параметров}
Для спокойной жизни человеку нужны комфортные показатели окружающей среды. 
Не должно быть слишком жарко или холодно, влажно или сухо и прочие.

На основании потребностей человека был определён список параметров, которые характеризуют местность:
\begin{itemize}
	\item \textbf{Высота над ровнем моря} - базовый географический параметр. Использование информации  области для создатния макета рельефа.
	\item \textbf{Облачность} - Один из климатических факторов, характеризующих комфортность среды. Показатель солнечных дней в году.
	\item \textbf{Температура} - Ключевой показатель комфортности среды.
	\item \textbf{Параметр растительности} - NDVI - индекс \cite{Cherepanov2011}, характеризующий скорость роста и качество растений в области.
	\item \textbf{Уровень осадков} - Показатель влажности региона, отражающий водный баланс.
	\item \textbf{Типы почв} - Характеристика местности, показывающая возможность для сельскохозяйственной деятельности и строительства.
	\item \textbf{Водные ресурсы} - Показатель наличия на поверхности источников питьевой воды и гидрологических объектово.
	\item \textbf{Инфраструктурные данные} - Отражает транспортную доступность и общее удобство жизни.
\end{itemize}

Для реализации сервиса был выбран подход постепенной реализации анализа выбранных параметров.
Были образованы три очереди разработки. 
Каждая из них характеризует определенный этап: начальный, включающий только основные паказатели местности;  базовый, включающий в себя реализацию анализа всех выбранных характеристик, и использование методов машинного обучения для подбора мест по описанию пользователя.
\newpage
\section{Приоритезация разработки}
Каждый этап ключает ряд признаков и подходов для их анализа.

Начальный этап (первая очередь):
\begin{itemize}
	\item Высота над уровнем моря;
	\item Облачности;
	\item Температура;
	\item NDVI-индекс
\end{itemize}

Базовый этап (вторая очередь):
\begin{itemize}
	\item Уровень осадков;
	\item Типы почв;
	\item Водные ресурсы;
	\item Инфраструктурные данные
\end{itemize}

Этап использование ML и BigData (третья очередь):
\begin{itemize}
	\item Анализ различных новостных агрегаторов на предмет важных событий: криминальная обстановка, экологические проблемы и прочие антропогенные факторы.
	\item Использование LLM для подбора места по описанию от пользователя 
\end{itemize}

Такой подход способствует концентрации на разработке только определённых функций.
Кроме того, это позволит провести тестирование идеи уже после завершения начального этапа.

Данная работа описывает только процеcc разработки начального этапа.