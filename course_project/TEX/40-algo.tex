\chapter{Поиск, верификация и предварительный анализ спутниковых данных}

\section{Поиск и отбор актуальных источников данных}

Для проведения пространственно-временного анализа климатических и ландшафтных показателей (NDVI, облачность, температура, высота местности) были использованы актуальные и проверенные источники спутниковых данных:

\begin{itemize}
	\item \textbf{Copernicus (ESA)} — источник снимков Sentinel-2 с высоким пространственным разрешением (10 м) и частотой съёмки (до 5 дней). Используемые коллекции:

		 \texttt{COPERNICUS/S2\_SR} — отражательная способность по спектральным каналам (в частности, B4 и B8 для расчёта NDVI);
		 \texttt{COPERNICUS/S5P/NRTI/L3\_CLOUD} — данные по облачности в режиме близком к реальному времени (переменная \texttt{cloud\_fraction}).

	\item \textbf{NASA EarthData} — глобальный источник цифровых моделей рельефа:  \texttt{NASA/NASADEM\_HGT/001} — цифровая модель высот с разрешением $\sim$30 м, на основе миссии SRTM.

	
	\item \textbf{ECMWF ERA5} — глобальные климатические данные, использованы для анализа температуры:
	 \texttt{ECMWF/ERA5\_LAND/HOURLY} — почасовые значения температуры на высоте 2 м (\texttt{temperature\_2m}), разрешение около 9 км.

\end{itemize}

\section{Верификация достоверности и актуальности данных}

Для обеспечения корректности анализа были выполнены следующие шаги верификации:

\begin{enumerate}
	\item \textbf{Пространственное покрытие}: все используемые коллекции фильтруются по координатам исследуемых участков с помощью метода \texttt{filterBounds()}.
	
	\item \textbf{Временной интервал}: фильтрация по дате осуществляется через \texttt{filterDate(start, end)}; используются интервалы по месяцам или сезонам.
	
	\item \textbf{Качество данных}:
	\begin{itemize}
		\item Sentinel-2 фильтруется по показателю \texttt{CLOUDY\_PIXEL\_PERCENTAGE < 20\%};
		\item исключаются значения \texttt{None} и выбросы по температуре и NDVI.
	\end{itemize}
	
	\item \textbf{Техническая пригодность}: все источники проверены на совместимость с API Google Earth Engine — коллекции успешно загружаются, ошибок доступа не зафиксировано.
\end{enumerate}

\section{Предварительный анализ данных}

\subsubsection*{NDVI (Normalized Difference Vegetation Index)}

\begin{itemize}
	\item Источник: \texttt{COPERNICUS/S2\_SR}
	\item Формула:
	\[
	\text{NDVI} = \frac{B8 - B4}{B8 + B4}
	\]
	где $B8$ — ближний инфракрасный канал, $B4$ — красный канал. Расчёт в GEE осуществляется через функцию \texttt{normalizedDifference(['B8', 'B4'])}.
	\item Разрешение: 10 м
	\item Получение значений: среднее по точке через \texttt{reduceRegion(..., scale=10)}
\end{itemize}

\subsubsection*{Облачность}

\begin{itemize}
	\item Источник: \texttt{COPERNICUS/S5P/NRTI/L3\_CLOUD}
	\item Метрика: \texttt{cloud\_fraction}
	\item Разрешение: 1000 м
	\item Среднее значение рассчитывается на каждый временной интервал для каждой координаты
\end{itemize}

\subsubsection*{Температура воздуха}

\begin{itemize}
	\item Источник: \texttt{ECMWF/ERA5\_LAND/HOURLY}
	\item Переменная: \texttt{temperature\_2m}
	\item Разрешение: 1000 м
	\item Применяется усреднение по времени и пространству для каждой точки
\end{itemize}

\subsubsection*{Цифровая модель рельефа (DEM)}

\begin{itemize}
	\item Источник: \texttt{NASA/NASADEM\_HGT/001}
	\item Разрешение: $\sim$30 м
	\item Методика: извлечение значений высот через \texttt{sampleRegions()} по геометрии исследуемых точек
	\item Масштаб автоматически регулируется (от 30 до 500 м) для оптимизации количества точек
\end{itemize}

Выбранные спутниковые источники Copernicus, NASA и ECMWF являются авторитетными и проверенными платформами, обеспечивающими доступ к достоверным пространственно-временным данным. Верификация по охвату, времени, качеству и совместимости подтвердили пригодность данных для целей проекта. Предварительный анализ NDVI, облачности, температуры и высот показал, что структура и формат полученных данных соответствуют задачам курсового исследования, а применённые методы обеспечивают надёжную основу для последующего машинного обучения и статистической обработки.
