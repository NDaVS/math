\documentclass[a4paper,12pt]{article}

% Подключаем пакеты
\usepackage[utf8]{inputenc} % Кодировка UTF-8
\usepackage{listings}
\usepackage{caption}
\usepackage{tikz}
\usepackage[english,russian]{babel} % Поддержка русского языка
\usepackage{indentfirst} % Отступ первой строки после заголовков
\usepackage{geometry} % Настройка полей
\geometry{top=2cm, bottom=2cm, left=2.5cm, right=2.5cm}

% Настройка межстрочного интервала
\usepackage{setspace}
\onehalfspacing % Полуторный интервал

% Настройка шрифтов
\usepackage{amsmath,amsfonts,amssymb} % Математические пакеты

\lstset{
	frame=single, % Рамка вокруг кода
	numbers=left, % Номера строк слева
	numberstyle=\tiny\color{gray}, % Стиль номеров строк
	keywordstyle=\color{blue}, % Цвет ключевых слов
	commentstyle=\color{green}, % Цвет комментариев
	stringstyle=\color{red}, % Цвет строк
	basicstyle=\ttfamily, % Основной шрифт
	breaklines=true % Перенос длинных строк
}

\begin{document}
	
	% Заголовок статьи
	\begin{center}
		{\LARGE Название статьи} \\[1cm]
		{\large Автор: Имя Фамилия} \\[0.5cm]
		{\large Дата: \today}
	\end{center}
	
	\newpage
	
	\pagenumbering{gobble}
	
	\tableofcontents
	
	% Включаем нумерацию страниц снова
	\newpage
	\pagenumbering{arabic}
	\setcounter{page}{3} % Нумерация страниц начинается с 3

	
	\section*{Введение}
	
	В данной работе рассматривается метод Ричардсона - итерационный процесс решения систем линейных алгебраических уравнений (СЛАУ) вида $Ax = b$, где $A$ - квадратная матрица, $x$ - искомый вектор, а $b$ - вектор правой части. Также представлена программная реализация этого процесса с тестированием на предложенных данных и анализ полученных результатов.
	
	К целям данной работы относится не только теоретическое ознакомление с методом Ричардсона, но и программная релизация данного метода.
	
	\newpage
	
	\section{Постановка задачи}
	\begin{itemize}
		\item Изучить теорию по методу Ричадсона.
		\item Реализовать метод программно на ЯП Python\cite{python} \cite{idris2014learning}.
		\item Применить программу для тестовых данных и проанализировать точность.
	\end{itemize}

	
	\subsection{Необходимые условия}
	\begin{enumerate}
		\item \textbf{Симметричность матрицы}. Хотя метод Ричардсона может работать для несимметричных матриц, для теоретической гарантии сходимости проще рассматривать случай, когда 
		$A$ — симметричная матрица.
		
		\item  \textbf{Положительная определённость}. Это гарантирует, что уравнение 
		$Ax=b$ имеет единственное решение, а спектр собственных значений $A$ будет положительным ($\forall i:\lambda_i > 0$).
	\end{enumerate}
	
	\subsection{Тестовые данные}
	\[
	A_0 =
	\begin{pmatrix}
		2 & 1 & 1 \\
		1 & 2.5 & 1 \\
		1 & 1 & 3
	\end{pmatrix};
	\]
	
	\[
	A_1 =
	\begin{pmatrix}
		-0.168700 & 0.353699 & 0.008540 & 0.733624 \\
		0.353699 & 0.056519 & -0.723182 & -0.076440 \\
		0.008540 & -0.723182 & 0.015938 & 0.342333 \\
		0.733624 & -0.076440 & 0.342333 & -0.045744
	\end{pmatrix};
	\]
	
	\[
	A_2 =
	\begin{pmatrix}
		1.00 & 0.42 & 0.54 & 0.66 \\
		0.42 & 1.00 & 0.32 & 0.44 \\
		0.54 & 0.32 & 1.00 & 0.22 \\
		0.66 & 0.44 & 0.22 & 1.00
	\end{pmatrix};
	\]
	
	\[A_3 = 
	\begin{pmatrix}
		2 & 1\\
		1 & 2\\
	\end{pmatrix}.
	\]
	
	В качестве вектора правой части будем использовать вектор, состоящий из единиц; длина которого равно количеству строк/столбцов рассматриваемой матрицы. Кроме последней матрицы: в этом случае будем использовать заданный вектор с компонентами 4 и 5.
	
	\newpage
	
	\section{Метод Ричардсона}
	\textbf{Метод Ричардсона\cite{cm}} — это метод, используемый для повышения точности численных вычислений, особенно при оценке пределов или приближённых решений. Метод часто применяется в задачах, связанных с решением дифференциальных уравнений, вычислением интегралов, интерполяцией или решением систем линейных уравнений.
	
	Итерационная формула метода:
	\[
		\frac{x^{(k+1)} - x ^{(k)}}{\tau_{k+1}}  + A x^{(k)}= b. 
	\]
	
	Отсюда получается следующий итерационный процесс:
	\[
		x^{(k+1)} = \left(b - A x^{(k)}\right) \tau_{k+1} + x^{(k)}.
	\]
	
	Чебышовский набор итерационных парметров:
	$$\tau_k = \frac{\tau_0}{1 + \rho_0 v_k} \text{ - параметр релаксации на следующей итерации, где:} $$
	$$\tau_0 = \frac{2}{\lambda_{\min}(A) + \lambda_{\max}(A)} \text{ - оптимальный базовый параметр релаксации},$$
	$$\rho_0 = \frac{1 - \eta}{1 + \eta},$$
	$$\eta = \frac{\lambda_{\min}(A)}{\lambda_{\max}(A)},$$
	$$v_k = \cos\left(\frac{(2k - 1)\pi}{2n}\right) \text{ - представление многочлена Чебышева},$$
	$$n \text{ - это общее число внутренних шагов или этапов итерации,}$$
	$$k \text{ - текущий номер итерации.}$$
	

	\newpage
	\section{Программаная реализация}
	\begin{figure}[h]
		\begin{verbatim}
import numpy as np
import math
from calculus.third_course.support.Rotation import RotationWithBarriers
from calculus.third_course.support.checker import check_answer


class Richardson:
	def __init__(self, A: np.array, b: np.array, x: np.array, p: int = 4):
		self._tol = math.pow(10, - (p-1))
		self._eta = None
		self._tau_0 = None
		self._rho_0 = None
		self._A = A
		self._b = b
		self._lambda_min = None
		self._lambda_max = None
		self._n = 8
		self._x = x
		self._p = p
	
	def _compute_SZ(self):
		rotator = RotationWithBarriers(self._A, self._p)
		sz = rotator.compute()
		self._lambda_max = sz[-1]
		self._lambda_min = sz[0]
	
	def _compute_tau_0(self):
		self._tau_0 = 2 / (self._lambda_min + self._lambda_max)
	
	def _compute_eta(self):
		self._eta = self._lambda_min / self._lambda_max
	
	def _compute_rho_0(self):
		self._rho_0 = (1 - self._eta) / (1 + self._eta)

	def _v_k(self, k):
		return np.cos((2 * k - 1) * np.pi /	(2 * self._n))
		\end{verbatim}
	\end{figure}
	\newpage
		\begin{figure}[h]
		\begin{verbatim}
	def _t_k(self, k):
		return self._tau_0 / (1 + self._rho_0 * self._v_k(k))

	def _compute_x(self, x: np.array):
		x_k = x
		
		for k in range(1, self._n + 1):
			x_k = (self._b - self._A @ x_k) * self._t_k(k) + x_k
	
	return x_k
	
	def compute(self):
		self._compute_SZ()
		self._compute_eta()
		self._compute_rho_0()
		self._compute_tau_0()
		x = self._x
		iterations = 0
		
		while np.linalg.norm(self._A @ x - self._b) > self._tol:
			x = self._compute_x(x)
			iterations += 1
			
		return x, iterations * self._n

def main():
	A = np.array([[2, 1],
				[1, 2]])
	b = np.array([4,5])
	x = np.zeros_like(b)
	richardson = Richardson(A, b, x)
	x, iterations = richardson.compute()
	
	print(f"Iteration proccess complete in {iterations} iterations)
	check_answer(A, b, x)
		\end{verbatim}
		\caption{Код программы для метода Ричардсона}
	\end{figure}
	\newpage
	\section{Анализ результатов решения}
	Рассмотрим результаты для матрицы $A_0$:
		\[
	\begin{pmatrix}
		2 & 1 & 1 \\
		1 & 2.5 & 1 \\
		1 & 1 & 3
	\end{pmatrix} x = 
	\begin{pmatrix}
		1\\
		1\\
		1
	\end{pmatrix}.
	\]
	
	При заданой точности $tol=10^{-4}$ и $n=8$, был получен результат за 16 итерации:
	\[
		x = 
		\begin{pmatrix}
			1.18509336\\ 1.7598606  \\4.55504604
		\end{pmatrix}
	\]
	
		Невязка результата\cite{mathcalc}:
		\[
			||Ax - b||_2 = 1.04e-07.
		\]
		
		Результат удовлетворяет заданной точности.
		
		Рассмотрим результаты для матрицы $A_1$:
		\[
		\begin{pmatrix}
			-0.168700 & 0.353699 & 0.008540 & 0.733624 \\
			0.353699 & 0.056519 & -0.723182 & -0.076440 \\
			0.008540 & -0.723182 & 0.015938 & 0.342333 \\
			0.733624 & -0.076440 & 0.342333 & -0.045744
		\end{pmatrix}
		x = 
		\begin{pmatrix}
			1\\
			1\\
			1\\
			1
		\end{pmatrix}.
		\]
	
		Для данной матрицы метод Ричардсона расходится, т.к. матрица $A_1$ обладает следующими собственными числами:
		\[ \lambda (A) = \{ -0.94356786, -0.74403641,  0.68784308,  0.85777418 \}.\]
	
		По условиям, накладываемым на матрицу $A$, метод не гарантирует сходимость из-за наличия отрицательных собственных значений. 
		
	
		Рассмотрим результаты для матрицы $A_2$:
		\[
			\begin{pmatrix}
				1.00 & 0.42 & 0.54 & 0.66 \\
				0.42 & 1.00 & 0.32 & 0.44 \\
				0.54 & 0.32 & 1.00 & 0.22 \\
				0.66 & 0.44 & 0.22 & 1.00
			\end{pmatrix} x = 
			\begin{pmatrix}
				1\\1\\1\\1
			\end{pmatrix}.
		\]	
	
		При заданой точности $tol=10^{-4}$ и $n=8$, был получен результат за 24 итерации:
		\[
			x = 
			\begin{pmatrix}
				0.24226071 \\ 0.6382838\\  0.79670669\\ 2.3227488
			\end{pmatrix}
		\]
	
		Невязка результата:
		\[
			||Ax - b||_2 = 1.66e-06.
		\]
		
	Результат удовлетворяет заданной точности.
	
	Рассмотрим результаты для матрицы $A_3$:
	\[
	\begin{pmatrix}
		2 & 1  \\
		1 & 2
		
	\end{pmatrix} x = 
	\begin{pmatrix}
		4\\5\\
	\end{pmatrix}.
	\]	
	
	При заданой точности $tol=10^{-4}$ и $n=8$, был получен результат за 16 итерации:
	\[
	x = 
	\begin{pmatrix}
		1 \\ 2
	\end{pmatrix}
	\]
	
	Невязка результата:
	\[
	||Ax - b||_2 = 1.81e-08.
	\]
	
	Результат удовлетворяет заданной точности.
	\newpage
	\section{Заключение}
	В результате лабораторной работы был успешно реализован метод Ричардсона. Проведено тестирование программы на разных видах матриц, в том числе неположительно определнных. Полученные результаты показывают эффективность метода для решения СЛАУ.
	\newpage
	
	\bibliographystyle{plain} % стиль библиографии, можно поменять на другой
	
	
	\bibliography{biblio}
	
\end{document}
