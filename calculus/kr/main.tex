\documentclass[12pt,a4paper]{article}
\usepackage[T2A]{fontenc}
\usepackage[utf8]{inputenc}
\linespread{1.3}

\usepackage[english,russian]{babel} 
\usepackage{fontspec} 
\usepackage{amsmath }
\defaultfontfeatures{Ligatures={TeX},Renderer=Basic} 
\setmainfont[Ligatures={TeX,Historic}]{Times New Roman}
\usepackage{geometry}
\usepackage{indentfirst}
\usepackage{spverbatim}
\geometry{
	left=3cm,
	right=1cm,
	top = 2cm,
	bottom = 2cm
}
\setlength{\parindent}{1.25cm}

% Настройка титульной страницы
%\setlength{\droptitle}{-3em} % Отступ заголовка
\title{Контрольная работа по теме численное интегрирование}
\author{Вершинин Данил Алексеевич}
\date{\today}

% Настройка размеров страницы
%\geometry{a4paper, margin=2cm}

\begin{document}
	
	% Автоматическая генерация оглавления (см. далее)
	\maketitle
	\thispagestyle{empty}
	
	
	\section{Формулировка задания}
	Для вычисления $\int\limits_0^1 f(x)dx$ применяется составная формула трапеций. Оценить минимальное число разбиений N, обеспечивающее точность $10^{-3}$ на классе функций: \newline$||f''(x)||_{L_1} = \int\limits_0^1|f''(x)dx| \le 1$
	
	\section{Решение}
	Погрешность вычисления интеграла на отрезке методом трапеций равна
	\[R_{2i}(f)=\int\limits_{x_i}^{x_{i+1}}f(x)dx - (x_{i+1} - x_i)\frac{f(x_{i+1})+f(x_i)}{2} \eqno(1)\]
	Заметим, что для оценки погрешности также справедлива следующая формула :
	\[R_{2i}(f)=\frac{1}{2}\int\limits_{x_i}^{x_{i+1}} (x-x_i)(x-x_{i+1}) f''(x)dx\]
	Её можно проверить, проинтегрировав два раза по частям
	\[\frac{1}{2}\int\limits_{x_i}^{x_{i+1}} (x-x_i)(x-x_{i+1}) f''(x)dx \eqno(2)\]
	Покажем это:
	\[\frac{1}{2}\int\limits_{a}^{b} (x-a)(x-b) f''(x)dx = \begin{vmatrix}
		u=(a-x)(b-x) & du=(x -a)+(x+b)dx\\
		dv=f''(x) & v = f'(x)
	\end{vmatrix}=\]
	\[=\frac{1}{2} (a-x)(b-x)f''(x)\Big|_a^b- \frac{1}{2}\int\limits_{a}^b(2x - (a+b))f'(x)dx  = - \frac{1}{2}\int\limits_{a}^b(2x-(a+b))f'(x)dx=\]
	\[ =-\int\limits_{a}^b\left(x-\frac{(a+b)}{2}\right)f'(x)dx = \begin{vmatrix}
		u = x - \frac{a+b}{2} & du = dx\\
		dv = f'(x)dx & v = f(x)
	\end{vmatrix} = \] 
	\[-(f(x)(x - \frac{a+b}{2}))\Big|_a^b + \int\limits_a^bf(x)dx = \int\limits_a^bf(x)dx  - \frac{b-a}{2}(f(b)-f(a)) = (1)\]
	\newline
	Оценим погрешность на отрезке
	\[\frac{1}{2}\int\limits_{x_i}^{x_{i+1}} (x-x_i)(x-x_{i+1}) f''(x)dx \le \frac{\max|((x-x_i)(x-x_{i+1}))|}{2} \int\limits_{x_i}^{x_{i+1}}f''(x)dx \eqno (3)\]
	Оценим $\max|((x-x_i)(x-x_{i+1}))|$:
	\[\text{Очевидно, максимальное значение будет при }x = \frac{x_i + x_{i+1}}{2}, \text{ т.е $x$ -- середина отрезка }[x_i, x_{i+1}]\]	
	Подставив это значение и получим:
	\[\left(\frac{x_i + x_{i+1}}{2} - \frac{2x_i}{2}\right)\left(\frac{x_i + x_{i+1}}{2} - \frac{2x_{i+1}}{2}\right) = -\frac{(x_{i+1} - x_i)^2}{4} = -\frac{h^2}{4}\]
	По модулю получим $h^2/4$. Тогда правая часть (3) перепишется слудующим образом:
	\[\frac{\max|((x-x_i)(x-x_{i+1}))|}{2} \int\limits_{x_i}^{x_{i+1}}f''(x)dx = \frac{h^2}{8}\int\limits_{x_i}^{x_{i+1}}f''(x)dx \]
	И так, мы получили оценку погрешности на отрезке. Теперь нехитрым образом найдём погрешность на всём интервале:
	\[R_2(f) = \sum\limits_{i=1}^{n-1}\frac{h^2}{8}\int\limits_{x_i}^{x_{i+1}}f''(x)dx  = \frac{h^2}{8}\sum\limits_{i=1}^{n-1}\int\limits_{x_i}^{x_{i+1}}f''(x)dx\]
	Зная, что наш интервал равен [0,1] и применяя свойство аддитивности для интегралов, получим:
	\[\frac{h^2}{8}\sum\limits_{i=1}^{n-1}\int\limits_{x_i}^{x_{i+1}}f''(x)dx = \frac{h^2}{8}\int\limits_0^1f''(x)dx\eqno (4)\]
	Но нам дано условие с модулем. Следовательно, применим свойства модуля интеграла:
	\[1 \ge \int |f''(x)|dx \ge \left|\int f''(x)dx\right| \Rightarrow \left|\int f''(x)dx\right| \le 1\]
	Раскрыв модуль, получим неравенство:
	\[-1 \le \int f''(x)dx \le 1\]
	Теперь мы можем оценить (4):
	\[\frac{h^2}{8}\int\limits_{0}^{1}f''(x)dx \le \frac{h^2}{8}\]
	Представив $h$ как отношение длины интервала (у нас 1)  и количества разбиений $n$, получим оценку:
	\[R_2 =\frac{1}{8n^2}\]
	Теперь остаётся только решить неравенство
	\[10^{-3} \ge \frac{1}{8n^2} \Rightarrow n^2 \ge \frac{1000}{8} \Rightarrow n \ge \frac{10\sqrt{10}}{2\sqrt{2}}\]
	\[n \ge 5\sqrt{5} \approx 11.18\]
	Следовательно, если взять $n \ge 12$, то полученная точность будет удовлетворять условию.
	\section{Ответ: N=12}
	

\end{document}
