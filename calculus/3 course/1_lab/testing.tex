\documentclass[12pt]{article}

\usepackage[T2A]{fontenc} % Кодировка шрифта
\usepackage[utf8]{inputenc} % Кодировка ввода
\usepackage[english,russian]{babel} % Языковые настройки
\usepackage{graphicx} % Для вставки изображений
\usepackage{amsmath} % Для использования математических формул
\usepackage{amssymb}
\usepackage{cancel}
\usepackage{tikz}
\usepackage{amsfonts} % Для использования математических символов и шрифтов
\usepackage{titlesec} % Для настройки заголовков разделов
\usepackage{titling} % Для настройки титульной страницы
\usepackage{geometry} % Для настройки размеров страницы
\usepackage{pgfplots}
\pgfplotsset{compat=1.9}

% Настройка заголовков разделов
\titleformat{\section}
{\normalfont\Large\bfseries}{\arabic{section}}{1em}{}
\titleformat{\subsection}
{\normalfont\large\bfseries}{}{1em}{}

% Настройка титульной страницы
\setlength{\droptitle}{-3em} % Отступ заголовка
\title{\vspace{-1cm}Резистивное расстояние}
\author{Вершинин Данил Алексеевич}
\date{\today}

% Настройка размеров страницы
\geometry{a4paper, margin=2cm}

\begin{document}
	
\section{Число обусловленности матрицы}
\subsection{Условие: }
$A = \begin{cases}
	2x+y=2 \\
	(2-\epsilon)x + y = 1
\end{cases}$, \ $\epsilon$ > 0

\subsection{Найти: $\mu(A), \ x, \ y$}
\subsection{Решение}
Найдём $x,y$\\
\[\begin{cases}
	y = 2 - 2x\\
	x = \frac{1-y}{2-\epsilon}
\end{cases} \Rightarrow 
\begin{cases}
	x = \frac{2x-1}{2-\epsilon}\\
	y = 2-2x
\end{cases} \Rightarrow
\begin{cases}
	x = \frac{1}{\epsilon}\\
	y = 2-\frac{2}{\epsilon}
\end{cases}\]
\[A = \begin{pmatrix}
	2 & 1\\
	2-\epsilon & 1
\end{pmatrix},  \ B = \begin{pmatrix}
2 \\ 1
\end{pmatrix}\]

$|A| = 2 - 2 + \epsilon = \epsilon$\\
$||A||_1 = max(2 + |2-\epsilon|, 2)$
\[A^{-1} = \frac{1}{\epsilon}\begin{pmatrix}
	1 & -1 \\
	\epsilon - 2 & 2
\end{pmatrix}\]
$||A^{-1}||_1 = \frac{1}{\epsilon} max(3, |\epsilon - 2| + 1)$\\
\begin{figure}[h] % Параметр h указывает LaTeX разместить рисунок "здесь"
	\centering
	\begin{tikzpicture}
		\begin{axis}[
			axis lines = middle,
			xlabel = \( x \),
			ylabel = { \( y \) },
			xtick = {-5,-4,...,5},
			ytick = {0,1,2,3,4,5},
			ymin = 0, ymax = 5,
			xmin = -5, xmax = 5,
			domain = -5:5,
			samples = 100,
			legend pos = north west,
			]
			\addplot[blue, thick] {2+abs(2 - x)}; 
			\addlegendentry{\(2+|2 - x|\)}
			
			\addplot[red, thick] {2}; 
			\addlegendentry{2}
		\end{axis}
	\end{tikzpicture}
	\caption{Графики функций \(2+|2 - x|\) и \(2\).}
	\label{fig:first}
\end{figure}\\
Из Рис.\ref{fig:first} видно, что максимумом из двух функций на промежутке положительных чисел является $2 + |2-\epsilon|$.
\begin{figure}[h] % Параметр h указывает LaTeX разместить рисунок "здесь"
	\centering
	\begin{tikzpicture}
		\begin{axis}[
			axis lines = middle,
			xlabel = \( x \),
			ylabel = { \( y \) },
			xtick = {-5,-4,...,5},
			ytick = {0,1,2,3,4,5},
			ymin = 0, ymax = 5,
			xmin = -5, xmax = 5,
			domain = -5:5,
			samples = 100,
			legend pos = north west,
			]
			\addplot[blue, thick] {1+abs(x - 2)}; 
			\addlegendentry{\(1+|x - 2|\)}
			
			\addplot[red, thick] {3}; 
			\addlegendentry{2}
		\end{axis}
	\end{tikzpicture}
	\caption{Графики функций \(1+|2 - x|\) и \(3\).}
	\label{fig:second}
\end{figure}\\

Из рис \ref{fig:second} видно, что на $(0,4)$ большей функцией является 3, а для $x\in(4, +\infty)$ -- 1+|x-2|.\\

Рассмотрим три промежутка: $(0,2), \ (2,4), \ (4,+\infty)$. \\
\subsection{$(0,2)$}
Норма $||A||_1$ будет равна $4-\epsilon$\\
Норма $||A^{-1}||_1$ будет равна $\frac{3}{\epsilon}$\\
Таким образом, число обусловленности на данном промежутке будет равно: $\mu(A) = ||A||\cdot ||A^{-1}|| = \frac{12}{\epsilon} - 3$
\subsection{$(2,4)$}
Норма $||A||_1$ будет равна $\epsilon$
Норма $||A^{-1}||_1$ будет равна $\frac{3}{\epsilon}$\\
Таким образом, число обусловленности на данном промежутке будет равно: $\mu(A) = ||A||\cdot ||A^{-1}|| = 3$
\subsection{$(4,+\infty)$}
Норма $||A||_1$ будет равна $\epsilon$\\
Норма $||A^{-1}||_1$ будет равна $\frac{\epsilon-1}{\epsilon}$\\
Таким образом, число обусловленности на данном промежутке будет равно: $\mu(A) = ||A||\cdot ||A^{-1}|| =\epsilon - 1$

\section{Ответ:}
\[\mu(a) = \begin{cases}
	\frac{12}{\epsilon} -3 & \epsilon \in (0, 2)\\
	3 &  \epsilon \in (2, 4)\\
	\epsilon - 1  & \epsilon \in (4, +\infty)\\
\end{cases}\]
\[\begin{cases}
	x = \frac{1}{\epsilon}\\
	y = 2-\frac{2}{\epsilon}
\end{cases}\]


\end{document}