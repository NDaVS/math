\documentclass[a4paper,12pt]{article}

% Подключаем пакеты
\usepackage[utf8]{inputenc} % Кодировка UTF-8
\usepackage{listings}
\usepackage{caption}
\usepackage{tikz}
\usepackage[english,russian]{babel} % Поддержка русского языка
\usepackage{indentfirst} % Отступ первой строки после заголовков
\usepackage{geometry} % Настройка полей
\geometry{top=2cm, bottom=2cm, left=2.5cm, right=2.5cm}

% Настройка межстрочного интервала
\usepackage{setspace}
\onehalfspacing % Полуторный интервал

% Настройка шрифтов
\usepackage{amsmath,amsfonts,amssymb} % Математические пакеты

% Пакет для добавления списка литературы в содержание
\usepackage[nottoc]{tocbibind}

% Пакет для точек в содержании
\usepackage{tocloft}
\renewcommand{\cftsecleader}{\cftdotfill{\cftdotsep}}

\usepackage{float}    % Для возможности размещения кода в виде плавающего окружения

\lstset{
	frame=single,              % Рамка вокруг кода
	numbers=left,              % Нумерация строк
	numberstyle=\tiny\color{gray}, % Стиль нумерации
	keywordstyle=\color{blue}, % Цвет ключевых слов
	commentstyle=\color{green}, % Цвет комментариев
	stringstyle=\color{red},   % Цвет строк
	basicstyle=\ttfamily\small, % Основной шрифт
	breaklines=true,           % Перенос строк
	breakatwhitespace=false,   % Переносить строки в любом месте
	extendedchars=true,        % Поддержка UTF-8
	showspaces=false,          % Не отображать пробелы
	showstringspaces=false,    % Не отображать пробелы в строках
	xleftmargin=15pt,          % Отступ слева
	framexleftmargin=15pt,     % Рамка слева
	postbreak=\mbox{\textcolor{gray}{$\hookrightarrow$}\space}, % Указатель переноса строки
}

\begin{document}
	
	% Заголовок статьи
	\begin{center}
		{\LARGE Название статьи} \\[1cm]
		{\large Автор: Имя Фамилия} \\[0.5cm]
		{\large Дата: \today}
	\end{center}
	
	\newpage
	
	\pagenumbering{gobble}
	
	\tableofcontents*
	
	% Включаем нумерацию страниц снова
	\newpage
	\pagenumbering{arabic}
	\setcounter{page}{3}
	\section{Введение}
	Решение систем линейных алгебраических уравнений (далее СЛАУ) - является одной из фундаментальных задач вычислительной математики, которая широко применяется в различных облостях, начиная с инженерии и заканчивая экономикой и построенимем моделей в биологии. Для решениях СЛАУ были созданы различные методы, каждый из которых имеет свои плюсы и минусы, которые важны для конкретных задач.
	
	В рамках данной курсовой работы исследуются метод Гаусса с выбором ведущего элемента по всей матрицы.
	
	\textbf{Цель работы:} исследовать метод Гаусса с выбором ведущего элемента по всей матрицы, сравнить с обычным методом гаусса.
	
	\textbf{Задачи:}
	\begin{enumerate}
		\item Изучить теоретические основы метода Гаусса с выбором главного элемента по всей матрицы
		
		\item Программно реализовать алгоритм метода
		
		\item Провести вычислительные эксперименты с различными типами матриц
		
		\item Сравнить результаты с обычным методом Гаусса
		
		\item Сформулировать выводы
	\end{enumerate}
	
	Работа направлена на развитие навыков программирования численных методов и углубленное понимание их свойсств.
	
	\section{Основная часть}
	\section*{Метод Гаусса с выбором ведущего элемента по всей матрице}
	Рассмотрим систему линейных уравнений:
	\[
	A \mathbf{x} = \mathbf{b},
	\]
	где \(A\) — квадратная матрица размера \(n \times n\), \(\mathbf{x}\) — вектор неизвестных, а \(\mathbf{b}\) — вектор правых частей.
	
	Метод Гаусса с выбором ведущего элемента по всей матрице выполняется в три этапа:
	
	\subsection*{1. Прямой ход}
	\begin{enumerate}
		\item На \(k\)-м шаге (\(k = 1, 2, \dots, n-1\)) среди оставшихся элементов матрицы \(A\) (начиная с \(k\)-й строки и \(k\)-го столбца) выбирается элемент с максимальным абсолютным значением:
		\[
		|a_{p,q}| = \max\limits_{i \geq k, j \geq k} |a_{i,j}|,
		\]
		где \(p\) и \(q\) — индексы строки и столбца, соответствующих максимальному элементу.
		
		\item Выполняется перестановка строк \(k\)-й и \(p\)-й, а также столбцов \(k\)-го и \(q\)-го. Эти перестановки должны быть учтены в решении.
		
		\item С использованием выбранного ведущего элемента \(a_{k,k}\) производится исключение переменных: строки матрицы и элементы правой части преобразуются по следующим формулам:
		\[
		a_{i,j} := a_{i,j} - \frac{a_{i,k} \cdot a_{k,j}}{a_{k,k}}, \quad b_i := b_i - \frac{a_{i,k} \cdot b_k}{a_{k,k}}, \quad \forall i > k, \, \forall j \geq k+1.
		\]
	\end{enumerate}
	
	\subsection*{2. Обратный ход}
	После завершения прямого хода матрица \(A\) превращается в верхнюю треугольную матрицу. Решение системы находится методом подстановки, начиная с последнего уравнения:
	\[
	x_i = \frac{b_i - \sum_{j=i+1}^{n} a_{i,j} x_j}{a_{i,i}}, \quad i = n, n-1, \dots, 1.
	\]
	
	\subsection*{3. Учет перестановок}
	Если в процессе решения выполнялись перестановки столбцов, то решение \(\mathbf{x}\) необходимо переставить обратно в соответствии с первоначальным порядком столбцов.
	\newpage
	\subsection{Программная реализация}
	Для реализации этого алгоритма был выбран ЯП Python из-за простоты реализации подобных методов
	\begin{figure} % 'H' для размещения здесь
		
		\begin{lstlisting}[language=Python]
class Solver:
 def __init__(self, A: np.array, b: np.array):
  self._A = A.astype(float)
  self._n = A.shape[0]
  self._A_resolve = None
  self._b = b.astype(float)
  self._perm = list(range(self._n))
 def _find_max_elem(self, k: int):
  max_value = np.max(np.abs(self._A_resolve[k:, k:]))
  max_index = np.where(abs(self._A_resolve[k:, k:]) == max_value)
  return max_index[0][0] + k, max_index[1][0] + k
 def solveGaussWithAllMax(self):
  self._A_resolve = self._A.copy() 
  for k in range(self._n - 1):	
   max_row, max_col = divmod(np.abs(self._A_resolve[k:, k:]).argmax(), self._n - k)
   max_row += k
   max_col += k	
   if max_row != k:
    self._A_resolve[[k, max_row]] = self._A_resolve[[max_row, k]]
    self._b[[k, max_row]] = self._b[[max_row, k]]
   if max_col != k:
     self._A_resolve[:, [k, max_col]] = self._A_resolve[:, [max_col, k]]
     self._perm[k], self._perm[max_col] = self._perm[max_col], self._perm[k]
   for i in range(k + 1, self._n):
    factor = self._A_resolve[i, k] / self._A_resolve[k, k]
    self._A_resolve[i, k:] -= factor * self._A_resolve[k, k:]
    self._b[i] -= factor * self._b[k]
  x = np.zeros(self._n)
  for i in range(self._n - 1, -1, -1):
  x[i] = (self._b[i] - np.dot(self._A_resolve[i, i + 1:], x[i + 1:])) / self._A_resolve[i, i]
  x_final = np.zeros(self._n)
  for i, p in enumerate(self._perm):
  x_final[p] = x[i]




















  return x_final
		\end{lstlisting}
		
		\caption{Метод Гаусса с выборов ведущего элемента по всей матрице}
		% Метка для ссылки на блок
	\end{figure}
	
	\newpage
	
	\section{Заключение}
	\section{Приложение}
	
	
	
	\section{Решения теоретических задач}
	
	\subsection{Неравенство 1}
		\[
			\frac{1}{\sqrt{n}} N(A) \le ||A||_2 \le N(A), \text{ где } N(A) = \sqrt{\sum_{i, j} |a_{i,j}|^2}.
		\]
		
		Рассмотрим левую часть этого двойного неравенства:
		$$	\frac{1}{\sqrt{n}} N(A) \le ||A||_2$$
		
		
		$$||A||_2 ^ 2 = \max_i \lambda_i \ge \frac{1}{n} (\lambda_1 + \lambda_2 + \dots + \lambda_n) = \frac{1}{n} \text{Sp}(A* A) = \frac{1}{n} N(A) ^ 2.$$
		
		Взяв квадратные корни от правой и левой части, получим исходное неравенство.
		
		\[
			||A||_2 \ge \frac{1}{n} N(A), \eqno (1)
		\]
		
		Для правой части проведём аналогичные рассуждения:
		$$||A||_2 \le N(A)$$
		
		$$||A||_2^2 = \max_i \lambda_i \le \lambda_1 + \lambda_2 + \dots + \lambda_n = \text{Sp}(A* A) = N(A)^2.$$
		
		Взяв квадратные корни от правой и левой части, получим исходное неравенство.
		\[
			||A||_2 \le N(A). \eqno(2)
		\]
		
		Из неравенств (1) и (2) следует целевое двойное неравенство:
		\[
		\frac{1}{\sqrt{n}} N(A) \le ||A||_2 \le N(A).
		\]
		
	\subsection{Неравенство 2}
	
		\[
			||A||_2^2 \le ||A||_1 \cdot ||A||_\infty
		\]
		
		Рассмотрим квадрат второй матричной нормы:
		
		$$ \max_i \lambda(A) \le \sum_i \lambda_i = \text{Sp} (A* A) = \sum_{i,j} |a_{i,j}|^2$$
		
		Распишем правую сумму через неравенство Коши-Буняковского:
		$$\sum_{i,j} |a_{i,j}|^2 = \sum_{i,j} |a_{i,j}| \cdot |a_{i,j}| \le $$
		
		$$  \le \max_i \sum_j |a_{i,j}| \cdot \max_j \sum_i |a_{i,j}| = ||A||_1 \cdot ||A||_\infty$$
	\bibliographystyle{plain} % стиль библиографии, можно поменять на другой
	\bibliography{biblio} % используемый файл с источниками
	
\end{document}
